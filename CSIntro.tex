\section{Introduction}
% no \IEEEPARstart

    One major hurdle in developing robust applications is anticipating and dealing with edge conditions and faults that
    compromise and applications ability to run as intended. A common source of faults is the environment in which an
    application runs. The typical approach to eliminating faults in an application is the creation of a suite of tests that
    exercise the applications functionality and ensure that it always acts appropriately. In the case of issues that arise
    from an application's environment it is not possible to enumerate and test for an acceptable amount of faults due to the
    variety of hardware and software systems that make up the environment.

    This problem is of particular concern in the context of today's highly network dependent applications. With the
    explosion of mobile computing and software as a service products applications that are completely reliant on
    well-behaved network communication are increasingly common. As a result a great deal of emphasis must be placed on
    properly dealing with faults in these network communications. Unfortunately, identifying these faults is difficult due
    to the inability of the developer to replicate all possible application environments and exhaust all possible code
    paths.

    Existing white-box testing tools like DART and SAGE testing tools rely on static analysis and symbolic execution of
    code in order to determine sets of inputs that exercise particular code paths. DART symbolically executes an
    applications source code in order to facilitate its testing. This limits DART to operating only on programs written
    in C. SAGE improves on this strategy symbolically executing an application's x86 machine code instructions but this
    results in an architecture based rather than programming language based limitation. CrashSimulator avoids these
    issues by relying on transparent observation of native execution of an application.

    %% Does this need to be here? Am I just arguing against static analysis???
    Symbolic execution is computationally expensive and can result in deviations where the symbolic execution of code
    Doesn't accurately represent the actual execution of code.
    %% Is this a true sentence??
    Both tools focus on how the application under test behaves when given a particular set of inputs without taking the
    program's environment into account. DART assumes that external functions are side effect free (I.e.\ their execution
    does not modify program memory). Alternatively, CrashSimulator executes standard, non-instrumented versions of an
    application and transparently injects faults via the programs interaction with the environment in which it is
    running through system calls.

    %%% Need examples of black-box testing tools
    The alternative, black-box tools, have their own set of issues. They do not have an understanding of what an
    application is actually doing during execution which means they are only able to submit inputs and observe outputs.
    They cannot identify specifically the source of faults in an application.

    In light of these shortcomings, an alternative approach is needed to provide developers with confidence that their
    applications are sufficiently free of environment induced faults.  CrashSimulator is a tool designed to meet this
    need by providing automated identification of environment-induced faults through simulated deviations from the
    normal flow of system calls an application makes to interact with its environment.

    CrashSimulator is able to test applications in a completely black-box manner. It utilizes system call traces
    collected during execution of the program to identify points for fault injection and injects the faults on
    subsequent runs using ptrace to hook the applications execution and access its memory and registers. It then
    evaluates whether the program successfully handled the injected fault or not and reports these results to the
    developer.


