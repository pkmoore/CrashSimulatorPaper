\section{Introduction}

    One major hurdle in developing robust applications is getting them to run reliably across all of the environments on
    which they may be installed. In an increasingly diverse software ecosystem it is not uncommon for problematic
    differences to appear between these environments. Whether it happens because of a lack of adherence to standards or
    developer error, the result is the same: software written to run in multiple environments falls victim to the
    differences between those environments. These can easily go unnoticed due to the infeasibility of testing software
    across every combination of hardware and software on which it is expected to run.  The typical approach to
    eliminating faults of any kind in an application is to test its functionality through a number of tests to ensure it
    always acts as expected. In the case of issues that arise from an application's environment it is not possible to
    enumerate and test for an acceptable amount of faults due to the variety of hardware and software systems that can
    make up the environment.

    This problem is of particular concern in the context of today's highly network dependent applications. With the
    explosion of mobile computing and services provided via the internet, applications that are completely reliant on
    well-behaved network communication are increasingly common. As a result a great deal of emphasis must be placed on
    properly dealing with faults in these network communications. Unfortunately, identifying these faults is difficult
    due to the inability of the developer to replicate all possible application environments, exhaust all possible
    code paths and to ensure consistent performance.

    In an attempt to remedy this situation, a wide variety of automated testing tools have been developed. These tools
    typically follow either a black-box or white-box approach. In the case of white-box testing, the tools attempt to
    gain an understanding of the application by analyzing its code, with the goal of generating inputs that most
    completely execute the application's code paths. In the case of black-box testing, the tools simply generate
    slightly abnormal inputs and present them to the application in order to achieve a similar goal. Both of these
    techniques have their advantages and disadvantages.

    CrashSimulator is a single utility that can combine the best qualities of the above techniques and would provide
    developers with the confidence that their applications are reasonably free of environment-induced faults.
    CrashSimulator meets this need by providing automated identification of environment-induced faults through simulated
    deviations from the normal flow of system calls an application makes when interacting with its environment. This
    allows a developer to test an application as if it was running in an environment different from the one in which it
    is being written. Like other white-box tools, CrashSimulator attempts to analyze an application in order to
    understand where it can inject faults to best test the application. However, unlike other tools it bypasses the
    application's source code entirely --- instead relying on system call traces. This reduces CrashSimulator's
    footprint by eliminating complex programming language analysis code while still allowing insight into the
    application's operation. Furthermore, because it bases its analysis on system call traces from the application
    running in a real environment this analysis takes into account the application's interactions with that environment.

    \emph{Everything above seems high level enough to be ok right now, I've outlined new technical details
      below. Removed material related to automatically identifying faults in other traces.}
    
    CrashSimulator's analysis relies on two concepts. The first is encoding bad system call patterns as deterministic
    computation models (i.e. DFA, PDA, etc.....) Encoding behavior in this way allows us to identify the correct point
    to inject failures \emph{wording bad... need to describe this better} into a given execution and to make an 
    assessment about whether the application responded to the injected failure \emph{wording...} correctly................

    The second is deterministic replay of the application under test's system call behavior. Replaying an earlier
    execution allows us to treat these anomalous executions like test cases. \emph{Wording... blah...}That is, they can be
    run in any environment simulating, from the application's perspective, the environment we want to test the application
    in............
    
    In short, CrashSimulator helps us find bugs faster by allowing to inject problematic behavior into an application at
    the system call level.  For example, we can make a call to rename() fail with the EXDEV errno value which should
    force well-behaved applications to either initiate a manual cross-device file copy procedure or fail out indicating
    an error.  If the application continues as if nothing is amiss, then it has failed to properly handle the anomaly
    (an ``unusual'' condition).

    If this process is repeated many times over with multiple anomaly types and checkers and it acts like a test suite.
    Because anomalies can be injected into any application that presents the appropriate opportunity, the ``tests'' that
    make up this ``suite'' are applicable across applications.  This allows CrashSimulator s user to construct a suite
    of tests built from all the applications they work on so it can be leveraged on any project  touch.

    CrashSimulator makes the following contributions: \emph{Do we need to rework contributions?}

    \begin{enumerate}
        \item{CrashSimulator is the first tool to use system call traces recorded in one environment as a basis for
            injecting faults into an application running in another environment.}
        \item{CrashSimulator is able to produce thousands of mutated system call traces per minute even when operating
            on large system call traces from complex applications.}
        \item{CrashSimulator is able to reliably identify faults that result from differences between an applications
            development environment and its deployment environment.}
        \item{CrashSimulator's efficacy can be validated by its ability to identify errors listed in the bug trackers of
            major projects and predicted by other testing tools.}
    \end{enumerate}
