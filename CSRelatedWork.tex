\section{Related Work}

CrashSimulator's trace analysis engine is based on NetCheck. NetCheck uses two strategies
    to identify potential fault areas from system call trace. The first is a model based simulation of the system calls
    relevant to network communication from the input trace. System calls are organized according to a POSIX socket API
    dependency graph and prioritized based on the order in which the system calls should be made in an ideal scenario.
    For example, a client application should not be making a \emph{connect} system call before it has set up its
    socket with the appropriate \emph{socket} system call. The model assumes that all system calls are atomic and that
    they cannot happen simultaneously. This allows a definite global order to be created.

    Once a global ordering is in place each system call is evaluated based on the previous system calls. Return values
    and parameters passed in are taken into account. If the system call is feasible it is accepted and the next system
    call is evaluated. If the system call is not feasible given the current system call context it is rejected and
    logged. These rejected system calls are treated as opportunities to inject faults during the next step of the
    process.
