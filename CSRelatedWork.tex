\section{Related Work} \label{sec:relatedwork}

\iffalse
While there is a vast literature on test
generation~\cite{ammann2008introduction, mcminn2004search,
  puasuareanu2009survey, dias2007survey}, much less work
has focused on issues of portability and testing whether software
behaves consistently in different environments.  Prior work on
CheckAPI~\cite{rasley2015detecting} and
NetCheck~\cite{Zhuang_NSDI_2014} begins to fill this gap and this paper
builds upon those results.
%
%\paragraph{Detecting Environmental Bugs.}
%
%NetCheck; CheckAPI; stub injection; detecting machine specific bugs
%(e.g. numeric/memory limitations); testing error handlers; \dots


Crash reproduction by test case mutation~\cite{DBLP:conf/sigsoft/XuanXM15}.

\fi


\noindent
{\bf Static analysis. }
Tools based on static analysis techniques such as abstract
interpretation, model checking, and symbolic execution have been used
successfully to detect bugs related to incorrect API usages. Examples
include SLAM~\cite{Ball_adecade, Ball:2002:SLP:503272.503274} and,
more recently, CORRAL~\cite{DBLP:conf/sigsoft/LalQ14} for conformance
checking of Windows device drivers against the Windows kernel API,
FindBugs~\cite{DBLP:conf/oopsla/HovemeyerP04} for detecting API usage
bugs in Java programs, FiSC~\cite{Musuvathi04modelchecking} for
finding bugs in TCP implemenations, and the Explode
system~\cite{Yang:2006:ELG:1298455.1298469} for detecting crash
recovery bugs in file system implementations. Unlike CrashSimulator,
these approaches depend on the availability of source or byte code and
are typically more prone to false alarms. On the other hand, static
analysis can provide stronger soundness guarantees.  Static analysis
has also been used to detect portability issues related to changes
between different versions of external components that a program
depends on~\cite{silakov2010improving, javacompliance-www}. Like
CrashSimulator, these address the application's interactions with its
environment, but this work focuses on testing the application's
response to anomalies, rather than on proving that the environments
behave as expected.

\iffalse
\noindent
{\bf Specification and run-time verification.}  Substantial work has
been done in validating API and protocol behaviors, e.g., finding
faults in the Linux TCP implementation,
SSH2 and RCP~\cite{Udrea:2008}, BGP
configuration~\cite{Feamster:2005}, and identifying network
vulnerabilities~\cite{ritchey-sp00}. 
\fi

\noindent
{\bf API protocol mining.}
There has been extensive work on mining source code to learn API
protocols and use them to detect common usage violations such as
missing method calls (see, e.g.,~\cite{mariani2007compatibility,
  DBLP:journals/ase/WasylkowskiZ11, DBLP:conf/icse/PradelJAG12,
  DBLP:journals/tosem/MonperrusM13,
  DBLP:conf/icse/JamrozikSZ16}). These techniques primarily target
object component interactions rather than system calls to generate
test suites. However, the techniques explored in these works could be
used to mine system call patterns in source code in order to learn
checkers that identify incorrect responses to environment
anomalies. So far, we have specified these checkers manually for
CrashSimulator.

\noindent {\bf Tracing and log mining.}  
Similar to API protocol mining, there has been substantive work on
using log files to detect anomalies~\cite{pinpoint,
  jiang_abnormal_trace_detection_icac_2005, xu2009detecting,
  lou2010mining2} and aid program understanding~\cite{yuan2010sherlog,
  beschastnikh_synoptic_fse_2011, csight_icse_2014}.
%In contrast to
%this work, our focus is on diagnosing network issues from logs
%of syscalls, though prior work on log mining can be used to
%expand our scope.
Khadke et al.~\cite{khadketransparent} introduced a performance
debugging approach that relies on system call tracing. Unlike this
prior work, our system does not assume synchronized clocks and
reconstructs a plausible global ordering.
CheckAPI~\cite{rasley2015detecting} and
NetCheck~\cite{Zhuang_NSDI_2014} use system call traces to diagnose an
application's violations of cross-platform portability that are
observed in the field. In contrast to this work, CrashSimulator uses
system call traces for a different purpose: mutating and replaying
traces to test an applications' responses to known environmental
anomalies before the application is deployed.


\noindent
{\bf Portability.}
Wrappers are commonly used to enable cross-platform portability of
APIs~\cite{bartolomeicompliance}. System call interposition is one
technique to generate such wrappers
automatically~\cite{Guo:2011:CUS:2002181.2002202} and has also been
used to detect and prevent security
violations~\cite{Hofmeyr:1998:IDU:1298081.1298084,
  Acharya:2000:MUP:1251306.1251307}.  CrashSimulator's goals are
complementary as it exploits environment anomalies discovered in
testing or analyzing one application to generate and execute tests for
another application.
%
Other works that target complementary classes of portability problems
include the detection of configuration-related
bugs~\cite{skoll:icse:2004, Yilmaz:issta:2004, Fouche:issta:2009,
  Kastner12, Nguyen14} and
cross-browser incompatibilities for web
applications~\cite{DBLP:conf/icsm/ChoudharyVO10, silakov2010improving,
  DBLP:conf/icse/Choudhary11, Mesbah:2011:ACC:1985793.1985870,
  DBLP:conf/icst/DallmeierP0MZ14}.


\noindent
{\bf Testing exception handling and conformance.}
A few researchers have developed testing techniques aimed at checking
whether programs respond appropriately to anomalous situations.  For
example, Fu et al. introduce data flow testing techniques that require
tests that cover paths from points at which exceptions are thrown to
points at which they are handled in Java code, in order to test
whether programs respond correctly to exceptional situations that the
programmer has anticipated~\cite{DBLP:journals/tse/FuMRW05}.  Koopman
and DeVale developed a system to detect bugs in error handling code
related to calls to POSIX functions~\cite{Koopman00theexception}.
%
Other approaches to conformance checking of POSIX operations use model
based testing~\cite{Dadeau:2008:CSM:1433121.1433137,Farchi02}.
%
Unlike these approaches, CrashSimulator does not exclusively
target error handling code or anomalies that only involve individual
system calls. However, such testing techniques can help identify
anomalies that can be added to our repository.


%\noindent
%{\bf Runtime verification:}
%Runtime verification (RV) 
%techniques~\cite{DBLP:conf/rv/2010, Liu:2007:WCC:1973430.1973449, Lu_ASPLOS_2006,
%Archer:2007:ICT:1236360.1236382, Verbowski_OSDI_2006, Tucek_SOSP_2007,
%Park_ASPLOS_2009,DBLP:journals/jlp/LeuckerS09}
%can detect violations of properties on specific executions
%but do not show that the software satisfies the specification for every
%possible input or on every possible execution.
%Many of these techniques find general violations of 
%properties, such as atomicity~\cite{Park_ASPLOS_2009, Verbowski_OSDI_2006}.
%Other RV techniques enable checking program-specific requirements 
%usually specified with formal languages, such as automata or logic
%formalisms~\cite{DBLP:conf/vmcai/BarringerGHS04, DBLP:conf/kbse/GiannakopoulouH01}.
%
%Many RV approaches instrument code to capture relevant
%events or application state and insert executable 
%assertions~\cite{Orso:2002:GSC:566172.566182, DBLP:conf/icse/ClauseO11,
%DBLP:books/sp/Liblit2007, Jin:2010:ISS:1932682.1869481, Barnett01spyingon, 
%DBLP:journals/jss/BarnettS03}.
%However, inserting pre- and post-conditions obscures the fact that the
%specification can be treated as a parallel construct to the
%implementation~\cite{Barnett01spyingon, DBLP:journals/jss/BarnettS03}.
%Instead, an architecture can be used for runtime verification of .NET
%components by running the model and the implementation side-by-side,
%comparing results at method boundaries~\cite{Barnett01spyingon,
%DBLP:journals/jss/BarnettS03}.
%CheckAPI does not require application instrumentation, assuming a tracing
%mechanism exists in the API~\cite{strace, Cappos_CCS_10}.

%Like CheckAPI, 
%several other (runtime and static) checking techniques
%allow the use of languages more familiar to 
%programmers. 
%The WiDS checker allows using a scripting language to specify properties of a
%distributed system~\cite{Liu:2007:WCC:1973430.1973449}.
%Contracts written in a C-like language can specify components for use
%in TinyOS applications~\cite{Archer:2007:ICT:1236360.1236382}.
%CheckAPI allows
%programmers to choose the language for PSI construction or simply to
%use an existing implementation. Lastly, work on deterministic 
%replay~\cite{Viennot13} enables record-replay techniques on multi-core
%systems and could help improve CheckAPI-MT performance.


\iffalse
\noindent
{\bf Application-specific fault detection.}
Pip~\cite{reynolds2006pip} and
Coctail~\cite{xue2012using} are distributed frameworks that enable developers to
construct application-specific models, which have proven effective at finding
detailed application flaws. However, to utilize these methods,
a knowledge of the nature of the failures needs to be acquired, and the
specific system properties must be specified. NetCheck
diagnoses application failures without application-specific models.
Khanna~\cite{khanna2007automated} identifies the source of failures using 
a rule base of allowed state transition paths.
%However, it requires specialized human-generated rules for each application. 
CrashSimulator leverages NetCheck's approach to simulate identified
anomalies in network behavior, file system behavior, etc. to test
applications other than the ones on which the anomalies were originally discovered.
\fi


%% The following is old stuff -- not sure how much should be integrated
%% into the above.
\begin{comment}
    \subsection{Existing Techniques}

    Existing tools can be roughly divided into two categories, black-box and white-box, based on the techniques they
    use to perform their testing. Black-box tools simply manipulate the inputs of the application under test and
    observe the resultant outputs. White-box tools, on the other hand, perform complex analysis of the application's
    source code in order to reason about what inputs are likely to produce interesting outputs. Each of these
    methodologies have their own advantages and disadvantages.

    White-box testing tools typically rely on a similar set of techniques, including constraint solving of branch
    statements in an application's code and symbolic execution of an application's code in order to generate inputs
    that optimally exercise the application's code paths. These techniques, while powerful, are not without their
    downsides. First, both techniques are computationally-expensive. Furthermore, symbolic execution can not always
    accurately represent actual execution and so there may be deviations in results. Similarly, efficiently solving
    a series of constraints in order to exercise a particular code path can be can be difficult to guarantee that a
    particular set of generated inputs will exercise the intended code path in many circumstances due to external
    dependencies that the tool cannot analyze. For example, a white-box testing tool cannot reliably generate inputs
    that are guaranteed to exercise a code path that relies on an operating system resource being available.
    Finally, white-box tools typically require that an application's source code be available which is not always
    the case. Even advanced white-box tools that analyze an application's machine code can be stymied in situations
    where an application's executable has been packed or encrypted.

    The alternative, black-box tools, have their own set of issues. They do not have an understanding of what an
    application is actually doing during execution which means they are only able to submit inputs and observe
    outputs.  The upside of this technique is simplicity. Black-box tools do not need the capability to understand
    and analyze an application's code which reduces their complexity immensely. Also, their testing process,
    mutating inputs and observing outputs, is computationally inexpensive. The downside of simplicity is that they
    cannot craft inputs with any sort of intelligence. This means that a great deal of time can be spent mutating
    inputs without much success in terms of bug identification. Also, they cannot identify specifically the source
    of faults in an application. They can only signal that a fault has occurred at some point during a test run.
    Furthermore, like white-box tools, these tools fail to take into account the environment in which the
    application is running.
    \subsection{White-box Tools}

        White-box testing has been an area of intense interest in recent writing. Microsoft's SAGE and Bell Labs' DART
        are two examples of such tools that take different approaches to the same overall white-box technique.

        \subsubsection{DART}

            DART is a white-box testing tool that supports testing of applications written in the C programming language. It
            is capable of generating a test harness for an application's functions by through static analysis of the
            application's source code. This harness is then used for two phases of testing. First, it performs random
            testing and observes the application's behavior. Based on this random testing and symbolic execution of the
            application's source code, DART generates a series of inputs that will be used in the second phase of testing.
            These inputs are designed to direct the application down specific execution paths, observing the programs
            behavior and reporting faults as they are identified. DART operates on the assumption that the functions being
            evaluated have no side-effects and that the application is able to interact appropriately with its environment.
            More information can be found in \textbf{\emph{PAPER TITLE HERE}}

        \subsubsection{SAGE}

            SAGE differs from many other white-box testing tools in that it analyzes a compiled application's machine code
            rather than the application's uncompiled source code. This allows SAGE to operate on applications that were
            compiled from a variety of programming languages. It first runs the application under test with a set of well
            formed inputs and records an instruction-level trace of the application's execution. Next, it analyzes this
            trace in order to identify constraints that guard different paths of execution. SAGE then solves these
            constraints and, based on these solutions, generates inputs that are able to exercise specific paths of
            execution.

    \subsection{Black-box tools}

        % TODO: Find examples of black box tools

    \subsection{Trace Analysis Tools}

        Much of CrashSimulator's work on system call trace analysis is based on previous work on NetCheck and CheckAPI

        \subsubsection{NetCheck}

            This implementation of CrashSimulator relies of NetCheck for system call trace analysis. NetCheck uses two
            strategies to identify potential fault areas from system call trace. The first is a model based simulation
            of the system calls relevant to network communication from the input trace. System calls are organized
            according to a POSIX socket API dependency graph and prioritized based on the order in which the system
            calls should be made in an ideal scenario.  For example, a client application should not be making a
            \emph{connect} system call before it has set up its socket with the appropriate \emph{socket} system call.
            The model assumes that all system calls are atomic and that they cannot happen simultaneously. This allows a
            definite global order to be created.

            Once a global ordering is in place each system call is evaluated based on the previous system calls. Return
            values and parameters passed in are taken into account. If the system call is feasible it is accepted and
            the next system call is evaluated. If the system call is not feasible given the current system call context
            it is rejected and logged. In addition, system calls that return a value indicating some sort of network
            failure are recorded. After all system calls have been evaluated a NetCheck attempts to diagnose the
            source of any errors encountered. It is this diagnosis that CrashSimulator uses when deciding where and how
            to mutate the ``ideal run'' system call trace it is operating on.

        \subsubsection{CheckAPI}

            % TODO: Expand this
            \textbf{\emph{This needs to be expanded}}
            CheckAPI attempts to identify
\end{comment}
