% TODO: Flesh this out a __LOT__ MORE
\section{Related Work}

    \subsection{White-box Tools}

        White-box testing has been an area of intense interest in recent writing. Microsoft's SAGE and Bell Labs' DART
        are two examples of such tools that take different approaches to the same overall white-box technique.

        \subsubsection{DART}

            DART is a white-box testing tool that supports testing of applications written in the C programming language. It
            is capable of generating a test harness for an application's functions by through static analysis of the
            application's source code. This harness is then used for two phases of testing. First, it performs random
            testing and observes the application's behavior. Based on this random testing and symbolic execution of the
            application's source code, DART generates a series of inputs that will be used in the second phase of testing.
            These inputs are designed to direct the application down specific execution paths, observing the programs
            behavior and reporting faults as they are identified. DART operates on the assumption that the functions being
            evaluated have no side-effects and that the application is able to interact appropriately with its environment.
            More information can be found in \textbf{\emph{PAPER TITLE HERE}}

        \subsubsection{SAGE}

            SAGE differs from many other white-box testing tools in that it analyzes a compiled application's machine code
            rather than the application's uncompiled source code. This allows SAGE to operate on applications that were
            compiled from a variety of programming languages. It first runs the application under test with a set of well
            formed inputs and records an instruction-level trace of the application's execution. Next, it analyzes this
            trace in order to identify constraints that guard different paths of execution. SAGE then solves these
            constraints and, based on these solutions, generates inputs that are able to exercise specific paths of
            execution.

    \subsection{Black-box tools}

        % TODO: Find examples of black box tools

    \subsection{Trace Analysis Tools}

        Much of CrashSimulator's work on system call trace analysis is based on previous work on NetCheck and CheckAPI

        \subsubsection{NetCheck}

            This implementation of CrashSimulator relies of NetCheck for system call trace analysis. NetCheck uses two
            strategies to identify potential fault areas from system call trace. The first is a model based simulation
            of the system calls relevant to network communication from the input trace. System calls are organized
            according to a POSIX socket API dependency graph and prioritized based on the order in which the system
            calls should be made in an ideal scenario.  For example, a client application should not be making a
            \emph{connect} system call before it has set up its socket with the appropriate \emph{socket} system call.
            The model assumes that all system calls are atomic and that they cannot happen simultaneously. This allows a
            definite global order to be created.

            Once a global ordering is in place each system call is evaluated based on the previous system calls. Return
            values and parameters passed in are taken into account. If the system call is feasible it is accepted and
            the next system call is evaluated. If the system call is not feasible given the current system call context
            it is rejected and logged. In addition, system calls that return a value indicating some sort of network
            failure are recorded. After all system calls have been evaluated a NetCheck attempts to diagnose the
            source of any errors encountered. It is this diagnosis that CrashSimulator uses when deciding where and how
            to mutate the ``ideal run'' system call trace it is operating on.

        \subsubsection{CheckAPI}

            % TODO: Expand this
            \textbf{\emph{This needs to be expanded}}
            CheckAPI attempts to identify
