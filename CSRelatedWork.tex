\section{Related Work} \label{sec:related}

While there is a vast literature on test
generation~\cite{ammann2008introduction, mcminn2004search,
  puasuareanu2009survey, dias2007survey}, relatively little prior work
has focused on issues of portability and testing whether software
behaves consistently in different environments.  Prior work on
CheckAPI~\cite{rasley2015detecting} and
NetCheck~\cite{Zhuang_NSDI_2014} begins to fill this gap and this paper
builds upon those results.
%
%\paragraph{Detecting Environmental Bugs.}
%
%NetCheck; CheckAPI; stub injection; detecting machine specific bugs
%(e.g. numeric/memory limitations); testing error handlers; \dots

\noindent
{\bf Static analysis. }
The Explode system uses model checking to check properties 
of file systems~\cite{Yang:2006:ELG:1298455.1298469}.
MC enables writing checkers that are compiled into
static checks for violations of system rules and security 
properties~\cite{Engler:2000:CSR:1251229.1251230, Ashcraft02usingprogrammer-written}.
SLAM~\cite{Ball_adecade, Ball:2002:SLP:503272.503274} 
uses model checking to check if an implementation
conforms to rules characterizing correct API usage;
it has been very successful in finding bugs
in Windows device drivers.
%By contrast, verification techniques like CheckAPI
%typically scale better and are less prone to false-positives,
%but they provide less strong guarantees of correctness.
Static analysis can integrate static API call verification into an API
implementation~\cite{spinellis2007framework}. This requires formally specifying each API call, whereas
a PSI specifies API behavior with a direct implementation. Silakov et
al.~\cite{silakov2010improving} statically detect portability problems
in Linux applications, while the Java API Compliance
Checker~\cite{javacompliance-www}
checks binary- and source-level compatibility of an application against
a Java library API. 
%These tools check static properties, while CheckAPI
%validates an API's runtime behavior.
Like CrashSimulator, these address the application's
interactions with the operating system and environment,
but the proposed work focuses on testing the application's
response to anomalies, rather than on proving that the
environments behave as expected.

\noindent
{\bf Portability.}
System call interposition can monitor executions of binaries to automatically
assemble packages of code needed to run applications on other
platforms~\cite{Guo:2011:CUS:2002181.2002202}.
%
Inferred behavioral models of
software component interactions can automatically generate compatibility test
suites~\cite{mariani2007compatibility}. Compliance-validated wrappers can be
used for principled migration of APIs across
platforms~\cite{bartolomeicompliance}.
In C, feature test macros can explicitly specify compatibility
with a specific OS.
These approaches are complementary to 
%CheckAPI,
CrashSimulator
whose goal is to 
%detect portability issues so that they can be fixed.
exploit anomalies discovered in testing or analyzing one
application to generate and execute tests for another application.
%
Skoll addresses portability bugs arising from configuration problems by
sampling the configuration space using distributed and diverse
machines~\cite{skoll:icse:2004}. Skoll, together with combinatorial
techniques, can systematically discover combinations of configuration options
(selected from a large but well-defined space) that cause a given test suite
to fail~\cite{Yilmaz:issta:2004, Fouche:issta:2009}. 
Recently, several tools have addressed the issue of cross-browser
incompatibilities for web applications~\cite{DBLP:conf/icse/Choudhary11,
DBLP:conf/icsm/ChoudharyVO10,
Mesbah:2011:ACC:1985793.1985870,silakov2010improving}.
%These involve computer vision algorithms to detect differences
%in page rendering, along with dynamic web crawling. Choudhary et
%al.~\cite{silakov2010improving}
% additionally compare web applications'
%possible states across browsers.
%Cross-browser
Incompatibilities detected by
these techniques can arise from differences in JavaScript
interpreters, along with other root causes.
%Integrating CheckAPI for JavaScript into such tools could
%improve their accuracy and help in fault diagnosis.
CrashSimulator helps developers to test 
whether applications deal with  such incompatibilities robustly.

\noindent
{\bf POSIX conformance testing.}
The Ballista system generates parameter values for POSIX functions aimed at
detecting implementation problems. In fifteen popular OS implementations,
Ballista revealed many bugs through crashes and abnormal task
termination~\cite{Koopman00theexception}.
%By contrast, CheckAPI detects violations that are the result of more
%subtle differences in OS behavior.
%
Model-based test and oracle
generation~\cite{Pretschner:2005:MT:1062455.1062636, JVCS07} can aid
conformance, and has been applied to POSIX file
systems~\cite{Dadeau:2008:CSM:1433121.1433137} and to other POSIX
aspects~\cite{Farchi02}.
%This work generates tests to validate library
%behavior and is complementary to CheckAPI, which finds violations at run-time.
Executing test suites across multiple configurations of a product or system,
known as variability execution~\cite{Nguyen14, Kastner12}, may reveal
differences between the configurations.
%By contrast, CheckAPI focuses on
%executing multiple implementations that share no or little code.
Such testing techniques can help identify
anomalies that can be added to our repository.

%\noindent
%{\bf Reliability via redundancy:}
%Developing and running multiple implementations of an application in
%parallel, comparing them at checkpoints, can achieve high reliability
%under the assumption that the implementations fail
%independently~\cite{DBLP:journals/tse/KnightL86}. Such reliability can
%be made optimal, in terms of the number of necessary
%implementations~\cite{Brun11icdcs}.
%CheckAPI's use of a PSI differs because a PSI represents an idealized execution 
%of the system, avoiding OS behavior to the greatest extent possible. For 
%example, the file system metadata in a CheckAPI PSI is likely to be entirely
%in memory. Thus, while n-version testing would have all implementations 
%heavily dependent on the same underlying OS behavior, a PSI is independent
%of the OS implementation and can discover violations caused by the OS. 
%
%\noindent
%{\bf Runtime verification:}
%Runtime verification (RV) 
%techniques~\cite{DBLP:conf/rv/2010, Liu:2007:WCC:1973430.1973449, Lu_ASPLOS_2006,
%Archer:2007:ICT:1236360.1236382, Verbowski_OSDI_2006, Tucek_SOSP_2007,
%Park_ASPLOS_2009,DBLP:journals/jlp/LeuckerS09}
%can detect violations of properties on specific executions
%but do not show that the software satisfies the specification for every
%possible input or on every possible execution.
%Many of these techniques find general violations of 
%properties, such as atomicity~\cite{Park_ASPLOS_2009, Verbowski_OSDI_2006}.
%Other RV techniques enable checking program-specific requirements 
%usually specified with formal languages, such as automata or logic
%formalisms~\cite{DBLP:conf/vmcai/BarringerGHS04, DBLP:conf/kbse/GiannakopoulouH01}.
%
%Many RV approaches instrument code to capture relevant
%events or application state and insert executable 
%assertions~\cite{Orso:2002:GSC:566172.566182, DBLP:conf/icse/ClauseO11,
%DBLP:books/sp/Liblit2007, Jin:2010:ISS:1932682.1869481, Barnett01spyingon, 
%DBLP:journals/jss/BarnettS03}.
%However, inserting pre- and post-conditions obscures the fact that the
%specification can be treated as a parallel construct to the
%implementation~\cite{Barnett01spyingon, DBLP:journals/jss/BarnettS03}.
%Instead, an architecture can be used for runtime verification of .NET
%components by running the model and the implementation side-by-side,
%comparing results at method boundaries~\cite{Barnett01spyingon,
%DBLP:journals/jss/BarnettS03}.
%CheckAPI does not require application instrumentation, assuming a tracing
%mechanism exists in the API~\cite{strace, Cappos_CCS_10}.

%Like CheckAPI, 
%several other (runtime and static) checking techniques
%allow the use of languages more familiar to 
%programmers. 
%The WiDS checker allows using a scripting language to specify properties of a
%distributed system~\cite{Liu:2007:WCC:1973430.1973449}.
%Contracts written in a C-like language can specify components for use
%in TinyOS applications~\cite{Archer:2007:ICT:1236360.1236382}.
%CheckAPI allows
%programmers to choose the language for PSI construction or simply to
%use an existing implementation. Lastly, work on deterministic 
%replay~\cite{Viennot13} enables record-replay techniques on multi-core
%systems and could help improve CheckAPI-MT performance.


%\noindent
%{\bf Blackbox diagnosis.}  Aguilera et al.\ introduced an important
%blackbox approach to debugging distributed
%systems~\cite{aguilera-sosp03}. In this approach,
%observations of a distributed executions are used to infer causality,
%dependencies, and other characteristics of the system. This approach
%was relaxed in later work to produce more informative and
%application-specific results in Magpie~\cite{magpie}
%and X-Trace~\cite{fonseca2007x}. 
%This prior work focuses on tracking request flows through a distributed
%system. BorderPatrol~\cite{koskinen_borderpatrol_eurosys_2008} is another
%approach that traces requests among binary modules.
%In contrast, ours is a blackbox approach to diagnosing
%network issues in applications.
%
%
%
%\noindent
%{\bf Ordering events in a distributed setting.}
%The happens-before relation logically
%orders events in a distributed
%system~\cite{lamport_clocks_cacm_1978}. This relation can be realized
%with vector time, which produces a partial ordering of events in the
%system~\cite{fidge_vector_clocks_1988,
%  mattern_vector_clocks_1989}. Vector time requires non-trivial
%instrumentation of the application. We will reconstruct a
%plausible order of the captured syscalls through heuristics, without
%modifying the application.
%
%Globally synchronized clocks can order events across
%hosts. However, over 90\% of syscall invocations we observed completed
%in less than 0.1~ms. Achieving synchronization at a granularity that
%is sufficient to order syscalls at hosts on a LAN is expensive and
%difficult. 

\noindent
{\bf Tracing and Log mining.}  Prior work that uses dynamically captured logs
of a program's execution is extensive and includes work on detecting
anomalies~\cite{pinpoint, jiang_abnormal_trace_detection_icac_2005,
  xu2009detecting, lou2010mining2}, linking logs and source
code~\cite{yuan2010sherlog}, identifying performance
bugs~\cite{sambasivan_req_flow_nsdi_2011, perf_skeletons_ipdps_2008},
and generating models to support system
understanding~\cite{beschastnikh_synoptic_fse_2011, csight_icse_2014}.
%In contrast to
%this work, our focus is on diagnosing network issues from logs
%of syscalls, though prior work on log mining can be used to
%expand our scope.
Khadke et al.~\cite{khadketransparent} introduced a performance
debugging approach that relies on system call tracing. Unlike this
prior work, our proposed system does not assume synchronized clocks and
reconstructs a plausible global ordering.
In contrast to this work, CrashSimulator uses system call traces
for a different purpose: mutating and replaying traces to test
applications' responses to environmental anomalies.

%\noindent
%{\bf  Debugging distributed systems.} Techniques for debugging
%distributed systems are relevant to our proposed tool.
%Many tools exist for run-time checking
%of distributed systems. These tools monitor a system's execution and
%check for specific property violations~\cite{reynolds2006pip,
%  geels_friday_nsdi_2007, liu_d3s_nsdi_2008, yabandeh2009crystalball, 
%  dao_live_dist_sys_debugging_cc_2009}. Ours is a more
%light-weight approach to diagnose issue observed through the
%syscall interface. 

\noindent
{\bf Specification and run-time verification.}  Substantial work has
been done in validating API and protocol behaviors, e.g., finding
faults in Linux TCP implementation~\cite{Musuvathi04modelchecking},
SSH2 and RCP~\cite{Udrea:2008}, BGP
configuration~\cite{Feamster:2005}, and identifying network
vulnerabilities~\cite{ritchey-sp00}. 

\noindent
{\bf Application-specific fault detection.}
Pip~\cite{reynolds2006pip} and
Coctail~\cite{xue2012using} are distributed frameworks that enable developers to
construct application-specific models, which have proven effective at finding
detailed application flaws. However, to utilize these methods,
a knowledge of the nature of the failures needs to be acquired, and the
specific system properties must be specified. NetCheck
diagnoses application failures without application-specific models.
Khanna~\cite{khanna2007automated} identifies the source of failures using 
a rule base of allowed state transition paths.
%However, it requires specialized human-generated rules for each application. 
CrashSimulator leverages NetCheck's approach to simulate identified
anomalies in network behavior, file system behavior, etc. to test
applications other than the ones on which the anomalies were originally discovered.



%% The following is old stuff -- not sure how much should be integrated
%% into the above.
\begin{comment}
    \subsection{Existing Techniques}

    Existing tools can be roughly divided into two categories, black-box and white-box, based on the techniques they
    use to perform their testing. Black-box tools simply manipulate the inputs of the application under test and
    observe the resultant outputs. White-box tools, on the other hand, perform complex analysis of the application's
    source code in order to reason about what inputs are likely to produce interesting outputs. Each of these
    methodologies have their own advantages and disadvantages.

    White-box testing tools typically rely on a similar set of techniques, including constraint solving of branch
    statements in an application's code and symbolic execution of an application's code in order to generate inputs
    that optimally exercise the application's code paths. These techniques, while powerful, are not without their
    downsides. First, both techniques are computationally-expensive. Furthermore, symbolic execution can not always
    accurately represent actual execution and so there may be deviations in results. Similarly, efficiently solving
    a series of constraints in order to exercise a particular code path can be can be difficult to guarantee that a
    particular set of generated inputs will exercise the intended code path in many circumstances due to external
    dependencies that the tool cannot analyze. For example, a white-box testing tool cannot reliably generate inputs
    that are guaranteed to exercise a code path that relies on an operating system resource being available.
    Finally, white-box tools typically require that an application's source code be available which is not always
    the case. Even advanced white-box tools that analyze an application's machine code can be stymied in situations
    where an application's executable has been packed or encrypted.

    The alternative, black-box tools, have their own set of issues. They do not have an understanding of what an
    application is actually doing during execution which means they are only able to submit inputs and observe
    outputs.  The upside of this technique is simplicity. Black-box tools do not need the capability to understand
    and analyze an application's code which reduces their complexity immensely. Also, their testing process,
    mutating inputs and observing outputs, is computationally inexpensive. The downside of simplicity is that they
    cannot craft inputs with any sort of intelligence. This means that a great deal of time can be spent mutating
    inputs without much success in terms of bug identification. Also, they cannot identify specifically the source
    of faults in an application. They can only signal that a fault has occurred at some point during a test run.
    Furthermore, like white-box tools, these tools fail to take into account the environment in which the
    application is running.
    \subsection{White-box Tools}

        White-box testing has been an area of intense interest in recent writing. Microsoft's SAGE and Bell Labs' DART
        are two examples of such tools that take different approaches to the same overall white-box technique.

        \subsubsection{DART}

            DART is a white-box testing tool that supports testing of applications written in the C programming language. It
            is capable of generating a test harness for an application's functions by through static analysis of the
            application's source code. This harness is then used for two phases of testing. First, it performs random
            testing and observes the application's behavior. Based on this random testing and symbolic execution of the
            application's source code, DART generates a series of inputs that will be used in the second phase of testing.
            These inputs are designed to direct the application down specific execution paths, observing the programs
            behavior and reporting faults as they are identified. DART operates on the assumption that the functions being
            evaluated have no side-effects and that the application is able to interact appropriately with its environment.
            More information can be found in \textbf{\emph{PAPER TITLE HERE}}

        \subsubsection{SAGE}

            SAGE differs from many other white-box testing tools in that it analyzes a compiled application's machine code
            rather than the application's uncompiled source code. This allows SAGE to operate on applications that were
            compiled from a variety of programming languages. It first runs the application under test with a set of well
            formed inputs and records an instruction-level trace of the application's execution. Next, it analyzes this
            trace in order to identify constraints that guard different paths of execution. SAGE then solves these
            constraints and, based on these solutions, generates inputs that are able to exercise specific paths of
            execution.

    \subsection{Black-box tools}

        % TODO: Find examples of black box tools

    \subsection{Trace Analysis Tools}

        Much of CrashSimulator's work on system call trace analysis is based on previous work on NetCheck and CheckAPI

        \subsubsection{NetCheck}

            This implementation of CrashSimulator relies of NetCheck for system call trace analysis. NetCheck uses two
            strategies to identify potential fault areas from system call trace. The first is a model based simulation
            of the system calls relevant to network communication from the input trace. System calls are organized
            according to a POSIX socket API dependency graph and prioritized based on the order in which the system
            calls should be made in an ideal scenario.  For example, a client application should not be making a
            \emph{connect} system call before it has set up its socket with the appropriate \emph{socket} system call.
            The model assumes that all system calls are atomic and that they cannot happen simultaneously. This allows a
            definite global order to be created.

            Once a global ordering is in place each system call is evaluated based on the previous system calls. Return
            values and parameters passed in are taken into account. If the system call is feasible it is accepted and
            the next system call is evaluated. If the system call is not feasible given the current system call context
            it is rejected and logged. In addition, system calls that return a value indicating some sort of network
            failure are recorded. After all system calls have been evaluated a NetCheck attempts to diagnose the
            source of any errors encountered. It is this diagnosis that CrashSimulator uses when deciding where and how
            to mutate the ``ideal run'' system call trace it is operating on.

        \subsubsection{CheckAPI}

            % TODO: Expand this
            \textbf{\emph{This needs to be expanded}}
            CheckAPI attempts to identify
\end{comment}
