%  -- What do you mean by environments?  Are you trying to detect bugs that would occur if an application was run on a different operating system?  The same operating system (Windows, Linux, etc) but different release?  Different kinds of filesystems (USB vs NTFS vs NFS?)?  *This is important: the reader needs to know extremely soon how big a problem we're tackling!*  

\paragraph{What is an ``Environment''}

For the purposes of this work an application's environment consists of two sets of entities.  The first is the
application's software dependencies on a given system.  For example, multiplatform applications intended for Unix-like
environments typically depend on a POSIX implementation to operate on both Linux and Apple's OS X.  The second set is
resources the application interacts with for the purposes of gathering the data it needs for processing.  Examples
include files on a filesystem from which the application needs to read or hosts on a network with which the application
exchanges data.


\paragraph{Why system call traces?}

System call traces record every system call an application makes including its parameters, return value, and data copied
into application buffers.  For example, a system call trace entry for a read() call can tell us what file descriptor was
read from, whether or not the call was successful (via its return value) and what data was copied into the
application-specified buffer by the kernel in as a result of the system call being fulfilled.  Having this information
allows us to construct a tool that can simulate all of these system call level operations completely (our replay tool)
in order to achieve the advantage that deterministic replay provides.
