%% This style is provided for the ICSE 2015 main conference,
%% ICSE 2015 co-located events, and ICSE 2015 workshops.

%% bare_conf_ICSE15.tex
%% V1.4
%% 2014/05/22


%% This is a skeleton file demonstrating the use of IEEEtran.cls
%% (requires IEEEtran.cls version 1.7 or later) with an IEEE conference paper.
%%
%% Support sites:
%% http://www.michaelshell.org/tex/ieeetran/
%% http://www.ctan.org/tex-archive/macros/latex/contrib/IEEEtran/
%% and
%% http://www.ieee.org/

%%*************************************************************************
%% Legal Notice:
%% This code is offered as-is without any warranty either expressed or
%% implied; without even the implied warranty of MERCHANTABILITY or
%% FITNESS FOR A PARTICULAR PURPOSE!
%% User assumes all risk.
%% In no event shall IEEE or any contributor to this code be liable for
%% any damages or losses, including, but not limited to, incidental,
%% consequential, or any other damages, resulting from the use or misuse
%% of any information contained here.
%%
%% All comments are the opinions of their respective authors and are not
%% necessarily endorsed by the IEEE.
%%
%% This work is distributed under the LaTeX Project Public License (LPPL)
%% ( http://www.latex-project.org/ ) version 1.3, and may be freely used,
%% distributed and modified. A copy of the LPPL, version 1.3, is included
%% in the base LaTeX documentation of all distributions of LaTeX released
%% 2003/12/01 or later.
%% Retain all contribution notices and credits.
%% ** Modified files should be clearly indicated as such, including  **
%% ** renaming them and changing author support contact information. **
%%
%% File list of work: IEEEtran.cls, IEEEtran_HOWTO.pdf, bare_adv.tex,
%%                    bare_conf.tex, bare_jrnl.tex, bare_jrnl_compsoc.tex
%%*************************************************************************

% *** Authors should verify (and, if needed, correct) their LaTeX system  ***
% *** with the testflow diagnostic prior to trusting their LaTeX platform ***
% *** with production work. IEEE's font choices can trigger bugs that do  ***
% *** not appear when using other class files.                            ***
% The testflow support page is at:
% http://www.michaelshell.org/tex/testflow/



% Note that the a4paper option is mainly intended so that authors in
% countries using A4 can easily print to A4 and see how their papers will
% look in print - the typesetting of the document will not typically be
% affected with changes in paper size (but the bottom and side margins will).
% Use the testflow package mentioned above to verify correct handling of
% both paper sizes by the user's LaTeX system.
%
% Also note that the "draftcls" or "draftclsnofoot", not "draft", option
% should be used if it is desired that the figures are to be displayed in
% draft mode.
%
\documentclass[conference]{IEEEtran}
\usepackage{booktabs}
\usepackage{multirow}
\usepackage{tabu}
\usepackage{float}
\usepackage{graphicx}
\usepackage{caption}
\usepackage[table,usenames,dvipsnames]{xcolor}
\usepackage{peanutgallery}
%
% If IEEEtran.cls has not been installed into the LaTeX system files,
% manually specify the path to it like:
% \documentclass[conference]{../sty/IEEEtran}





% Some very useful LaTeX packages include:
% (uncomment the ones you want to load)


% *** MISC UTILITY PACKAGES ***
%
%\usepackage{ifpdf}
% Heiko Oberdiek's ifpdf.sty is very useful if you need conditional
% compilation based on whether the output is pdf or dvi.
% usage:
% \ifpdf
%   % pdf code
% \else
%   % dvi code
% \fi
% The latest version of ifpdf.sty can be obtained from:
% http://www.ctan.org/tex-archive/macros/latex/contrib/oberdiek/
% Also, note that IEEEtran.cls V1.7 and later provides a builtin
% \ifCLASSINFOpdf conditional that works the same way.
% When switching from latex to pdflatex and vice-versa, the compiler may
% have to be run twice to clear warning/error messages.






% *** CITATION PACKAGES ***
%
%\usepackage{cite}
% cite.sty was written by Donald Arseneau
% V1.6 and later of IEEEtran pre-defines the format of the cite.sty package
% \cite{} output to follow that of IEEE. Loading the cite package will
% result in citation numbers being automatically sorted and properly
% "compressed/ranged". e.g., [1], [9], [2], [7], [5], [6] without using
% cite.sty will become [1], [2], [5]--[7], [9] using cite.sty. cite.sty's
% \cite will automatically add leading space, if needed. Use cite.sty's
% noadjust option (cite.sty V3.8 and later) if you want to turn this off.
% cite.sty is already installed on most LaTeX systems. Be sure and use
% version 4.0 (2003-05-27) and later if using hyperref.sty. cite.sty does
% not currently provide for hyperlinked citations.
% The latest version can be obtained at:
% http://www.ctan.org/tex-archive/macros/latex/contrib/cite/
% The documentation is contained in the cite.sty file itself.






% *** GRAPHICS RELATED PACKAGES ***
%
\ifCLASSINFOpdf{}
  % \usepackage[pdftex]{graphicx}
  % declare the path(s) where your graphic files are
  % \graphicspath{{../pdf/}{../jpeg/}}
  % and their extensions so you won't have to specify these with
  % every instance of \includegraphics
  % \DeclareGraphicsExtensions{.pdf,.jpeg,.png}
\else
  % or other class option (dvipsone, dvipdf, if not using dvips). graphicx
  % will default to the driver specified in the system graphics.cfg if no
  % driver is specified.
  % \usepackage[dvips]{graphicx}
  % declare the path(s) where your graphic files are
  % \graphicspath{{../eps/}}
  % and their extensions so you won't have to specify these with
  % every instance of \includegraphics
  % \DeclareGraphicsExtensions{.eps}
\fi
% graphicx was written by David Carlisle and Sebastian Rahtz. It is
% required if you want graphics, photos, etc. graphicx.sty is already
% installed on most LaTeX systems. The latest version and documentation can
% be obtained at:
% http://www.ctan.org/tex-archive/macros/latex/required/graphics/
% Another good source of documentation is "Using Imported Graphics in
% LaTeX2e" by Keith Reckdahl which can be found as epslatex.ps or
% epslatex.pdf at: http://www.ctan.org/tex-archive/info/
%
% latex, and pdflatex in dvi mode, support graphics in encapsulated
% postscript (.eps) format. pdflatex in pdf mode supports graphics
% in .pdf, .jpeg, .png and .mps (metapost) formats. Users should ensure
% that all non-photo figures use a vector format (.eps, .pdf, .mps) and
% not a bitmapped formats (.jpeg, .png). IEEE frowns on bitmapped formats
% which can result in "jaggedy"/blurry rendering of lines and letters as
% well as large increases in file sizes.
%
% You can find documentation about the pdfTeX application at:
% http://www.tug.org/applications/pdftex





% *** MATH PACKAGES ***
%
%\usepackage[cmex10]{amsmath}
% A popular package from the American Mathematical Society that provides
% many useful and powerful commands for dealing with mathematics. If using
% it, be sure to load this package with the cmex10 option to ensure that
% only type 1 fonts will utilized at all point sizes. Without this option,
% it is possible that some math symbols, particularly those within
% footnotes, will be rendered in bitmap form which will result in a
% document that can not be IEEE Xplore compliant!
%
% Also, note that the amsmath package sets \interdisplaylinepenalty to 10000
% thus preventing page breaks from occurring within multiline equations. Use:
%\interdisplaylinepenalty=2500
% after loading amsmath to restore such page breaks as IEEEtran.cls normally
% does. amsmath.sty is already installed on most LaTeX systems. The latest
% version and documentation can be obtained at:
% http://www.ctan.org/tex-archive/macros/latex/required/amslatex/math/





% *** SPECIALIZED LIST PACKAGES ***
%
%\usepackage{algorithmic}
% algorithmic.sty was written by Peter Williams and Rogerio Brito.
% This package provides an algorithmic environment fo describing algorithms.
% You can use the algorithmic environment in-text or within a figure
% environment to provide for a floating algorithm. Do NOT use the algorithm
% floating environment provided by algorithm.sty (by the same authors) or
% algorithm2e.sty (by Christophe Fiorio) as IEEE does not use dedicated
% algorithm float types and packages that provide these will not provide
% correct IEEE style captions. The latest version and documentation of
% algorithmic.sty can be obtained at:
% http://www.ctan.org/tex-archive/macros/latex/contrib/algorithms/
% There is also a support site at:
% http://algorithms.berlios.de/index.html
% Also of interest may be the (relatively newer and more customizable)
% algorithmicx.sty package by Szasz Janos:
% http://www.ctan.org/tex-archive/macros/latex/contrib/algorithmicx/




% *** ALIGNMENT PACKAGES ***
%
%\usepackage{array}
% Frank Mittelbach's and David Carlisle's array.sty patches and improves
% the standard LaTeX2e array and tabular environments to provide better
% appearance and additional user controls. As the default LaTeX2e table
% generation code is lacking to the point of almost being broken with
% respect to the quality of the end results, all users are strongly
% advised to use an enhanced (at the very least that provided by array.sty)
% set of table tools. array.sty is already installed on most systems. The
% latest version and documentation can be obtained at:
% http://www.ctan.org/tex-archive/macros/latex/required/tools/


%\usepackage{mdwmath}
%\usepackage{mdwtab}
% Also highly recommended is Mark Wooding's extremely powerful MDW tools,
% especially mdwmath.sty and mdwtab.sty which are used to format equations
% and tables, respectively. The MDWtools set is already installed on most
% LaTeX systems. The lastest version and documentation is available at:
% http://www.ctan.org/tex-archive/macros/latex/contrib/mdwtools/


% IEEEtran contains the IEEEeqnarray family of commands that can be used to
% generate multiline equations as well as matrices, tables, etc., of high
% quality.


%\usepackage{eqparbox}
% Also of notable interest is Scott Pakin's eqparbox package for creating
% (automatically sized) equal width boxes - aka "natural width parboxes".
% Available at:
% http://www.ctan.org/tex-archive/macros/latex/contrib/eqparbox/





% *** SUBFIGURE PACKAGES ***
%\usepackage[tight,footnotesize]{subfigure}
% subfigure.sty was written by Steven Douglas Cochran. This package makes it
% easy to put subfigures in your figures. e.g., "Figure 1a and 1b". For IEEE
% work, it is a good idea to load it with the tight package option to reduce
% the amount of white space around the subfigures. subfigure.sty is already
% installed on most LaTeX systems. The latest version and documentation can
% be obtained at:
% http://www.ctan.org/tex-archive/obsolete/macros/latex/contrib/subfigure/
% subfigure.sty has been superceeded by subfig.sty.



%\usepackage[caption=false]{caption}
%\usepackage[font=footnotesize]{subfig}
% subfig.sty, also written by Steven Douglas Cochran, is the modern
% replacement for subfigure.sty. However, subfig.sty requires and
% automatically loads Axel Sommerfeldt's caption.sty which will override
% IEEEtran.cls handling of captions and this will result in nonIEEE style
% figure/table captions. To prevent this problem, be sure and preload
% caption.sty with its "caption=false" package option. This is will preserve
% IEEEtran.cls handing of captions. Version 1.3 (2005/06/28) and later
% (recommended due to many improvements over 1.2) of subfig.sty supports
% the caption=false option directly:
%\usepackage[caption=false,font=footnotesize]{subfig}
%
% The latest version and documentation can be obtained at:
% http://www.ctan.org/tex-archive/macros/latex/contrib/subfig/
% The latest version and documentation of caption.sty can be obtained at:
% http://www.ctan.org/tex-archive/macros/latex/contrib/caption/




% *** FLOAT PACKAGES ***
%
%\usepackage{fixltx2e}
% fixltx2e, the successor to the earlier fix2col.sty, was written by
% Frank Mittelbach and David Carlisle. This package corrects a few problems
% in the LaTeX2e kernel, the most notable of which is that in current
% LaTeX2e releases, the ordering of single and double column floats is not
% guaranteed to be preserved. Thus, an unpatched LaTeX2e can allow a
% single column figure to be placed prior to an earlier double column
% figure. The latest version and documentation can be found at:
% http://www.ctan.org/tex-archive/macros/latex/base/



%\usepackage{stfloats}
% stfloats.sty was written by Sigitas Tolusis. This package gives LaTeX2e
% the ability to do double column floats at the bottom of the page as well
% as the top. (e.g., "\begin{figure*}[!b]" is not normally possible in
% LaTeX2e). It also provides a command:
%\fnbelowfloat
% to enable the placement of footnotes below bottom floats (the standard
% LaTeX2e kernel puts them above bottom floats). This is an invasive package
% which rewrites many portions of the LaTeX2e float routines. It may not work
% with other packages that modify the LaTeX2e float routines. The latest
% version and documentation can be obtained at:
% http://www.ctan.org/tex-archive/macros/latex/contrib/sttools/
% Documentation is contained in the stfloats.sty comments as well as in the
% presfull.pdf file. Do not use the stfloats baselinefloat ability as IEEE
% does not allow \baselineskip to stretch. Authors submitting work to the
% IEEE should note that IEEE rarely uses double column equations and
% that authors should try to avoid such use. Do not be tempted to use the
% cuted.sty or midfloat.sty packages (also by Sigitas Tolusis) as IEEE does
% not format its papers in such ways.





% *** PDF, URL AND HYPERLINK PACKAGES ***
%
%\usepackage{url}
% url.sty was written by Donald Arseneau. It provides better support for
% handling and breaking URLs. url.sty is already installed on most LaTeX
% systems. The latest version can be obtained at:
% http://www.ctan.org/tex-archive/macros/latex/contrib/misc/
% Read the url.sty source comments for usage information. Basically,
% \url{my_url_here}.





% *** Do not adjust lengths that control margins, column widths, etc. ***
% *** Do not use packages that alter fonts (such as pslatex).         ***
% There should be no need to do such things with IEEEtran.cls V1.6 and later.
% (Unless specifically asked to do so by the journal or conference you plan
% to submit to, of course. )


% correct bad hyphenation here
\hyphenation{op-tical net-works semi-conduc-tor}

% So Justin can comment in the text
\usepackage{ifthen}
\usepackage[normalem]{ulem} % for \sout
\usepackage{xcolor}
\let\lneq\undefined  % removes amssymb conflict with some packages
\usepackage{amssymb}
\newcommand{\ra}{$\rightarrow$}
\newboolean{showedits}
\setboolean{showedits}{true} % toggle to show or hide edits
\ifthenelse{\boolean{showedits}}
{
	\newcommand{\ugh}[1]{\textcolor{red}{\uwave{#1}}} % please rephrase
	\newcommand{\ins}[1]{\textcolor{blue}{\uline{#1}}} % please insert
	\newcommand{\del}[1]{\textcolor{red}{\sout{#1}}} % please delete
	\newcommand{\chg}[2]{\textcolor{red}{\sout{#1}}{\ra}\textcolor{blue}{\uline{#2}}} % please change
}{
	\newcommand{\ugh}[1]{#1} % please rephrase
	\newcommand{\ins}[1]{#1} % please insert
	\newcommand{\del}[1]{} % please delete
	\newcommand{\chg}[2]{#2}
}

\newboolean{showcomments}
%\setboolean{showcomments}{true}
\setboolean{showcomments}{true}
\newcommand{\id}[1]{$-$Id: scgPaper.tex 32478 2010-04-29 09:11:32Z oscar $-$}
\newcommand{\yellowbox}[1]{\fcolorbox{gray}{yellow}{\bfseries\sffamily\scriptsize#1}}
\newcommand{\triangles}[1]{{\sf\small$\blacktriangleright$\textit{#1}$\blacktriangleleft$}}
\ifthenelse{\boolean{showcomments}}
%{\newcommand{\nb}[2]{{\yellowbox{#1}\triangles{#2}}}
{\newcommand{\nbc}[3]{
 {\colorbox{#3}{\bfseries\sffamily\scriptsize\textcolor{white}{#1}}}
 {\textcolor{#3}{\sf\small$\blacktriangleright$\textit{#2}$\blacktriangleleft$}}}
 \newcommand{\version}{\emph{\scriptsize\id}}}
{\newcommand{\nbc}[3]{}
 \renewcommand{\ugh}[1]{#1} % please rephrase
 \renewcommand{\ins}[1]{#1} % please insert
 \renewcommand{\del}[1]{} % please delete
 \renewcommand{\chg}[2]{#2} % please change
 \newcommand{\version}{}}
\newcommand{\nb}[2]{\nbc{#1}{#2}{orange}}

\definecolor{jccolor}{rgb}{0.2,0.4,0.6}
\definecolor{pfcolor}{rgb}{0.4,0.2,0.4}
\definecolor{pmcolor}{rgb}{0.8,0.4,0.6}
\definecolor{clcolour}{rgb}{0.5,0.7,0.9}
\newcommand\cappos[1]{\nbc{JC}{#1}{jccolor}}
\newcommand\phyllis[1]{\nbc{PF}{#1}{pfcolor}}
\newcommand\preston[1]{\nbc{PM}{#1}{pmcolor}}
\usepackage{wasysym}



\begin{document}
%
% paper title
% can use linebreaks \\ within to get better formatting as desired
\title{CrashSimulator: Identifying Environment Induced Faults Through System Call Trace Analysis, Mutation, and Replay}


% author names and affiliations
% use a multiple column layout for up to three different
% affiliations
%\author{\IEEEauthorblockN{Michael Shell}
%\IEEEauthorblockA{School of Electrical and\\Computer Engineering\\
%Georgia Institute of Technology\\
%Atlanta, Georgia 30332--0250\\
%Email: http://www.michaelshell.org/contact.html}
%\and
%\IEEEauthorblockN{Homer Simpson}
%\IEEEauthorblockA{Twentieth Century Fox\\
%Springfield, USA\\
%Email: homer@thesimpsons.com}
%\and
%\IEEEauthorblockN{James Kirk\\ and Montgomery Scott}
%\IEEEauthorblockA{Starfleet Academy\\
%San Francisco, California 96678-2391\\
%Telephone: (800) 555--1212\\
%Fax: (888) 555--1212}}

% conference papers do not typically use \thanks and this command
% is locked out in conference mode. If really needed, such as for
% the acknowledgment of grants, issue a \IEEEoverridecommandlockouts
% after \documentclass

% for over three affiliations, or if they all won't fit within the width
% of the page, use this alternative format:
%

% use for special paper notices
%\IEEEspecialpapernotice{(Invited Paper)}




% make the title area
\maketitle

\begin{abstract}

  This work introduces CrashSimulator, a technique and tool that identifies
  incorrect application behavior caused by unhandled anomalous environmental
  conditions. CrashSimulator replays system call traces recorded from the
  application under test as a means to monitor and control execution. During
  replay CrashSimulator modifies system call return values and program state to
  simulate an anomalous environment. Next, a checker determines whether the
  application is responding to the anomaly appropriately. This process allows
  CrashSimulator to use incorrect behavior identified in one application running
  in one environment as a test case that determines whether another application
  in a different environment is making the same mistake. By testing an
  application against a series of these test cases, CrashSimulator provides an
  applications developers with an idea of how the application will respond were
  it executed in the environment from which the test cases were
  generated. CrashSimulator was able to identify bugs resultant from anomalous
  filesystem and network conditions in 30 popular applications. The damage
  caused by these bugs ranges from application hangs to data loss and remote
  denial of service conditions. While one of the bugs we found was independently
  identified and reported to Pythons bug tracker, the rest had not yet been
  discovered.

\end{abstract}


\IEEEpeerreviewmaketitle{}

\section{Introduction}

\textit{No Battle Plan Survives Contact With the Enemy --- Helmuth von Moltke}

No matter how well an application is tested before being released, new
bugs always seem to be found after deployment. An important reason
explaining this observation is that applications are running in a
diverse set of different deployment \emph{environments}. Many
applications seek to work across different operating systems, network
types, etc.  Changing the operating system can cause new bugs to
occur~\cite{LinuxGlibcChanges}, file systems exhibit subtle but
critical differences among each other~\cite{EXT4Layout, AppleHFS}, and differences
in network behavior can be substantial, even in situations where the
network type or even the adapter are identical~\cite{vbox}. These
environmental differences greatly exacerbate the problems in ensuring
that an application functions correctly before it is deployed.

A recent survey conducted by ClusterHQ~\cite{ClusterHQSurvey}
confirmed that application developers spend a significant portion of
their time debugging errors that are only discovered in production.
The survey participants cited the inability to recreate production
environments for testing as the main reason why bugs are not
discovered earlier.  Putting forward enormous developer effort may be
insufficient to uncover these bugs before deployment.  Microsoft
employs thousands of engineers with nearly a 1:1 ratio of testers to
developers~\cite{Page2009}.  In spite of this enormous emphasis on
testing a recent Windows Update released in response to the Spectre
Intel CPU vulnerability resulted in machines with certain hardware
configurations being rendered unbootable~\cite{kb4056892}.  Even
specialized ``Write-Once, Run Anywhere'' environments that attempt to
hide these environmental differences, such as the Java Runtime
Environment, are not perfect, leading them to be rechristened
``Write-Once, Debug Everywhere''~\cite{WODE}.  Clearly, bugs due to
differences between development and production environments are being
missed in testing only to be found in deployment.



%For example, in some environments the {\tt select()} system call
%correctly updates the state of a variable tracking the length of time
%the call was blocked, while in other environments this variable is not
%correctly updated.  Consequently, applications that depend on one
%behavior or another may fail when the call acts
%differently. Similarly, consider an application that has been
%developed with the assumption that the socket returned by {\tt
%  accept()} will inherit the settings configured on the parent socket.
%This is how {\tt accept()} behaves in the canonical BSD sockets
%implementation so this is a reasonable assumption.  Unfortunately,
%this is not the case in Linux where the returned socket does not
%inherit these options causing subsequent system calls to misbehave if
%the application does not account for the anomaly. In fact, this
%anomaly resulted in a portability bug in
%Python~\cite{Zhuang_NSDI_2014}.  Interactions with file systems are
%also prone to small differences in environments.  In testing GnuPG we
%discovered that developers implicitly assumed that a file being
%provided as input to the key import process was a regular file; when
%this assumption was violated, the application failed.
%
In this paper, we introduce {\em CrashSimulator}\footnote{
Our approach is loosely inspired by flight simulators, which test pilot
aptitude under a variety of rare, adverse scenarios (water landings, 
engine failures, etc.) before the pilots are certified to work in practice.}, 
a testing approach
and tool that can test an application as though it is running in a large number
of diverse environments likely to cause crashes.  It is not 
feasible to test all possible environments, so an important consideration
is how to choose which environments to test.  These environments can be chosen
by  identifying 
situations that caused environmental bugs in other applications and then
testing for them systematically.
Selecting situations in this way means that an environmental bug that is detected in one application can be 
detected by CrashSimulator in other applications. 
This merely requires CrashSimulator be configured to identify the environmental
bug in question.
Once this is done, any application can
be tested for the bug using CrashSimulator with no added effort.  CrashSimulator
can be configured to test a suite of bugs accumulated from a set of applications
running in a variety of environments.
This is somewhat analagous to
CrashSimulator automatically creating unit tests for all environments
in which an application could run.
Each of these ``tests'' consists of rules for
mutating system call behavior to give the application the illusion that it is
executing in an anomalous environment and rules for determining whether or not
the application responds appropriately.
Because these rules can be applied to many
applications that may be prone to mishandling the
anomalous environment in question, prior knowledge
gathered from past deployment experiences can be used to more thoroughly work
out a new application so that it can
handle the challenges of its target environments correctly.

CrashSimulator's technique for performing this testing procedure is based on the
insight that environmental anomalies can be represented as anomalies in results
and side effects of the system calls an application makes.  CrashSimulator tests
an application by exposing it to these anomalous conditions during the course of
execution and evaluating the application's response to them.  This details of
this exposure are configured by taking a system call trace of the application
under normal conditions and modifying it such that the anomalous conditions are
represented.  CrashSimulator uses these modified system call traces to control a
replay execution of the application in which these conditions will be encountered.

%CrashSimulator is based on the
%observation that: {\em Insight gleaned from a fault that is
%  caused by unexpected interactions of {\em one application} with one
%  of its deployment environments, can be leveraged to discover related
%  faults in a {\em wide range} of applications that may run in that
%  environment.}  
% of the problematic environment.  In the case of CrashSimulator,
%we are concerned with the implicit inputs, implementation details, and resource
%constraints an application's environment supplies.  Rather than stormy
%weather and foggy landings, applications experience anomalous conditions such as varying
%kernel versions with slightly different implementations details,
%unusual filesystem configurations, and unexpected network conditions that
%affect performance. Our approach tests an application against such anomalous
%conditions by modifying the responses to the
%application's system calls, thereby simulating potential deployment environments.


%\cappos{Please check}
%There
%are other testing techniques that can provide environmental bugs for later
%analysis with CrashSimulator.  This includes
%\cappos{remove or move later?  This shouldn't be 'related work', but should
%mention tools we leverage to get our anomalies...
%approach~\cite{mariani2007compatibility,
%  DBLP:journals/ase/WasylkowskiZ11, DBLP:conf/icse/PradelJAG12,
%  DBLP:journals/tosem/MonperrusM13, DBLP:conf/icse/JamrozikSZ16}, }
%techniques that detect bugs involving individual system
%calls~\cite{Koopman00theexception,Dadeau:2008:CSM:1433121.1433137,Farchi02}.
%However, a correct response to an anomaly triggered by one system call
%often requires the application to make several other system calls,
%each of which may trigger further anomalies. Tools such as
%NetCheck~\cite{Zhuang_NSDI_2014} and
%CheckAPI~\cite{rasley2015detecting} can analyze complex response
%patterns while the anomaly is observed in the deployment
%environment. However, they do not support systematic testing of
%applications before deployment.  Consequently, relying only on these
%capabilities results in applications being deployed with a significant
%number of bugs that are only discovered after the fact ---
%necessitating costly bug triage and application re-deployment.
%We are the first to
%explore techniques for finding bugs related to complex
%environment interactions that involve multiple interdependent system
%calls, before the application is deployed. 


% CrashSimulator is concerned with bugs that arise from subtle
% differences in behavior between executions in one environment versus
% another, such as the files and network communications visible to the
% application through the system calls the application makes during the
% course of its execution.

% In each of these cases, when problems of this nature are discovered in one application, not only should that application
% be fixed, but a huge group of other applications that use similar constructs and that may run in the problematic
% environment should also be tested to see whether they have the same problems. CrashSimulator facilitates that task.

%Since an application's interactions with its environment are mediated
%by system calls, it is possible to represent the difference between
%the environment in which testing is performed (referred to here as the
%{\em test environment}) and the environment under simulation (referred
%to here as the {\em anomalous environment}) as a set of differences in
%system call interaction results. This means a given target
%environment can be simulated, from the application's perspective, by
%influencing the results and side effects of the system calls the
%applications makes such that they appear to be coming from the target
%environment.  CrashSimulator interposes on system calls to allow an
%application to be tested as if it were running in the target
%environment by creating a simulation of the environment using this
%technique and executing the application within it.

%\cappos{I would cut this.}
%Here is how CrashSimulator works in detail. Suppose an application
%$A$ has failed due to a newly discovered environmental anomaly.  A
%developer using CrashSimulator can then test whether other
%applications $A_1$, $A_2,\dots$ suffer from the same problem.  The
%first step is to run these other applications in their development
%environment in order to gather traces.  Next, she writes a simple
%script to evaluate whether traces are relevant to this anomaly and to
%mutate traces to give the appearance that they were generated in the
%anomalous environment.  Finally, she attempts to replay the modified
%trace, i.e. allow the application to run under the control of
%CrashSimulator, substituting the system call return values (and
%side-effects) that would occur in the anomalous environment. In
%parallel with the replay, a checker will run under the control of
%CrashSimulator that decides whether the application's behavior is
%indicative of a fault.

%Given that we have a simulation of a given environment, the next step is to make decisions about whether an application
%is executing correctly in this environment.  The set of interactions (i.e. system calls) necessary to correctly perform
%a given operation in a given environment can be thought of as a protocol that the application must follow.  With this
%idea in mind it is possible to monitor the execution of an application and, at each system call, update its state as it
%carries out the protocol.  At the end of execution this this state can examined in order to determine whether or not the
%appliation successfully completed the protocol and, as a result, performed the operation correctly in terms of the
%simulated environment.
%


%MOVE NEXT FEW PARAGRAPHS TO SUMMARY IN CONCLUSION ???
%
%From a usability perspective, the above provides a repeatable way to inject anomalous behavior into executions of any
%application that needs to be tested.  Essentially, a ``test suite'' of interesting injectable environment conditions can
%be accumulated from work across some number projects and applied across another project that needs to be tested.  When
%these ``tests'' are collected based on issues experienced in other environments, the collection of ``tests''
%approximates the unusual conditions that any application might expect to encounter if it was run in this environment.
%By executing any application against the ``suite'' of ``tests'' gathered from a particular environment, CrashSimulator
%is effectively executing the application inside a simulation of that environment complete with all its peculiarity.  In
%this way, CrashSimulator is able to give developers a sense of how an application will perform in that evironment so
%that they can construct to deal with the enviroment correctly.
%
%CrashSimulator achieves this goal by analyzing and manipulating the execution of the application being tested.
%Specifically, CrashSimulator ``replays'' a previously recorded system call trace of the application being tested and
%uses the control this gives in order to observe the system calls an application makes and to intercede at specific
%points in order to direct execution down a path that is interesting from a testing perspective.
%
%In order to evaluate the effectiveness of this approach this work introduces CrashSimulator.  CrashSimulator provides
%environment simulation by replaying a system call trace of an application while intervening at appropriate times in
%order to inject the anomalies that make the target environment unique.  During the course of each execution
%CrashSimulator evaluates the applications behavior by providing the sequence of system calls it makes to deterministic
%computational models that encode protocols required for the application to interact with the simulated environment
%successfully.  Experiments using CrashSimulator have shown that it can rapidly perform a high volume of replay
%executions in simulated environments.  CrashSimulator has also been successful in identifying bugs in several popular
%applications.  Testing these applications by executing them in an environment provided by CrashSimulator while
%monitoring the execution using models that encoded the protocol around moving files across storage devices in Linux
%resulted in the discovery of new bugs with effects ranging from application crashes to loss of data.
%
%END -- move to conclusion

%To illustrate CrashSimulator's operation, consider applications that
%at some point ``move'' a file from a source filename to a destination
%filename.  Here, the relevant aspect of the environment is the state
%of the filesystem.  In many environments, it is sufficient for the
%application to simply call the {\tt rename()} system call, so having
%tested the application only in such environments, developers might
%erroneously conclude that the application handles the ``move''
%correctly.  However, there are many situations (environments) in which
%this is not sufficient, including environments involving symbolic
%links to files, cross-device moves, and/or special files.  In this
%paper, we will illustrate CrashSimulator's operation on an example in
%which a special file is being moved to a different device.  This
%anomaly is handled incorrectly by several widely used
%applications. Other similar anomalies that CrashSimulator can detect
%are discussed in section~\ref{sec:evaluation}.
%
%After discovering that some application $A$ fails in an environment
%where it tries to move a special file (such as {\tt /dev/urandom)}
%across devices, crash simulator can test other applications $A'$ as
%follows:
%\begin{enumerate}
%\item Run the application under test, $T$, in its normal testing environment (e.g. regular file is
%  moved on the same device), recording a trace of $T$'s system calls,
%  including their arguments and return values.
%\item Check that this trace satisfies some of the requirements for
%  handling the anomalous environment correctly. In this case, the
%  trace must include a call to {\tt stat()} or a related system call
%  to check the status of the file being moved. If it does not, the
%  application almost certainly will not work correctly in the
%  anomalous environment.
%\item Execute the application under the control of CrashSimulator,
%  replaying the trace up to the point of the {\tt stat()} call. Here,
%  each time the application makes a system call, CrashSimulator
%  intervenes, supplying return values and causing side effects as the
%  actual call would, without executing the actual call.
%\item Modify the structure returned by the {\tt stat()} call so that
%  it returns values indicating that the file is a special file,
%  i.e. so it simulates execution in the anomalous environment.
%\item Continue execution of $T$ under the control of
%  CrashSimulator. Here the series of system calls made by the
%  application may diverge from the original trace, as $T$ responds to
%  the modified return value from the {\tt stat()} call. CrashSimulator
%  feeds the series of system calls made into a {\em checker},
%  customized for this particular anomaly, that determines whether the
%  application is attempting to respond to the anomaly and either
%  reports a bug (failure to respond to the anomaly) or success (the
%  application appears to be making an effort to respond to the
%  anomaly).
%\end{enumerate}



This paper makes the following contributions:

\begin{enumerate}
\item{It proves the importance of the interaction between an application and
    its environment in creating potential flaws upon deployment.}
\item{It proposes the idea that by manipulating these interactions through
    the use of system calls, the responses of an application to any given
    environment can be accurately simulated without an actual deployment to
    that environment.}
\item{It offers an approach for encoding an appropriate flow of
    interactions between an application and its environment as a model that
    can be later used to check for correct behavior.}
\end{enumerate}

This paper also introduces a proof of concept implementation of CrashSimulator's
technique that was able to find bugs, both known and unknown, in popular Linux
applications as ranked by Debian's popularity contest~\cite{DebPopCon}.  These
bugs were found by exposing the applications to environmental conditions that
simulate unusual file system configurations, file types, and network delays.
When the applications in question were actually exposed to these conditions a
variety of failures including hangs, crashes, and filesystem damage occurred.  In
total, 84 bugs were identified.

%\thomas{I don't think that the introduction should end with an overview of the
%  paper. I prefer to
%  end the intro with a strong argument for why the paper should be accepted. In our
%  case, this is the summary of the evaluation and what is novel about
%  CrashSimulator. This way of organizing the intro also makes it
%  easier for the reviewer. The overview rarely adds anything meaningful. We
%  could even drop it altogether. Alternatively, move it to the
%  beginning of the next section.}
%In the remainder of this paper we
%offer additional details on CrashSimulator's operation.
%In section~\ref{sec:approach} we present the rationale for our system call trace
%and replay system and explain how anomalies are identified, extracted,
%converted, and injected while section~\ref{sec:evaluation} analyzes the results
%of testing conducted with the tool and evaluates its effectiveness at checking
%behavior in simulated environments.  Related work is discussed in
%section~\ref{sec:relatedwork} and we offer concluding thoughts and suggestions
%for future work in section~\ref{sec:conclusion}.


%%% Local Variables: 
%%% mode: latex
%%% TeX-master: "CrashSimulator"
%%% End: 


\section{Background and Motivation}

    CrashSimulator operates in a unique niche in relation to other testing tools and techniques. As a result, it is
    better suited to identify and report on certain types of errors that relate to the environment in which an
    application is running.

    \subsection{Existing Techniques}

    Existing tools can be roughly divided into two categories, black-box and white-box, based on the techniques they
    use to perform their testing. Black-box tools simply manipulate the inputs of the application under test and
    observe the resultant outputs. White-box tools, on the other hand, perform complex analysis of the application's
    source code in order to reason about what inputs are likely to produce interesting outputs. Each of these
    methodologies have their own advantages and disadvantages.

    White-box testing tools typically rely on a similar set of techniques, including constraint solving of branch
    statements in an application's code and symbolic execution of an application's code in order to generate inputs
    that optimally exercise the application's code paths. These techniques, while powerful, are not without their
    downsides. First, both techniques are computationally-expensive. Furthermore, symbolic execution can not always
    accurately represent actual execution and so there may be deviations in results. Similarly, efficiently solving
    a series of constraints in order to exercise a particular code path can be can be difficult to guarantee that a
    particular set of generated inputs will exercise the intended code path in many circumstances due to external
    dependencies that the tool cannot analyze. For example, a white-box testing tool cannot reliably generate inputs
    that are guaranteed to exercise a code path that relies on an operating system resource being available.
    Finally, white-box tools typically require that an application's source code be available which is not always
    the case. Even advanced white-box tools that analyze an application's machine code can be stymied in situations
    where an application's executable has been packed or encrypted.

    The alternative, black-box tools, have their own set of issues. They do not have an understanding of what an
    application is actually doing during execution which means they are only able to submit inputs and observe
    outputs.  The upside of this technique is simplicity. Black-box tools do not need the capability to understand
    and analyze an application's code which reduces their complexity immensely. Also, their testing process,
    mutating inputs and observing outputs, is computationally inexpensive. The downside of simplicity is that they
    cannot craft inputs with any sort of intelligence. This means that a great deal of time can be spent mutating
    inputs without much success in terms of bug identification. Also, they cannot identify specifically the source
    of faults in an application. They can only signal that a fault has occurred at some point during a test run.
    Furthermore, like white-box tools, these tools fail to take into account the environment in which the
    application is running.


    \subsection{What is an ``environment''}

    An application's environment is the collection of all resources external to the application with which the
    application communicates.  These external resources can be thought of as an implicit input to the program as their
    presence and contents alter the flow of any application executed in there presence.  This work is concerned with bugs
    that arise from subtle differences in behavior between executions in one environment versus another. Specifically, the
    files and network communications visible through the monitoring the system calls the application makes during the course
    of its execution.  Consider the following hypothetical example: an application utilizing the POSIX API is executed on a
    Linux machine and an OS X machine.  Because the API's called in each situation superficially act the same, a developer might
    make the assumption that they are identical in all regards -- an assumption that will result in bugs when in cases where
    it does not hold.

    Another interesting case is the situation where the actual API's in question work correctly but the resources they
    operate on do not.  Consider the case where an application is executed on two different hosts with the same environment
    present locally but connected two different networks.  In this case, differences in the network itself can cause issues
    with the application's execution.  For example, if the application assumes that it will receive messages of a particular
    length from a network host it will not execute correctly when messages of that length are dropped, or fragmented, by a
    intermediary host with strict MTU requirements. \emph{Do I need to define/describe MTU here? -Preston}


    \subsection{A Real World Example}

    During the course of its evaluation, NetCheck was able to identify a bug in Python's socket handling code that
    compromised portability of Python applications. In short, calling \emph{accept} on a socket that was created in
    non-blocking mode would return a new socket in blocking mode on Linux where the same calls would return a
    non-blocking socket on OS X, Windows, and various flavors of BSD\@. This caused code that assumed the return
    socket would always be blocking to work correctly on Linux but fail when an unhandled EWOULDBLOCK error was
    encountered on the non-blocking socket under OS X, Windows, and BSD\@. This fault was a source of error for
    several major Python applications. Many put in place work-arounds without knowing the actual source of the
    problems.  This type of environment-related error is exactly the type of fault that CrashSimulator was designed
    to identify.

    Because NetCheck was able to identify this anomaly in Python, CrashSimulator is able to produce mutated system
    call traces that can induce it in other applications. NetCheck identifies situations where a socket received
    from a call to \emph{accept} has inherited the non-blocking flag from the socket initially passed into
    \emph{accept}. This anomaly will be recored and during future test runs CrashSimulator will be able to produce
    mutated system calls containing when it encounters a call to accept on a non-blocking socket in a system call
    trace taken from the application under test.


\section{CrashSimulator Approach Details}

    %%% Verify information on supported trace formats %%%
    \subsection{System Call Traces}

    The first step in CrashSimulator's operation is to gather a trace of the system
    calls made by the application during a normal run. Where other tools base their operation of direct analysis of the
    application under test CrashSimulator operates based on information gleaned from system call traces. This gives
    CrashSimulator several advantages over similar tools. First, CrashSimulator operates in a language independent
    manner. It can test any program given two conditions hold true:

    \begin{enumerate}
        \item{The application can run in the testing environment}
        \item{The testing environment has the tooling required to take a system call trace of the application during a
        normal run}
    \end{enumerate}

    This removes the need for the complex language parsing that other similar tools rely on. Second, the faults injected
    by CrashSimulator test the interface between the application under test and its environment. Other similar testing
    tools focus on testing the logic within the application under test which missing faults that only appear when the
    application is run in an imperfect, real world environment.

    Because CrashSimulator's trace analysis engine is based on prior work from the NetCheck supported trace gathering
    tools include \emph{strace} on Linux and \emph{dtrace} on OS X.


    \subsection{Supported Anomaly Types}

    \textbf{I would really like to find another term besides anomaly}

    CrashSimulator's goal when analyzing a normal run system call trace is to identify individual system calls or
    patterns of system calls that it recognizes as an opportunity to inject a fault during subsequent runs. These
    signatures are referred to as ``potential anomalies.'' CrashSimulator has the ability to identify and make use of
    several classifications of potential anomalies.

    \subsubsection{Return Value Modification}

    \subsubsection{Data Reordering}

    \subsubsection{Data Truncation}

    \subsection{Trace Analysis}



    \subsection{Fault Injection}

     Once a set of potential fault injection points has been identified CrashSimulator uses
    \emph{ptrace} inject faults on a live execution of the program. At this point, two faults types identifiable by
    NetCheck are supported: modification of return values from network system calls and reordering of UDP packets.
    These faults were chosen because of high impact they can have on applications that don't handle them properly and
    the frequency with which developers have an incorrect understanding of how these system calls can behave in an
    imperfect environment.

    %% I can go into much more detail with the ptrace implementation stuff if I need to
    \subsection{Return Value Modification}

    One way CrashSimulator can inject faults into the running application is to
    modify the return values of interesting system calls identified in the previous trace analysis step. First, the
    CrashSimulator parent application is launched and configured.

    \begin{verbatim}
    launching of parent application
        setup of child application under test
        run child application
        For for each system call determine if it is one we are interested in as determined by analysis
            e.g.\ is this the 4th call to recv?
        Modify EAX/RAX register after system call completion to contain new return value
            New return value random? Smaller than present return value? Some constant -1?
            What do we do about 32-bit vs 64-bit
    \end{verbatim}

    %% What is the process for culling our set of unit tests for instances where there are a huge number of permutations
    %% of packet orderings
    \subsection{Catalog-Based Anomalies}

    A second category of faults CrashSimulator can produce are known as catalog-based
    faults. These faults are injected in two steps. First, the normal run trace is parsed in its entirety for system
    calls in a specific set associated with the fault being injected and the data items passed into these system calls
    are recorded in a data item catalog. Next, the application under test is run repeatedly with each run receiving a
    different ordering of data items from the catalog for the corresponding system calls. One example of a fault
    CrashSimulator can inject in such situations is unhandled out of order UDP datagrams. Consider the following
    pseudo-code listing:

    %% Use the real C Code here
    \begin{verbatim}
        int main() {
            socket = setupUdpSocket()
            data1 = recvfrom(socket)
            processData1(data1)
            data2 = recvfrom(socket)
            processData2(data2)
        }
    \end{verbatim}

    This listing sets up a UDP socket and receives two datagrams from the socket processing each with the appropriate
    function. A C program that implements this pseudo-code will produce a normal flow system call trace as follows:

    \begin{verbatim}
        ...
        socket(PF_INET, SOCK_DGRAM, IPPROTO_UDP) = 3
        bind(3, {sa_family=AF_INET, sin_port=htons(6666), sin_addr=inet_addr("0.0.0.0")}, 16) = 0
        recvfrom(3, "test\n", 256, 0, {sa_family=AF_INET, sin_port=htons(51490), sin_addr=inet_addr("127.0.0.1")}, [16]) = 5
        ...
        write(1, "Process 1: test\n", 16)       = 16
        recvfrom(3, "testagain\n", 256, 0, {sa_family=AF_INET, sin_port=htons(51490), sin_addr=inet_addr("127.0.0.1")}, [16]) = 10
        write(1, "Process 2: testagain\n", 21)  = 21
        ...
    \end{verbatim}

    The above program assumes that datagram 1 will always arrive first and datagram 2 will always arrive second. UDP
    makes no ordering guarantees so the reverse is possible. This would result in datagram 2 being processed as datagram
    1 and vice versa. Crash simulator would inject this fault as follows. First, it would parse the system call trace
    and identify all calls to recvfrom, storing the data that was received in a data catalog and the identifying
    information pertaining to the socket it was received from. Second, it would re-run the application under test and
    send a different ordering of data from the data catalog to the socket in question. From the above example,
    ``testagain'' would be sent to the first receive from and ``test'' would be sent to the second recvfrom resulting in
    each data item being parsed by the incorrect function. CrashSimulator would then report abnormalities in the
    application's behavior. \textbf{How are we going to handle situations where this is a silent failure}

    \subsection{Limitations}

    %% This text will need to be updated based on discussion about handling testing of interpreted languages
    In situations where CrashSimulator is testing an application written in an interpreted language the possibility
    exists that faults will be found in the interpreter rather than the application itself. For example, CrashSimulator
    may modify system calls made by the interpreter for purposes that are independent from the application under test
    This could result in the interpreter itself behaving improperly...........


% TODO: Evaluation: Time the amount of time to run the tests
% TODO: Evaluation: Count how many anomalies could be injected into a particular trace
% TODO: Evaluation: How many of these are useful
% TODO: Evaluation: Threats to validity
%     Programming languages of test applications
% TODO: Do different input trace applications have an effect?

\section{Evaluation}

    This work hopes to answer the following questions about CrashSimulator's operation:

        \begin{enumerate}
            \item{Is CrashSimulator successful in identifying flaws in new and existing applications?}
            \item{Is CrashSimulator able to generate tests in a performant manner?}
            \item{Is CrashSimulator able to execute tests in a performant manner?}
        \end{enumerate}

    In order to measure its efficacy CrashSimulator was implemented in Python. Python was chosen in order to facilitate
    code reuse from the previous work on CheckAPI and NetCheck. This implementation of CrashSimulator was evaluated on
    the basis of execution performance and number of bugs detected across a set of small test programs and larger, more
    mainstream applications.

    % TODO: Talk about the test platform, OS, Hardware, etc.
    \subsection{Execution Performance}

        One key attribute of successful testing tools is that they be able to complete their tests in a timely manner.
        If a tool takes too long to complete its tests users will be less likely to run it frequently, or at all,
        reducing the tools overall usefulness dramatically. To this end, the performance of CrashSimulator was evaluated
        in order to determine whether or not it was able to complete its test executions in an acceptable time frame.
        \textbf{There probably needs to be a citation here related to how long a developer is typically willing to wait
        for a test execution to complete}

        \subsubsection{Evaluation Against Sample Programs}

            The Python implementation of CrashSimulator was performance tested against two sample programs that were
            seeded with specific counts of potential anomalies. The two sample programs were seeded as follows:

            \begin{table}[H]
                \scriptsize{}
                \begin{tabular}{l  l  l  l}
                    \toprule{}
                        Title & Return Value Modification & Catalog Based & Data Fragmentation \\
                        Sample A & 20 & 2 & 2 \\
                        Sample B & 100 & 20 & 20 \\
                    \bottomrule{}
                \end{tabular}
            \end{table}

            These sample applications were written with two goals in mind. First, their system call related behavior
            should be deterministic and repeatable in nature. Second, they should be able to run to completion without
            user interaction and with a completion time that would allow for hundreds of executions in a reasonable
            amount of time. In short, the sample programs allow an accurate evaluation of CrashSimulator's performance
            by acting as ideal testing candidates.

            Repeated executions of the sample programs outside of crash simulator provide a control run time for
            comparison to times recorded from the CrashSimulator test sessions. Because CrashSimulator will potentially
            execute the application under test \emph{thousands} of times, it is not appropriate to compare test run
            times to the run time of a single run of the application. The results are recorded below.

            \begin{table}[H]
                \scriptsize{}
                \begin{tabular} {l  l  l}
                    \toprule{}
                    This Table Consists Entirely of \textbf{FAKE DATA} \\
                    Title & Number of Executions & Run Time \\
                    Sample A (No CrashSimulator Tests) & 10000 & 1m 50s \\
                    Sample B (No CrashSimulator Tests) & 10000 & 2m 30s \\
                    Sample A (CrashSimulator) & 10000 & 2m 13s \\
                    Sample B (CrashSimulator) & 10000 & 3m 15s \\
                    \bottomrule{}
                \end{tabular}
            \end{table}

            As expected, CrashSimulator is responsible for adding some overhead as compared to the non-CrashSimulator
            executions of the sample applications. For each sample program, a group of executions through CrashSimulator
            took approximately 30 percent longer to complete than an identically sized group of executions performed by
            repeatedly executing the sample program.

        \subsubsection{Evaluation Against Major Programs}

            Next, CrashSimulator's performance was evaluated against two major open source applications: Firefox and
            Apache 2. In these cases, the evaluation methodology required modification because these applications are
            not specifically engineered to meet the same criteria the sample programs were. Firefox, on one hand,
            typically requires user input in order carry out its purpose of retrieving content hosted on a remote
            server. On the other hand, Apache 2 runs as a background service and is intended under ideal conditions to
            continually serve content indefinitely. In both cases, the complexity of these pieces of software result in
            system call traces that are likely not deterministic or repeatable.

            \paragraph{Firefox}

                Because Firefox is a large application with a great deal of complex functionality and behavior it was
                determined that a specific use case should be identified for the purposes of evaluating CrashSimulator's
                performance. The primary use of Firefox for most of its users is the retrieval, rendering and display of
                content from a remote server. In the vast majority of cases this is web content retrieved over HTTP As
                such, it was decided that this use case would be appropriate for CrashSimulator's performance
                evaluation.

                In order to achieve consistent test runs the following setup was constructed. The Firefox scripting API
                was used in order to modify the behavior of the browser in two key aspects. First, the browser was
                configured to automatically request a static web page with no dynamic content from a server hosted and
                operated by the authors. A static web page was chosen to reduce the chances of any script parsing and
                execution code present in Firefox confounding the run time measurements. Second, Firefox's scripting API
                was used to end execution of the browser immediately upon the completion of the page's rendering. This
                allows for the measurement of one complete ``execution'' of Firefox from start to finish. Additionally,
                Firefox was configured to cache no content so each run was required to perform the complete set of
                communications with the server necessary to retrieve the static web page prior to displaying it. In
                order to reduce the effect of network latency on execution times both the client machine running Firefox
                and the server machine serving the static web page were located on the same network. Also, the server
                software was configured to cache no content in order to reduce the chances of server side resource
                caching affecting execution times.

                Actual evaluation of CrashSimulator's performance was conducted in a similar manner to the methodology
                described for the sample programs. The scripted version of Firefox was executed a number of times by
                CrashSimulator for testing purposes and the total time required to complete these test executions was
                recorded. For comparison purposes, the scripted version of Firefox was executed the same number of times
                \emph{without} CrashSimulator.  The following table contains the total execution times required to
                complete the listed number of executions.

                \begin{table}[H]
                    \scriptsize{}
                    \begin{tabular}{l l l}
                        \toprule{}
                        Title & Number Of Executions & Execution Time \\
                        \multirow{3}{*}{Firefox} \\
                        & 1 & 300ms \\
                        & 10 & 3000ms \\
                        & 100 & 30000ms \\
                        \multirow{3}{*}{Firefox (CrashSimulator)} \\
                        & 1 & 450ms \\
                        & 10 & 4500ms \\
                        & 100 & 45000ms \\
                        \bottomrule{}
                    \end{tabular}
                \end{table}

                From these results it is apparent that CrashSimulator  increased the time required to complete a given
                number of executions of the modified version of Firefox. The results above show an approximately 50
                percent increase in run time between the non-CrashSimulator and CrashSimulator runs. This increase is
                greater than the increase observed during the sample program evaluations. This is likely due to the high
                count of opportunities for fault injection present in a complex piece of software like Firefox as
                opposed to the sample programs. From this, it can be inferred that CrashSimulator would cause a similar
                increase in the amount of time required to complete a test of a standard version of Firefox.

            \paragraph{Apache}

                Evaluating CrashSimulator's performance against Apache saw similar challenges to its evaluation against
                Firefox. Apache is a complex piece of software and, while its normal use cases don't require user
                interaction, it presents challenges due to its typically long-lived nature. Apache's primary use case
                serving documents and files to a client program over a network connection. As such, it was decided that
                this evaluation should focus on this use case. To this end, the following evaluation setup was
                constructed.

                For this evaluation one execution of Apache consists of the following steps:

                \begin{enumerate}
                    \item{} Launch an instance of Apache
                    \item{} Request the static web page from Apache
                    \item{} Wait for transmission of the static web page to complete
                    \item{} Terminate the execution of Apache
                \end{enumerate}

                A Python script was written in order to execute these steps in a consistent manner. First, an instance
                of Apache was launched. It was configured to serve a single directory on the host machine using a single
                static host entry. This directory contained a single static web page with 100 bytes of HTML content.
                Apache was also configured to perform \emph{no} caching of local resources. This setup was chosen in
                order to reduce the amount of extraneous functionality executed by Apache. A single static host entry
                eliminates Apache's virtual host parsing and a single static HTML document eliminates the need to
                execute any external parsing engine such as mod\_php. Once Apache has completed its startup, the script
                requests the static web page using the \emph{wget} utility. The page is not rendered or displayed in any
                way upon receipt as these processes are not relevant to this evaluation.  Once the web page has been
                retrieved the script instructs the Apache instance to terminate and waits for it to do so.  Termination
                of the Apache instance concludes the measured run time for the execution.  Because the control script
                needed to be able to directly manage both client and server activates, in this setup, both the
                ``client'' script and the Apache instance were run on the same machine.

                Once again, CrashSimulator's performance was evaluated in terms of the run time of its executions versus
                the run time of the same number of executions performed without CrashSimulator's involvement. The actual
                run times measured are for the complete execution for the Apache control script described above. As
                such, some run time accumulated is the result of non-Apache client code and control script overhead. The
                following table contains the results collected.

                \begin{table}[H]
                    \scriptsize{}
                    \begin{tabular}{l l l}
                        \toprule{}
                        Title & Number Of Executions & Execution Time \\
                        \multirow{3}{*}{Apache Control Script} \\
                        & 1 & 300ms \\
                        & 10 & 3000ms \\
                        & 100 & 30000ms \\
                        \multirow{3}{*}{Apache Control Script (CrashSimulator)} \\
                        & 1 & 450ms \\
                        & 10 & 4500ms \\
                        & 100 & 45000ms \\
                        \bottomrule{}
                    \end{tabular}
                \end{table}

                As expected, the run times of the CrashSimulator-involved executions shows an increase over the
                non-CrashSimulator executions. The results above show an approximately 50 percent increase in run time
                between the CrashSimulator controlled executions and the non-CrashSimulator executions. This increase is
                in line with what was expected based on the previously discussed Firefox evaluation for similar reasons
                related to complexity. One difference with this evaluation is there is a higher likely-hood of the
                accumulated run time being affected by overhead introduced by the control script. That said, this
                overhead would be similar across all executions.

                This evaluation has shown that CrashSimulator causes an increase in run time for executions of both the
                sample programs and common major applications. That said, the total time is still well within what the
                authors define as a reasonable time frame for test suite completion. Given the high degree of success
                CrashSimulator sees in identifying faults in the applications under test, the authors do not see this
                increase in execution time as a significant impediment to the adoption of CrashSimulator as part of a
                rigorous testing process.
    \subsection{System Call Trace Mutation Performance}

        Another critical metric for CrashSimulator's feasibility in the real world is the speed with which it can
        mutate original system call traces with the anomalies from its store. CrashSimulator must be able to accomplish
        this task with sufficient speed that it shows a marked improvement over the alternative --- deploying the
        application to an environment and testing it there. To this end, the following evaluation was performed.

        CrashSimulator's anomaly store was manually seeded with 10 easily reproducible network anomalies. These
        anomalies were chosen based on their high likely-hood of being able to be injected into original system call
        traces. Next, original system call traces were collected from 10 common network-oriented applications.
        Finally CrashSimulator was executed using this seeded anomaly store and the 10 original traces. The time to
        complete the resultant mutations was recorded.

        From these inputs CrashSimulator was able to generate 100 mutated traces in 5 minutes and 30 seconds.
        This means that CrashSimulator was able to provide what amounts to 100 new unit tests with a few minutes of
        processing time. This is orders of magnitude faster than these unit tests could have been written by hand. Based
        on this result, the authors feel that CrashSimulator is a worthwhile addition to standard testing processes.

    \subsection{Bugs Identified}

        A second way CrashSimulator was evaluated was on the basis of how well it was able to independently identify
        known bugs that were listed in the bug tracking suites of two major open source projects. Firefox and Apache 2
        were chosen because of their use in the previously discussed evaluations. The goal was to identify bugs from
        each tool's bug tracker (both fixed and unfixed) that are related to the faults that CrashSimulator can inject
        and determine whether or not it was able to identify them. As a side effect, this process also yielded a number
        of new bugs that were previously unreported in these projects' bug trackers.

        \subsection{Firefox}

            For this evaluation, the same Firefox test setup as was described in the Firefox performance evaluation was
            used.

            \textbf{\emph{Note: This is fake data. Also, I would like to have better ways of quantifying some of this}}
            From Firefox's bug tracker 10 bugs in the ``networking'' category with successful resolutions were selected
            as good candidates for this portion of CrashSimulator's evaluation. Successfully resolved bugs were chosen
            because they had sufficient documentation to identify their exact cause. This was essential to determining
            the likelihood of their being identified.

            Five of the bugs were judged as ``likely to be identified by CrashSimulator'' by the authors. This opinion
            was based on how closely the abnormal behavior, and its resolution, described in the bug report related to
            faults that CrashSimulator can inject. \textbf{\emph{FAKE}} For example, one of these bugs related to a
            failure in Firefox's handling of DNS responses due to duplicated UDP datagrams. This is a fault that
            CrashSimulator is readily able to inject so it was expected that CrashSimulator would cause the fault to
            occur and report it.

            The other five bugs where judged by the authors as ``unlikely to be identified by CrashSimulator.'' These
            bugs were, once again, selected from the ``networking'' category and had successful resolutions. These bugs
            were chosen because the abnormal behavior they produced and their resolutions indicated that they were
            caused by faults that CrashSimulator does not currently have the capability of injecting. For example, one
            bug related to an application logic error related to a specific edge case malformation in a TCP packet
            received from a particular version of the IIS web server. CrashSimulator is unlikely identify this bug
            because it does not support the specific modification of TCP packets required to trigger it.

            The identification status of each bug is listed in the table below.

            \begin{table}[H]
                \scriptsize{}
                \begin{tabular}{l  l  l  l}
                    \toprule{}
                        Bug Tracker ID & Likely to be Identified & Identified & Notes \\
                        0001 & True & True & Note \\
                        0002 & True & True & Note \\
                        0003 & True & True & Note \\
                        0004 & True & True & Note \\
                        0005 & True & True & Note \\
                        0006 & False & False & Note \\
                        0007 & False & False & Note \\
                        0008 & False & False & Note \\
                        0009 & False & False & Note \\
                        0010 & False & False & Note \\
                    \bottomrule{}
                \end{tabular}
            \end{table}

            As can be see from the results above, CrashSimulator performed as expected based on the selected bug's
            likeliness to be identified. In addition, CrashSimulator also identified a previously unreported bug in
            Firefox's network code. This bug results in a denial of service condition caused by a failure to check the
            return value of a call to \emph{socket} and the subsequent use of the invalid socket handle it can produce.
            This bug was reported to Mozilla (bug ID\# 0011) and was corrected in the next release of Firefox.

        \subsection{Apache 2}

            \textbf{\emph{This is fake, placeholder data and information}}
            For this evaluation, the same setup as was described in the Apache performance evaluation was used.

            Like the Firefox evaluation, a set of bugs with successful resolutions were chosen from Apache's bug tracker
            from the ``networking and communication'' category. Once again, these bugs were divided into ``likely to be
            identified by CrashSimulator'' and ``not likely to be identified by CrashSimulator'' based on an analysis of
            their root cause and eventual resolution. The table below lists the bugs that were chosen, whether or not
            they were likely to be identified and whether or not CrashSimulator identified them in the end.

            \begin{table}[H]
                \scriptsize{}
                \begin{tabular}{l  l  l  l}
                    \toprule{}
                        Bug Tracker ID & Likely to be Identified & Identified & Notes \\
                        0001 & True & True & Note \\
                        0002 & True & True & Note \\
                        0003 & True & True & Note \\
                        0004 & True & True & Note \\
                        0005 & True & True & Note \\
                        0006 & False & False & Note \\
                        0007 & False & False & Note \\
                        0008 & False & False & Note \\
                        0009 & False & False & Note \\
                        0010 & False & False & Note \\
                    \bottomrule{}
                \end{tabular}
            \end{table}

            In addition to these bugs, one yet undiscovered bug was identified during the course of this evaluation.
            The bug related to a denial of service condition caused by one of Apache's low level network operations
            failing to handling low MTU values. This bug was reported to Apache's bug tracker (Bug ID\# 2007) and was
            successfully resolved in the latest release.

    \subsection{Test Coverage Improvement}

        \textbf{\emph{There needs to be more exploration put into this test methodology. Much more description needs to
        be written.}}

        A final way CrashSimulator was evaluated was on the extent to which it was able to improve upon the existing
        unit test suites of major applications make them available publicly. At a high level, this evaluation was
        performed as follows. The unit test suites of Firefox and Apache were reviewed and a count was made of tests
        that fell into CrashSimulator's domain was made. Next, a CrashSimulator was run against each application and a
        count of tests performed was recorded. These values were used to determine a percent increase number of useful
        tests ``added'' through the use of CrashSimulator

    \subsection{Limitations}

        Evaluation of CrashSimulator has yielded the following limitations to be addressed by future work.

        \paragraph{Coupling to Architecture}

            As some faults injected by CrashSimulator require low level access to the test system's hardware or
            operating system data structures there exists some degree of coupling between CrashSimulator and these
            components. One area of expansion for CrashSimulator is support for more processor architectures and more
            operating systems.  CrashSimulator's test launcher as been designed in such a way that these improvements
            should be trivial to plug in once they have been implemented.

        \paragraph{Parallel Execution of Tests}

            In its current implementation, CrashSimulator injects faults in a serial manner. The primary cause of this
            design decision was the potential existence of dependencies between the application under test and finite
            operating system resources such as network ports, database connections, or hardware devices. Should an
            application under test have such a dependency it is non-trivial to transparently allocate this resource to
            the hundreds of instances of the application CrashSimulator could potentially launch simultaneously if it
            ran tests in parallel.

        % TODO: Verify this stuff
        \paragraph{Analysis of System Call Traces for Multi-Threaded Applications}

            Part of CrashSimulator's system call analysis is determining a global ordering for system calls based on the
            input traces. This is accomplished by making the assumption that all system calls are atomic and executed
            one at a time. This assumption does not hold in the case of multi-threaded applications. As a result, in
            these cases CrashSimulator is only able to inject faults that depend on single system calls.

        \paragraph{Testing of Applications With No Definite End Point}

            Some applications, typically daemons of some sort, are written in such a way that they will not end unless
            they are manually stopped if they encounter some fatal error condition. In these cases CrashSimulator must
            take a heuristic approach to its application monitoring post-injection. Cases where the application exhibits
            some invalid behavior immediately after injection are trivially identified. Cases where the injected fault
            causes a invalid behavior to become apparent at some undetermined future time are much more difficult to
            nail down. In these cases, CrashSimulator uses a user defined time period to kill the process and move on to
            the next test.

        \paragraph{Testing of Applications Written in Interpreted Languages}

            Because CrashSimulator operates on the system calls made by an application it does not make any attempt to
            determine what the cause of a fault may have been at a higher level. For example, in situations where
            CrashSimulator is testing an application written in an interpreted language the possibility exists that
            faults will be found in the interpreter rather than the application itself. For example, CrashSimulator may
            modify system calls made by the interpreter for purposes that are independent from the application under
            test.  If this results results in improper output CrashSimulator will simply report it as a fault in the
            application despite the fact that the user's code was not responsible for the error.


\section{Related Work} \label{sec:relatedwork}

\iffalse
While there is a vast literature on test
generation~\cite{ammann2008introduction, mcminn2004search,
  puasuareanu2009survey, dias2007survey}, much less work
has focused on issues of portability and testing whether software
behaves consistently in different environments.  Prior work on
CheckAPI~\cite{rasley2015detecting} and
NetCheck~\cite{Zhuang_NSDI_2014} begins to fill this gap and this paper
builds upon those results.
%
%\paragraph{Detecting Environmental Bugs.}
%
%NetCheck; CheckAPI; stub injection; detecting machine specific bugs
%(e.g. numeric/memory limitations); testing error handlers; \dots


Crash reproduction by test case mutation~\cite{DBLP:conf/sigsoft/XuanXM15}.

\fi


\noindent
{\bf Static analysis. }
Tools based on static analysis techniques such as abstract
interpretation, model checking, and symbolic execution have been used
successfully to detect bugs related to incorrect API usages. Examples
include SLAM~\cite{Ball_adecade, Ball:2002:SLP:503272.503274} and,
more recently, CORRAL~\cite{DBLP:conf/sigsoft/LalQ14} for conformance
checking of Windows device drivers against the Windows kernel API,
FindBugs~\cite{DBLP:conf/oopsla/HovemeyerP04} for detecting API usage
bugs in Java programs, FiSC~\cite{Musuvathi04modelchecking} for
finding bugs in TCP implemenations, and the Explode
system~\cite{Yang:2006:ELG:1298455.1298469} for detecting crash
recovery bugs in file system implementations. Unlike CrashSimulator,
these approaches depend on the availability of source or byte code and
are typically more prone to false alarms. On the other hand, static
analysis can provide stronger soundness guarantees.  Static analysis
has also been used to detect portability issues related to changes
between different versions of external components that a program
depends on~\cite{silakov2010improving, javacompliance-www}. Like
CrashSimulator, these address the application's interactions with its
environment, but this work focuses on testing the application's
response to anomalies, rather than on proving that the environments
behave as expected.

\iffalse
\noindent
{\bf Specification and run-time verification.}  Substantial work has
been done in validating API and protocol behaviors, e.g., finding
faults in the Linux TCP implementation,
SSH2 and RCP~\cite{Udrea:2008}, BGP
configuration~\cite{Feamster:2005}, and identifying network
vulnerabilities~\cite{ritchey-sp00}. 
\fi

\noindent
{\bf API protocol mining.}
There has been extensive work on mining source code to learn API
protocols and use them to detect common usage violations such as
missing method calls (see, e.g.,~\cite{mariani2007compatibility,
  DBLP:journals/ase/WasylkowskiZ11, DBLP:conf/icse/PradelJAG12,
  DBLP:journals/tosem/MonperrusM13,
  DBLP:conf/icse/JamrozikSZ16}). These techniques primarily target
object component interactions rather than system calls to generate
test suites. However, the techniques explored in these works could be
used to mine system call patterns in source code in order to learn
checkers that identify incorrect responses to environment
anomalies. So far, we have specified these checkers manually for
CrashSimulator.

\noindent {\bf Tracing and log mining.}  
Similar to API protocol mining, there has been substantive work on
using log files to detect anomalies~\cite{pinpoint,
  jiang_abnormal_trace_detection_icac_2005, xu2009detecting,
  lou2010mining2} and aid program understanding~\cite{yuan2010sherlog,
  beschastnikh_synoptic_fse_2011, csight_icse_2014}.
%In contrast to
%this work, our focus is on diagnosing network issues from logs
%of syscalls, though prior work on log mining can be used to
%expand our scope.
Khadke et al.~\cite{khadketransparent} introduced a performance
debugging approach that relies on system call tracing. Unlike this
prior work, our system does not assume synchronized clocks and
reconstructs a plausible global ordering.
CheckAPI~\cite{rasley2015detecting} and
NetCheck~\cite{Zhuang_NSDI_2014} use system call traces to diagnose an
application's violations of cross-platform portability that are
observed in the field. In contrast to this work, CrashSimulator uses
system call traces for a different purpose: mutating and replaying
traces to test an applications' responses to known environmental
anomalies before the application is deployed.


\noindent
{\bf Portability.}
Wrappers are commonly used to enable cross-platform portability of
APIs~\cite{bartolomeicompliance}. System call interposition is one
technique to generate such wrappers
automatically~\cite{Guo:2011:CUS:2002181.2002202} and has also been
used to detect and prevent security
violations~\cite{Hofmeyr:1998:IDU:1298081.1298084,
  Acharya:2000:MUP:1251306.1251307}.  CrashSimulator's goals are
complementary as it exploits environment anomalies discovered in
testing or analyzing one application to generate and execute tests for
another application.
%
Other works that target complementary classes of portability problems
include the detection of configuration-related
bugs~\cite{skoll:icse:2004, Yilmaz:issta:2004, Fouche:issta:2009,
  Kastner12, Nguyen14} and
cross-browser incompatibilities for web
applications~\cite{DBLP:conf/icsm/ChoudharyVO10, silakov2010improving,
  DBLP:conf/icse/Choudhary11, Mesbah:2011:ACC:1985793.1985870,
  DBLP:conf/icst/DallmeierP0MZ14}.


\noindent
{\bf Testing exception handling and conformance.}
A few researchers have developed testing techniques aimed at checking
whether programs respond appropriately to anomalous situations.  For
example, Fu et al. introduce data flow testing techniques that require
tests that cover paths from points at which exceptions are thrown to
points at which they are handled in Java code, in order to test
whether programs respond correctly to exceptional situations that the
programmer has anticipated~\cite{DBLP:journals/tse/FuMRW05}.  Koopman
and DeVale developed a system to detect bugs in error handling code
related to calls to POSIX functions~\cite{Koopman00theexception}.
%
Other approaches to conformance checking of POSIX operations use model
based testing~\cite{Dadeau:2008:CSM:1433121.1433137,Farchi02}.
%
Unlike these approaches, CrashSimulator does not exclusively
target error handling code or anomalies that only involve individual
system calls. However, such testing techniques can help identify
anomalies that can be added to our repository.


%\noindent
%{\bf Runtime verification:}
%Runtime verification (RV) 
%techniques~\cite{DBLP:conf/rv/2010, Liu:2007:WCC:1973430.1973449, Lu_ASPLOS_2006,
%Archer:2007:ICT:1236360.1236382, Verbowski_OSDI_2006, Tucek_SOSP_2007,
%Park_ASPLOS_2009,DBLP:journals/jlp/LeuckerS09}
%can detect violations of properties on specific executions
%but do not show that the software satisfies the specification for every
%possible input or on every possible execution.
%Many of these techniques find general violations of 
%properties, such as atomicity~\cite{Park_ASPLOS_2009, Verbowski_OSDI_2006}.
%Other RV techniques enable checking program-specific requirements 
%usually specified with formal languages, such as automata or logic
%formalisms~\cite{DBLP:conf/vmcai/BarringerGHS04, DBLP:conf/kbse/GiannakopoulouH01}.
%
%Many RV approaches instrument code to capture relevant
%events or application state and insert executable 
%assertions~\cite{Orso:2002:GSC:566172.566182, DBLP:conf/icse/ClauseO11,
%DBLP:books/sp/Liblit2007, Jin:2010:ISS:1932682.1869481, Barnett01spyingon, 
%DBLP:journals/jss/BarnettS03}.
%However, inserting pre- and post-conditions obscures the fact that the
%specification can be treated as a parallel construct to the
%implementation~\cite{Barnett01spyingon, DBLP:journals/jss/BarnettS03}.
%Instead, an architecture can be used for runtime verification of .NET
%components by running the model and the implementation side-by-side,
%comparing results at method boundaries~\cite{Barnett01spyingon,
%DBLP:journals/jss/BarnettS03}.
%CheckAPI does not require application instrumentation, assuming a tracing
%mechanism exists in the API~\cite{strace, Cappos_CCS_10}.

%Like CheckAPI, 
%several other (runtime and static) checking techniques
%allow the use of languages more familiar to 
%programmers. 
%The WiDS checker allows using a scripting language to specify properties of a
%distributed system~\cite{Liu:2007:WCC:1973430.1973449}.
%Contracts written in a C-like language can specify components for use
%in TinyOS applications~\cite{Archer:2007:ICT:1236360.1236382}.
%CheckAPI allows
%programmers to choose the language for PSI construction or simply to
%use an existing implementation. Lastly, work on deterministic 
%replay~\cite{Viennot13} enables record-replay techniques on multi-core
%systems and could help improve CheckAPI-MT performance.


\iffalse
\noindent
{\bf Application-specific fault detection.}
Pip~\cite{reynolds2006pip} and
Coctail~\cite{xue2012using} are distributed frameworks that enable developers to
construct application-specific models, which have proven effective at finding
detailed application flaws. However, to utilize these methods,
a knowledge of the nature of the failures needs to be acquired, and the
specific system properties must be specified. NetCheck
diagnoses application failures without application-specific models.
Khanna~\cite{khanna2007automated} identifies the source of failures using 
a rule base of allowed state transition paths.
%However, it requires specialized human-generated rules for each application. 
CrashSimulator leverages NetCheck's approach to simulate identified
anomalies in network behavior, file system behavior, etc. to test
applications other than the ones on which the anomalies were originally discovered.
\fi


%% The following is old stuff -- not sure how much should be integrated
%% into the above.
\begin{comment}
    \subsection{Existing Techniques}

    Existing tools can be roughly divided into two categories, black-box and white-box, based on the techniques they
    use to perform their testing. Black-box tools simply manipulate the inputs of the application under test and
    observe the resultant outputs. White-box tools, on the other hand, perform complex analysis of the application's
    source code in order to reason about what inputs are likely to produce interesting outputs. Each of these
    methodologies have their own advantages and disadvantages.

    White-box testing tools typically rely on a similar set of techniques, including constraint solving of branch
    statements in an application's code and symbolic execution of an application's code in order to generate inputs
    that optimally exercise the application's code paths. These techniques, while powerful, are not without their
    downsides. First, both techniques are computationally-expensive. Furthermore, symbolic execution can not always
    accurately represent actual execution and so there may be deviations in results. Similarly, efficiently solving
    a series of constraints in order to exercise a particular code path can be can be difficult to guarantee that a
    particular set of generated inputs will exercise the intended code path in many circumstances due to external
    dependencies that the tool cannot analyze. For example, a white-box testing tool cannot reliably generate inputs
    that are guaranteed to exercise a code path that relies on an operating system resource being available.
    Finally, white-box tools typically require that an application's source code be available which is not always
    the case. Even advanced white-box tools that analyze an application's machine code can be stymied in situations
    where an application's executable has been packed or encrypted.

    The alternative, black-box tools, have their own set of issues. They do not have an understanding of what an
    application is actually doing during execution which means they are only able to submit inputs and observe
    outputs.  The upside of this technique is simplicity. Black-box tools do not need the capability to understand
    and analyze an application's code which reduces their complexity immensely. Also, their testing process,
    mutating inputs and observing outputs, is computationally inexpensive. The downside of simplicity is that they
    cannot craft inputs with any sort of intelligence. This means that a great deal of time can be spent mutating
    inputs without much success in terms of bug identification. Also, they cannot identify specifically the source
    of faults in an application. They can only signal that a fault has occurred at some point during a test run.
    Furthermore, like white-box tools, these tools fail to take into account the environment in which the
    application is running.
    \subsection{White-box Tools}

        White-box testing has been an area of intense interest in recent writing. Microsoft's SAGE and Bell Labs' DART
        are two examples of such tools that take different approaches to the same overall white-box technique.

        \subsubsection{DART}

            DART is a white-box testing tool that supports testing of applications written in the C programming language. It
            is capable of generating a test harness for an application's functions by through static analysis of the
            application's source code. This harness is then used for two phases of testing. First, it performs random
            testing and observes the application's behavior. Based on this random testing and symbolic execution of the
            application's source code, DART generates a series of inputs that will be used in the second phase of testing.
            These inputs are designed to direct the application down specific execution paths, observing the programs
            behavior and reporting faults as they are identified. DART operates on the assumption that the functions being
            evaluated have no side-effects and that the application is able to interact appropriately with its environment.
            More information can be found in \textbf{\emph{PAPER TITLE HERE}}

        \subsubsection{SAGE}

            SAGE differs from many other white-box testing tools in that it analyzes a compiled application's machine code
            rather than the application's uncompiled source code. This allows SAGE to operate on applications that were
            compiled from a variety of programming languages. It first runs the application under test with a set of well
            formed inputs and records an instruction-level trace of the application's execution. Next, it analyzes this
            trace in order to identify constraints that guard different paths of execution. SAGE then solves these
            constraints and, based on these solutions, generates inputs that are able to exercise specific paths of
            execution.

    \subsection{Black-box tools}

        % TODO: Find examples of black box tools

    \subsection{Trace Analysis Tools}

        Much of CrashSimulator's work on system call trace analysis is based on previous work on NetCheck and CheckAPI

        \subsubsection{NetCheck}

            This implementation of CrashSimulator relies of NetCheck for system call trace analysis. NetCheck uses two
            strategies to identify potential fault areas from system call trace. The first is a model based simulation
            of the system calls relevant to network communication from the input trace. System calls are organized
            according to a POSIX socket API dependency graph and prioritized based on the order in which the system
            calls should be made in an ideal scenario.  For example, a client application should not be making a
            \emph{connect} system call before it has set up its socket with the appropriate \emph{socket} system call.
            The model assumes that all system calls are atomic and that they cannot happen simultaneously. This allows a
            definite global order to be created.

            Once a global ordering is in place each system call is evaluated based on the previous system calls. Return
            values and parameters passed in are taken into account. If the system call is feasible it is accepted and
            the next system call is evaluated. If the system call is not feasible given the current system call context
            it is rejected and logged. In addition, system calls that return a value indicating some sort of network
            failure are recorded. After all system calls have been evaluated a NetCheck attempts to diagnose the
            source of any errors encountered. It is this diagnosis that CrashSimulator uses when deciding where and how
            to mutate the ``ideal run'' system call trace it is operating on.

        \subsubsection{CheckAPI}

            % TODO: Expand this
            \textbf{\emph{This needs to be expanded}}
            CheckAPI attempts to identify
\end{comment}


\section{Conclusion} \label{sec:conclusion}
Many post-deployment failures result from
unexpected interactions between an application and its environment.
Although finding and eliminating faults in applications is a key concern for software developers, it is impractical for them to test an application in all of the
environments in which it will be deployed.
 To address this
problem, we introduced CrashSimulator, a testing
approach that utilizes mutation, replay, and analysis of system call traces
in order to determine whether an application responds correctly to
anomalous environmental conditions.
A significant benefit of CrashSimulator is that failures observed
in executing one application in an anomalous environment can
easily be leveraged to test whether {\em other} applications
suffer from the same underlying problem.

CrashSimulator's key features, operating on system calls and analysis of replayed executions allow it to identify bugs
resulting from an application's interactions with its environment.
%Performing its analysis on a replayed execution of an application allows
%CrashSimulator to remove the filesystem and network dependancies of the application under test.  
To test whether an application responds appropriately to a particular
anomaly (such as an unexpected file type, an unexpected device, an unexpected
network conditions, etc.)
CrashSimulator records system call traces from the application under test,
mutates return values and/or program state to simulate execution in 
the anomalous environment, and invokes a checker to see whether
subsequent system calls indicate a correct or incorrect response.
This allows
CrashSimulator to test an application running in one environment
as if it were running in another.

The tool evaluates application behavior in the (simulated) anomalous environment
by using
finite state models to characterize correct (or incorrect) behavior.
This provides a generic ``signature'' for that behavior, which is
portable across applications. 
Consequently, a set of mutations and checkers can be collected from any existing
application for use in testing other new or existing applications. 
In this way, an ever-expanding ``test suite'' can
be created, allowing lessons learned from bugs in one application to benefit many
others.

Our evaluation of CrashSimulator has shown it to be both effective and
efficient.  CrashSimulator was able to identify bugs related to unusual file
types in 15 applications and bugs related to slow network performance in 10
network applications and libraries.  Additionally CrashSimulator found
filesystem related bugs in 7 applications and libraries with facilities for
moving files within a system's filesystem.  The low overhead
introduced by CrashSimulator's technique meant that it was able to
find these bugs quickly in spite of its unoptimized implementation.

Future work will include expanding the repository of anomalies and their
checkers, as well as exploring opportunities to further automate the discovery
of anomalies and checkers.  In the long term, we envision a public repository of
anomalies along with CrashSimulator test patterns that can be applied to new or
existing applications.


\section*{Acknowledgment}
The authors would like to thank\ldots


\begin{thebibliography}{1}

\bibitem{IEEEhowto:kopka}
H.~Kopka and P.~W. Daly, \emph{A Guide to \LaTeX}, 3rd~ed.\hskip 1em plus
  0.5em minus 0.4em\relax Harlow, England: Addison-Wesley, 1999.

\end{thebibliography}


\end{document}

% An example of a floating figure using the graphicx package.
% Note that \label must occur AFTER (or within) \caption.
% For figures, \caption should occur after the \includegraphics.
% Note that IEEEtran v1.7 and later has special internal code that
% is designed to preserve the operation of \label within \caption
% even when the captionsoff option is in effect. However, because
% of issues like this, it may be the safest practice to put all your
% \label just after \caption rather than within \caption{}.
%
% Reminder: the "draftcls" or "draftclsnofoot", not "draft", class
% option should be used if it is desired that the figures are to be
% displayed while in draft mode.
%
%\begin{figure}[!t]
%\centering
%\includegraphics[width=2.5in]{myfigure}
% where an .eps filename suffix will be assumed under latex,
% and a .pdf suffix will be assumed for pdflatex; or what has been declared
% via \DeclareGraphicsExtensions.
%\caption{Simulation Results}
%\label{fig_sim}
%\end{figure}

% Note that IEEE typically puts floats only at the top, even when this
% results in a large percentage of a column being occupied by floats.


% An example of a double column floating figure using two subfigures.
% (The subfig.sty package must be loaded for this to work.)
% The subfigure \label commands are set within each subfloat command, the
% \label for the overall figure must come after \caption.
% \hfil must be used as a separator to get equal spacing.
% The subfigure.sty package works much the same way, except \subfigure is
% used instead of \subfloat.
%
%\begin{figure*}[!t]
%\centerline{\subfloat[Case I]\includegraphics[width=2.5in]{subfigcase1}%
%\label{fig_first_case}}
%\hfil
%\subfloat[Case II]{\includegraphics[width=2.5in]{subfigcase2}%
%\label{fig_second_case}}}
%\caption{Simulation results}
%\label{fig_sim}
%\end{figure*}
%
% Note that often IEEE papers with subfigures do not employ subfigure
% captions (using the optional argument to \subfloat), but instead will
% reference/describe all of them (a), (b), etc., within the main caption.


% An example of a floating table. Note that, for IEEE style tables, the
% \caption command should come BEFORE the table. Table text will default to
% \footnotesize as IEEE normally uses this smaller font for tables.
% The \label must come after \caption as always.
%
%\begin{table}[!t]
%% increase table row spacing, adjust to taste
%\renewcommand{\arraystretch}{1.3}
% if using array.sty, it might be a good idea to tweak the value of
% \extrarowheight as needed to properly center the text within the cells
%\caption{An Example of a Table}
%\label{table_example}
%\centering
%% Some packages, such as MDW tools, offer better commands for making tables
%% than the plain LaTeX2e tabular which is used here.
%\begin{tabular}{|c||c|}
%\hline
%One & Two\\
%\hline
%Three & Four\\
%\hline
%\end{tabular}
%\end{table}


% Note that IEEE does not put floats in the very first column - or typically
% anywhere on the first page for that matter. Also, in-text middle ("here")
% positioning is not used. Most IEEE journals/conferences use top floats
% exclusively. Note that, LaTeX2e, unlike IEEE journals/conferences, places
% footnotes above bottom floats. This can be corrected via the \fnbelowfloat
% command of the stfloats package.

% trigger a \newpage just before the given reference
% number - used to balance the columns on the last page
% adjust value as needed - may need to be readjusted if
% the document is modified later
%\IEEEtriggeratref{8}
% The "triggered" command can be changed if desired:
%\IEEEtriggercmd{\enlargethispage{-5in}}

% references section

% can use a bibliography generated by BibTeX as a .bbl file
% BibTeX documentation can be easily obtained at:
% http://www.ctan.org/tex-archive/biblio/bibtex/contrib/doc/
% The IEEEtran BibTeX style support page is at:
% http://www.michaelshell.org/tex/ieeetran/bibtex/
%\bibliographystyle{IEEEtran}
% argument is your BibTeX string definitions and bibliography database(s)
%\bibliography{IEEEabrv,../bib/paper}
%
% <OR> manually copy in the resultant .bbl file
% set second argument of \begin to the number of references
% (used to reserve space for the reference number labels box)
% IEEEtran.cls defaults to using nonbold math in the Abstract.  This preserves the distinction between vectors and
    % scalars. However, if the conference you are submitting to favors bold math in the abstract, then you can use
    % LaTeX's standard command \boldmath at the very start of the abstract to achieve this. Many IEEE
    % journals/conferences frown on math in the abstract anyway.

% no keywords




% For peer review papers, you can put extra information on the cover
% page as needed:
% \ifCLASSOPTIONpeerreview
% \begin{center} \bfseries EDICS Category: 3-BBND \end{center}
% \fi
%
% For peerreview papers, this IEEEtran command inserts a page break and
% creates the second title. It will be ignored for other modes.
