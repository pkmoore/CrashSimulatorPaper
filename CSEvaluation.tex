\section{Evaluation}

    This work hopes to answer the following questions about CrashSimulator's operation:

        \begin{enumerate}
            %% First cut at questions
            \item{Is CrashSimulator successful in identifying previously-unknown flaws in existing applications?}
            \item{Can CrashSimulator's efficacy be shown through identifying known bugs in existing applications?}
            \item{Is CrashSimulator able to execute tests in a performant manner?}
            \item{What type of bugs is CrashSimulator able to identify?}
            \item{Are there types of bugs that CrashSimulator is not able to identify?}
            \item{What technical limitations are present in this implementation of CrashSimulator}
            \item{How can CrashSimulator deal with false positives}
        \end{enumerate}

    \subsection{Experimental Setup}

    CrashSimulator was implemented in a combination of Python 2 and C.  Evaluations were performed using Ubuntu Linux
    15.04 with several modifications put in place to allow CrashSimulator to operate correctly. First, because
    CrashSimulator relies on an application's memory layout being the same across repeated executions, address space
    layout randomization must be disabled.  Second, the kernel's virtual dynamic shared object feature must be
    disabled.  This feature allows certain kernel data items (timestamps, kernel version information, etc.) to be stored
    in user space.  When the system calls that retrieve this information are made by an application they are instead
    intercepted and handled in user space -- a behavior that would preclude CrashSimulator from being able to interact
    with these calls.  Finally, a 32-bit Linux kernel and supporting environment were chosen.  While CrashSimulator
    could have been implemented on either a 32-bit kernel or a 64-bit kernel such a goal would have resulted in a great
    deal of duplicated effort around dealing with the low level differences between these two kernel versions.
    \emph{I've got a lot more I could talk about here but I'm not sure what info normally goes into this section}

    \subsection{Execution Performance}

        One key attribute of successful testing tools is that they be able to complete their tests in a timely manner.
        If a tool takes too long to complete its tests users will be less likely to run it frequently, or at all,
        reducing the tools overall usefulness dramatically. To this end, the performance of CrashSimulator was evaluated
        in order to determine whether or not it was able to complete its test executions in an acceptable time frame.

        \subsubsection{Evaluation Against Sample Programs}

        Each of the following evaluations eamines the completion times for 20 consecutive executions of the specified
        application in both native and replay execution modes.  Multiple conseuritve executions were used for two
        reasons.  First, a single execution in either mode completes so quickly for the applications in question as to
        be nearly indistinguishable from the other mode.  Second, multiple executions somewhat contros for varying
        conditions like filesystem activity or network latency that can vary between individual executions.

            \begin{table}[H]
                \scriptsize{}
                \begin{tabular}{l  l  l  l}
                    \toprule{}
                        Execution Description & Native Eecution & Replay Execution\\
                        network\_speedtest & 0.024 & 2.407 \\
                        filesystem\_speedtest & 0.038 & 2.604 \\
                        mv (cross-disk file move) & 0.016 & 2.995 \\
                    \bottomrule{}
                \end{tabular}
            \end{table}

        \paragraph{Discussion on network\_speedtest}

        Sample application opens a TCP connection to an already running listener (netcat), sends a message, and
        exits. For simple applications like this, native execution is significantly faster than replay execution...

        \paragraph{Discussion on filesystem\_speedtest}

        This sample application create a new file with open(), writes a message to it, closes the file descriptor, and
        exit()'s. This test is particularly meaningful as it replicates a pattern of system calls that is used for in
        all sorts of applications in Linux...

        \paragraph{Discussion on mv}

        In addition to the newly constructed sample programs, performance values for replay of an execution of {\tt
          mv} moving a file between two separate disks were recorded.  Results indicate that CrashSimulator is able to
        replay this operation in a similar time period to the time in which it is able to replay the dramatically
        simpler programs.  The fact that {\tt mv} makes an order of magnitude more system calls did not substantially
        increase replay time.

            
    \subsection{Bugs Identified in Major Applications}

        \subsubsection{Identification of Known Bugs}

        Another way we evaluated the efficacy of CrashSimulator was on its ability to detect known bugs as recorded in
        the bug trackers of major open source software projects.

            \paragraph{Race Condition in shutil.copy, shutil.copy2, shutil.copyfile - Bug ID\# 15100}
              
            This bug describes a situation where the described functions in the shutil package do not perform the checks
            necessary to prevent a race condition around the process of copying a file from one filesystem to another --
            a situation that prevents the use of the safer {\tt rename()} system call.  CrashSimulator is able to detect this
            issue in replayed executions of python applications that make use of these functions by identifying the
            applications failure to use {\tt fstat()} to verify that a file has not changed between the time the file was
            first checked with {\tt stat()} and when it was open()'d at the start of the copy process.

        \subsubsection{Identification of Unknown Bugs}

        \subsubsection{Cross-Device Move Bugs}

        In Linux, the rename() system call will only move a file if the source and destination are on the same device.
        This means that moving a file from a directory structure location on one storage device to a directory structure location on
        another device must be handled on a case-by-case basis by any application that relies on this operation.  With
        this in mined, we examined the process by which the coreutils {\tt mv}command handles this task and constructed
        computational models that can determine whether or not a replayed execution of an application has performed all
        the necessary steps to successfully carry out a cross device move.  Next, these models were employed during
        replayed executions of the listed applications in order to determine whether or not their copy operation is correct.

        \paragraph{Verify Destination is not Target of Source}

        In this condition, CrashSimulator confirms whether or not the application under test performs a check to ensure
        that the source file name is not a symlink pointing to the destination file name.  If this check is not
        performed, loss of data is possible due to removal or overwriting of the destination during the copy process
        resulting in the source symlink pointing to nothing. \emph{!!!! This is likely one that we would have trouble
          detecting based only on system call behavior}

        \paragraph{Preserve Extended File Attributes}

        In this condition, CrashSimulator determines whether or not the application under test correctly preserves
        extended file attributes by reading them from the source file and applying them to the destination file prior to
        removing the source file.  Extended file attributes are often used to store information such as the original
        providance of the file (i.e. downloaded from the internet, received from an trusted vendor, etc.).  Loss of
        these attributes can result in security issues. For example, Apple's Gatekeeper relies on extended file
        attributes to prevent applications downloaded untrusted developers from being executed without user
        confirmation. \emph{Detect passively by monitoring a trace of cross-device copy}


        \paragraph{Ensure Source is not Replaced Between Check and Copy}

        This is an example of a classic race condition.  In this case, CrashSimulator examines whether or not the
        appication under test makes uses of fstat() to ensure that the inode number of the file being copied has not
        changed between initial examination with a stat()-like call and the eventual copy. \emph{Detect passively by
          monitoring a trace of cross-device copy}

        \paragraph{Preserve Timestamp and Mode}

        Truely copying a file means preserving the metadata of the file as well as its contents.  In this condition,
        CrashSimulator ascertains whether or not the application under test restores the appropriate timestamps and
        modeline to the destination file after (or before) its contents have been copied. \emph{Detect passively by
          monitoring a trace of cross-device copy}

        \paragraph{Move Infinite Character Device Across Disks}

        Many applications fail to examine the nature of a file before engaging in a manual, cross-disk copy process.  In
        situations where the source is a special file such as /dev/urandom.  If an application fails to detect this
        situation, the typical steps in a manual copy process could result in the application filling the destination
        device, consuming available memory, or simply hanging indefinitely as it attempts to read in the full contents
        of an effectively infinite-sized file. \emph{Detect actively by injecting st_mode=S_IFCHR, or passively by
          monitoring trace where an actual /dev/urandom copy is attempted -Preston}

\begin{figure*}[t]
 %           \begin{table}%[t]
                \scriptsize{}
                \begin{tabular}{l l l l l l | l}
                \toprule{}
                  Condition & mv & mmv & shutils & rust & boost::copyfile & Mode\\
                  Verify Destination is not Target of Source & Yes & Yes & No & ??? & No & Passive\\
                  Preserve Extended File Attributes & Yes & No & No & No & No & Passive\\
                  File Replaced Between Check and Copy & Yes & No & No & Yes & No & Passive\\
                  Preserve Timestamp and Mode & Yes & No & Yes & Yes & No & Passive\\
                  Move Directory Into Itself & Yes & Yes & Yes & N/A, won't move directories & N/A, won't move directories & Passive\\
                  Move Block Device Across Disks & Yes & Yes & Yes & No & No & Injected\\
                \bottomrule{}
                \end{tabular}
 %           \end{table}
\end{figure*}

        \subsubsection{Non-Regular File Bugs}

        These bugs involve modifying the executions of applications that work with regular files such that a call to
        {\tt stat()} or {\tt lstat()} indicates that the file in question is instead some sort of special file.  Well
        behaved applications should verify that the files they are accessing are regular files before processing them.
        Many applications make the assumption that they will only be used to process regular files.  Encountering
        special files tends to induce a denial of service condition that in certain circumstances, such as an automated
        environment, can halt a scripted process or workflow.  The applications below were chosen either because of
        their widespread use (as is the case with vim, nano, sed, and gnu-gpg) or their presence in the GNU Coreutils
        package.  When selecting applications for this work applications with non-trivial file manipulation and data
        processing were chosen.  Applications that acted only as a dumb wrapper around some operating system
        functionality were not examined.

            \begin{table*}[t]
                \scriptsize{}
                \begin{tabular}{l  l  l  | l}
                \toprule{}
                  Application & Condition Tested           & S_IFREG        & S_IFDIR        & S_IFCHR     & S_IFBLK    & S_FIFO      & S_IFLNK    & S_IFSOCK\\
                  Aspell      & Dictionary File            & Initial Value  & Fail           & Recognizes  & Fail       & Fail        & Fail       & Fail\\
                  Aspell      & File being checked         & Initial Value  & Fail           & Recognizes  & Fail       & Fail        & Fail       & Fail\\
                  gnu-gpg     & secring.gpg                & Initial Value  & Fail           & Fail        & Fail       & Fail        & Fail       & Fail\\
                  vim         & File being opened          & Initial Value  & Recognizes     & Recognizes  & Recognizes & Recognizes* & Recognizes & Fail\\
                  nano        & File being opened          & Initial Value  & Recognizes     & Recognizes  & Recognizes & Fail        & Fail       & Fail\\
                  sed         & File being edited          & Initial Value  & Fail           & Recognizes  & Fail       & Fail        & Fail       & Fail\\
                  df          & /proc                      & Fail           & Initial Value  & Fail        & Fail       & Fail        & Fail       & Fail\\
                  wc          & File being checked         & Initial Value  & Recognizes     & Recognizes  & Recognizes & Recognizes  & Recognizes & Recognizes\\
                  du          & Directory being checked    & Recognizes     & Initial Value  & Recognizes  & Recognizes & Recognizes  & Recognizes & Recognizes\\
                  install     & File being installed       & Initial Value  & Recognizes     & Fails       & Fails      & Fails       & Recognizes & Fails\\
                  fmt         & File being formatted       & Initial Value  & Fails          & Recognizes  & Fails      & Fails       & Fails      & Fails\\
                  od          & File being dumped          & Initial Value  & Fails          & Recognizes  & Fails      & Fails       & Fails      & Fails\\
                  ptx         & File being read            & Initial Value  & Recognizes     & Recognizes  & Recognizes & Recognizes  & Recognizes & Recognizes\\
                  comm        & Second file being compared & Initial Value  & Fail           & Recognizes  & Fail       & Fail        & Fail       & Fail\\
                  pr          & File being read            & Initial Value  & Fail           & Fail        & Fail       & Fail        & Fail       & Fail\\
                \bottomrule{}
                \end{tabular}
            \end{table*}

    \subsection{Dealing with False Positives}

    A the primary source of false positives for CrashSimulator is an application using a different sequence of system
    calls than a checker was anticipating to implement a given operation.  Fortuantely, CrashSimulator's approach allows
    these situations to be easily corrected once identified.

    In many cases a given operation can performed in several ways.  This means that a given operation may be correctly
    implemented by a set of system call sequences that differ from each other on a variety of dimensions.  In order to
    correctly classify applications that make use of alternative methods for a given operation the compuational models
    CrashSimulator relies on must accept the associated alternative system call sequences.

    As an example consider GNOME's glib file handling facilties.  When an application makes use of these facilities to move
    a file across storage devices the library itself correctly handles the above behaviors we definied earlier as making up
    a "correct" file move operation.  When we used CrashSimulator to analyze a minimal application using glib to perform
    this operation we initially received failing results.  We found through manual examination of a the system call trace
    used in this analysis that, while glib correctly performs the required behaviors, it does so using alternative system
    call sequences.  For example, the typical way to move the actual contents of a file from one file to another is to {\tt
      read()} the data from the source file and {\tt write()} the data back out to the destination file.  Glib instead
    creates a pipe and uses the {\tt splice()} system call to copy the contents out of the source file, through the pipe,
    and into the destination file.

    This approach is viable because there are a finite number of system calls, and a given operation can be mapped to a
    managable subset of system calls.  Given the above example around moving files, consider the following mapping from
    high level ``operation'' to the set of system calls that can implement it.

        \begin{table*}[t]
            \scriptsize{}
            \begin{tabular}{l | l }
            \toprule{}
              Operation                                               & Potential System calls\\
              Examine source file                                     & stat64(), lstat64(), fstat64()\\
              Examine destination file                                & stat64(), lstat64(), fstat64()\\
              Open source file                                        & open()\\
              Read contents of source file                            & read(), splice() with a pipe\\
              List source file's extended file attributes             & listxattr(), llistxattr(), flistxattr()\\
              Read contents of source file's extended file attributes & getxattr(), lgetxattr(), fgetxattr()\\
              Open destination file                                   & open(), optionally unlink() the file first\\
              Write contents to destination file                      & write(), splice() with a pipe\\
              Apply extended file attributes to destination file      & setxattr(), lsetxattr(), fsetxattr()\\
              Apply proper timestamps to destination file             & utimens(), futimens()\\
              Apply proper permissions to destination file            & chmod()\\
              Close the source file                                   & close()\\
              Close the destination file                              & close()\\

              
            \bottomrule{}
            \end{tabular}
        \end{table*}

    
   In situations where two system call sequences can correctly implement the same operation CrashSimulator simply runs
   two computational models in parallel and accepts the execution if either model ends in an accepting condition. 

    \subsection{Limitations}

        The current implementation of CrashSimulator has the following limitations that could be addressed by future work.

        \paragraph{Coupling to Architecture}

        As some faults injected by CrashSimulator require low level access to the test system's hardware or operating
        system data structures there exists some degree of coupling between CrashSimulator and these components. One
        area of expansion for CrashSimulator is support for more processor architectures and more operating systems.
        CrashSimulator's test launcher as been designed in such a way that these improvements should be trivial to plug
        in once they have been implemented.

        \paragraph{Multi-threaded or Multi-process Applications}

        The current implementation cannot correctly replay applications that rely on multi-threading or
        multi-processing.  Is due to two factors.  First, the replay tool must be able to force some pre-determined
        order onto the threads or processes in order to prevent situations where the execution portion of a replay ends
        up requesting system calls in a different order than was recorded in the system call trace the system is
        attempting to replay.  Second, the tool needs to be able to correctly monitor multiple processes. {\tt Ptrace}
        may have appropriate facilities for this task but they were not explored for this version of the tool.

        \paragraph{Multi-Platform Tracing Tool}

        While there are system call tracing tools available for most of the major operating systems available today
        there is not one tool that works across all of them.  This means that if CrashSimulator is going to be able to
        to work with traces from a given operating system, it will require specific parsing logic for the output of the
        tool used to record the traces.  Additionally, CrashSimulator requires that the tracing tools record all data
        passed into a system call and all data returned from the system call either as a return value or as some block
        of memory written into the calling process's memory.  It is possible that some system call tracing tools do not
        support this level of detail.

        \paragraph{Bugs Outside the View of CrashSimulator}

        CrashSimulator is not able to detect bugs that do not result from or have a visible effect on the system calls
        an application makes.  An example of this situation exposed during work on this evaluation is applications
        mishandling a situation around file moves where the where, due to a strange volume mounting arrangement, the
        source directory is moved into itself.  In order to detect this situation the application must make an effort to
        gather the full path of both the source and destination and compare them.  While CrashSimulator is able to
        observe some aspects of the ``gathering'' process (i.e. the stat() or lstat() calls) it is not able to observe
        whether a proper comparison takes place.  As a result, the best a CrashSimulator checker could do in this
        situation is observe a lack of gathering-related system calls and issue a warning that the issue might be
        present.  \emph{CrashSimulator could detect this issue during a specially crafted execution which replays a
          trace recorded where the issue was present.  I'm not sure how useful this would be as presumably the while
          recording the trace the user would be able to directly observe whether things happened correctly or not}
        % Maybe discuss this as related work