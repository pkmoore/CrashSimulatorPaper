\section{Evaluation}

    \subsection{Bugs Identified}

    %% Intro paragraph
    First, CrashSimulator was evaluated against NUMBER of popular tools with public bug trackers available. The goal was
    to identify bugs from each tool's bug tracker (both fixed and unfixed) that are related to the faults that
    CrashSimulator can inject. Next, CrashSimulator was run against these tools fault counts were collected. A
    comparison of the number of faults identified by the tool's users in the tool's bug tracker to the number of faults
    identified by CrashSimulator is presented in the table below.

    \emph{\textbf{INSERT TABLE HERE}}

    \subsection{Limitations}

        Multi-threaded stuff is an issue for both NetCheck and CheckAPI

        %% This text will need to be updated based on discussion about handling testing of interpreted languages
        Because CrashSimulator operates on the system calls made by an application it does not make any attempt to
        determine what the cause of a fault may have been at a higher level. For example, in situations where
        CrashSimulator is testing an application written in an interpreted language the possibility exists that faults
        will be found in the interpreter rather than the application itself. For example, CrashSimulator may modify
        system calls made by the interpreter for purposes that are independent from the application under test.  If this
        results results in improper output CrashSimulator will simply report it as a fault in the application despite
        the fact that the user's code was not responsible for the error.
