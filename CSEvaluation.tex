\section{Evaluation}

    In order to measure its efficacy CrashSimulator was implemented in Python. Python was chosen in order to facilitate
    code reuse from the previous work on CheckAPI and NetCheck. This implementation of CrashSimulator was evaluated on
    the basis of execution performance and number of bugs detected across a set of small test programs and larger, more
    mainstream applications.


    %%%% Talk about the test platform, OS, Hardware, etc.

    \subsection{Execution Performance}

        One key attribute of successful testing tools is that they be able to complete their tests in a timely manner.
        If a tool takes too long to complete its tests users will be less likely to run it frequently, or at all,
        reducing the tools overall usefulness dramatically. To this end, the performance of CrashSimulator was evaluated
        in order to determine whether or not it was able to complete its test executions in an acceptable time frame.
        \textbf{There probably needs to be a citation here related to how long a developer is typically willing to wait
        for a test execution to complete}

        \subsubsection{Evaluation Against Sample Programs}

            The Python implementation of CrashSimulator was performance
            tested against two sample programs that were seeded with specific counts of potential anomalies. The two sample
            programs were seeded as follows:

            \begin{tabular}{l  l  l  l}
                \toprule{}
                    Title & Return Value Modification & Catalog Based & Data Fragmentation \\
                    Sample A & 20 & 2 & 2 \\
                    Sample B & 100 & 20 & 20 \\
                \bottomrule{}
            \end{tabular}

            These sample applications were written with two goals in mind. First, their system call related behavior
            should be deterministic and repeatable in nature. Second, they should be able to run to completion without
            user interaction and with a completion time that would allow for hundreds of executions in a reasonable
            amount of time. In short, the sample programs allow an accurate evaluation of CrashSimulator's performance
            by acting as ideal testing candidates.

            Performance tests using the sample programs were performed five times in order to account for differences in
            execution times related varying resource loads on the test hardware at the time the tests were run. The
            results are recorded below. Repeated executions of the sample programs outside of crash simulator provide a
            control run time for comparison to times recorded from the CrashSimulator test sessions. Because
            CrashSimulator will potentially execute the application under test \emph{thousands} of times, it is not
            appropriate to compare test run times to the run time of a single run of the application.

            \begin{tabular} {l  l  l}
                \toprule{}
                This Table Consists Entirely of \textbf{FAKE DATA} \\
                Title & Number of Executions & Run Time \\
                Sample A (No CrashSimulator Tests) & 10000 & 1m 50s \\
                Sample B (No CrashSimulator Tests) & 10000 & 2m 30s \\
                Sample A (CrashSimulator) & 10000 & 2m 13s \\
                Sample B (CrashSimulator) & 10000 & 3m 15s \\
                \bottomrule{}
            \end{tabular}

            As expected, CrashSimulator is responsible for adding some overhead as compared to the non-CrashSimulator
            executions of the sample applications. For each sample program, a group of executions through CrashSimulator
            took approximately 30 percent longer to complete than an identically sized group of executions performed by
            repeatedly executing the sample program.

        \subsubsection{Evaluation Against Major Programs}

            Next, CrashSimulator's performance was evaluated against two major open source applications: Firefox and
            Apache 2. In these cases, the evaluation methodology required modification because these applications are
            not specifically engineered to meet the same criteria the sample programs were. Firefox, on one hand,
            typically requires user input in order carry out its purpose of retrieving content hosted on a remote
            server. On the other hand, Apache 2 runs as a background service and is intended under ideal conditions to
            continually serve content indefinitely. In both cases, the complexity of these pieces of software result in
            system call traces that are likely not deterministic or repeatable.

            \paragraph{Firefox}

                Because Firefox is a large application with a great deal of complex functionality and behavior it was
                determined that a specific use case should be identified for the purposes of evaluating CrashSimulator's
                performance. The primary use of Firefox for most of its users is the retrieval, rendering and display of
                content from a remote server. In the vast majority of cases this is web content retrieved over HTTP As
                such, it was decided that this use case would be appropriate for CrashSimulator's performance
                evaluation.

                In order to achieve consistent test runs the following setup was constructed. The Firefox scripting API
                was used in order to modify the behavior of the browser in two key aspects. First, the browser was
                modified to automatically request a static web page with no dynamic content from a server hosted and
                operated by the authors. A static web page was chosen to reduce the chances of any script parsing and
                execution code present in Firefox confounding the run time measurements. Second, Firefox's scripting API
                was used to end execution of the browser immediately upon the completion of the page's rendering. This
                allows for the measurement of one complete ``execution'' of Firefox from start to finish. Additionally,
                Firefox was configured to cache no content so each run was required to perform the complete set of
                communications with the server necessary to retrieve the static web page prior to displaying it. In
                order to reduce the effect of network latency on execution times both the client machine running Firefox
                and the server machine serving the static web page were located on the same network. Also, the server
                software was configured to cache no content in order to reduce the chances of server side resource
                caching affecting execution times.

                Actual evaluation of CrashSimulator's performance was conducted in a similar manner to the methodology
                described for the sample programs. The scripted version of Firefox was executed a number of times by
                CrashSimulator for testing purposes and the total time required to complete these test executions was
                recorded. For comparison purposes, the scripted version of Firefox was executed the same number of times
                \emph{without} CrashSimulator.  The following table contains the total execution times required to
                complete the listed number of executions.

                \begin{tabular}{l l l}
                    \toprule{}
                    Title & Number Of Executions & Execution Time \\
                    \multirow{3}{*}{Firefox} \\
                    & 1 & 300ms \\
                    & 10 & 3000ms \\
                    & 100 & 30000ms \\
                    \multirow{3}{*}{Firefox (CrashSimulator)} \\
                    & 1 & 450ms \\
                    & 10 & 4500ms \\
                    & 100 & 45000ms \\
                    \bottomrule{}
                \end{tabular}

                From these results it is apparent that CrashSimulator  increased the time required to complete a given
                number of executions of the modified version of Firefox. The results above show an approximately 50
                percent increase in run time between the non-CrashSimulator and CrashSimulator runs. This increase is
                greater than the increase observed during the sample program evaluations. This is likely due to the high
                count of opportunities for fault injection present in a complex piece of software like Firefox as
                opposed to the sample programs. From this, it can be inferred that CrashSimulator would cause a similar
                increase in the amount of time required to complete a test of a standard version of Firefox. That said,
                the total time is still well within what the authors define as a reasonable time frame for test suite
                completion. Given the high degree of success CrashSimulator sees in identifying faults in the
                applications under test, the authors do not see this increase in execution time as a significant
                impediment to the adoption of CrashSimulator as part of a rigorous testing process.

            \paragraph{Apache}

                Evaluating CrashSimulator's performance against Apache saw similar challenges to its evaluation against
                Firefox. Apache is a complex piece of software and, while its normal use cases don't require user
                interaction, it presents challenges due to its typically long-lived nature. Apache's primary use case is
                evaluation methodology should primarily exercise this functionality to serve content to a client program
                upon the client's request. As such, it was determined that the evaluation methodology should primarily
                exercise this functionality evaluation methodology should primarily exercise this functionality. To this
                end, the following evaluation setup was constructed.

                For this evaluation one execution of Apache consists of the following steps:

                \begin{enumerate}
                    \item{} Launch an instance of Apache
                    \item{} Request the static web page from Apache
                    \item{} Wait for transmission of the static web page to complete
                    \item{} Terminate the execution of Apache
                \end{enumerate}

                A Python script was written in order to execute these steps in a consistent manner. First, an instance
                of Apache was launched. This instance of Apache was configured to serve a single directory on the host
                machine using a single static host entry. This directory contained a single static web page with 100
                bytes of HTML content. Apache was also configured to perform \emph{no} caching of local resources. This
                setup was chosen in order to reduce the amount of extraneous functionality executed by Apache. A single
                static host entry eliminates Apache's virtual host parsing and a single static HTML document eliminates
                the need for apache to execute any external parsing engine such as mod\_php. Once Apache has completed
                its startup, the script requests the static web page using the \emph{wget} utility. The page is not
                rendered or displayed in any way upon receipt as these processes are not relevant to this evaluation.
                Once the web page has been retrieved the script instructs the Apache instance to terminate and waits for
                it to do so. Termination of the Apache instance concludes the measured run time for the execution.
                Because the control script needed to be able to directly manage both client and server activates, in
                this setup, both the ``client'' script and the Apache instance were run on the same machine.



    \subsection{Bugs Identified}

    First, CrashSimulator was evaluated against NUMBER of popular tools with public bug trackers available. The goal was
    to identify bugs from each tool's bug tracker (both fixed and unfixed) that are related to the faults that
    CrashSimulator can inject. Next, CrashSimulator was run against these tools fault counts were collected. A
    comparison of the number of faults identified by the tool's users in the tool's bug tracker to the number of faults
    identified by CrashSimulator is presented in the table below.

    \emph{\textbf{INSERT TABLE HERE}}

    \subsection{Limitations}

        Multi-threaded stuff is an issue for both NetCheck and CheckAPI

        %% This text will need to be updated based on discussion about handling testing of interpreted languages
        Because CrashSimulator operates on the system calls made by an application it does not make any attempt to
        determine what the cause of a fault may have been at a higher level. For example, in situations where
        CrashSimulator is testing an application written in an interpreted language the possibility exists that faults
        will be found in the interpreter rather than the application itself. For example, CrashSimulator may modify
        system calls made by the interpreter for purposes that are independent from the application under test.  If this
        results results in improper output CrashSimulator will simply report it as a fault in the application despite
        the fact that the user's code was not responsible for the error.
