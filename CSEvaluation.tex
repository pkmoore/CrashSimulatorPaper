\section{Evaluation}

    In order to measure its efficacy CrashSimulator was implemented in Python. Python was chosen in order to facilitate
    code reuse from the previous work on CheckAPI and NetCheck. This implementation of CrashSimulator was evaluated on
    the basis of execution performance and number of bugs detected across a set of small test programs and larger, more
    mainstream applications.

    % TODO: Talk about the test platform, OS, Hardware, etc.
    \subsection{Execution Performance}

        One key attribute of successful testing tools is that they be able to complete their tests in a timely manner.
        If a tool takes too long to complete its tests users will be less likely to run it frequently, or at all,
        reducing the tools overall usefulness dramatically. To this end, the performance of CrashSimulator was evaluated
        in order to determine whether or not it was able to complete its test executions in an acceptable time frame.
        \textbf{There probably needs to be a citation here related to how long a developer is typically willing to wait
        for a test execution to complete}

        \subsubsection{Evaluation Against Sample Programs}

            The Python implementation of CrashSimulator was performance
            tested against two sample programs that were seeded with specific counts of potential anomalies. The two sample
            programs were seeded as follows:

            \begin{table}[H]
                \scriptsize{}
                \begin{tabular}{l  l  l  l}
                    \toprule{}
                        Title & Return Value Modification & Catalog Based & Data Fragmentation \\
                        Sample A & 20 & 2 & 2 \\
                        Sample B & 100 & 20 & 20 \\
                    \bottomrule{}
                \end{tabular}
            \end{table}

            These sample applications were written with two goals in mind. First, their system call related behavior
            should be deterministic and repeatable in nature. Second, they should be able to run to completion without
            user interaction and with a completion time that would allow for hundreds of executions in a reasonable
            amount of time. In short, the sample programs allow an accurate evaluation of CrashSimulator's performance
            by acting as ideal testing candidates.

            Performance tests using the sample programs were performed five times in order to account for differences in
            execution times related varying resource loads on the test hardware at the time the tests were run. The
            results are recorded below. Repeated executions of the sample programs outside of crash simulator provide a
            control run time for comparison to times recorded from the CrashSimulator test sessions. Because
            CrashSimulator will potentially execute the application under test \emph{thousands} of times, it is not
            appropriate to compare test run times to the run time of a single run of the application.

            \begin{table}[H]
                \scriptsize{}
                \begin{tabular} {l  l  l}
                    \toprule{}
                    This Table Consists Entirely of \textbf{FAKE DATA} \\
                    Title & Number of Executions & Run Time \\
                    Sample A (No CrashSimulator Tests) & 10000 & 1m 50s \\
                    Sample B (No CrashSimulator Tests) & 10000 & 2m 30s \\
                    Sample A (CrashSimulator) & 10000 & 2m 13s \\
                    Sample B (CrashSimulator) & 10000 & 3m 15s \\
                    \bottomrule{}
                \end{tabular}
            \end{table}

            As expected, CrashSimulator is responsible for adding some overhead as compared to the non-CrashSimulator
            executions of the sample applications. For each sample program, a group of executions through CrashSimulator
            took approximately 30 percent longer to complete than an identically sized group of executions performed by
            repeatedly executing the sample program.

        \subsubsection{Evaluation Against Major Programs}

            Next, CrashSimulator's performance was evaluated against two major open source applications: Firefox and
            Apache 2. In these cases, the evaluation methodology required modification because these applications are
            not specifically engineered to meet the same criteria the sample programs were. Firefox, on one hand,
            typically requires user input in order carry out its purpose of retrieving content hosted on a remote
            server. On the other hand, Apache 2 runs as a background service and is intended under ideal conditions to
            continually serve content indefinitely. In both cases, the complexity of these pieces of software result in
            system call traces that are likely not deterministic or repeatable.

            \paragraph{Firefox}

                Because Firefox is a large application with a great deal of complex functionality and behavior it was
                determined that a specific use case should be identified for the purposes of evaluating CrashSimulator's
                performance. The primary use of Firefox for most of its users is the retrieval, rendering and display of
                content from a remote server. In the vast majority of cases this is web content retrieved over HTTP As
                such, it was decided that this use case would be appropriate for CrashSimulator's performance
                evaluation.

                In order to achieve consistent test runs the following setup was constructed. The Firefox scripting API
                was used in order to modify the behavior of the browser in two key aspects. First, the browser was
                modified to automatically request a static web page with no dynamic content from a server hosted and
                operated by the authors. A static web page was chosen to reduce the chances of any script parsing and
                execution code present in Firefox confounding the run time measurements. Second, Firefox's scripting API
                was used to end execution of the browser immediately upon the completion of the page's rendering. This
                allows for the measurement of one complete ``execution'' of Firefox from start to finish. Additionally,
                Firefox was configured to cache no content so each run was required to perform the complete set of
                communications with the server necessary to retrieve the static web page prior to displaying it. In
                order to reduce the effect of network latency on execution times both the client machine running Firefox
                and the server machine serving the static web page were located on the same network. Also, the server
                software was configured to cache no content in order to reduce the chances of server side resource
                caching affecting execution times.

                Actual evaluation of CrashSimulator's performance was conducted in a similar manner to the methodology
                described for the sample programs. The scripted version of Firefox was executed a number of times by
                CrashSimulator for testing purposes and the total time required to complete these test executions was
                recorded. For comparison purposes, the scripted version of Firefox was executed the same number of times
                \emph{without} CrashSimulator.  The following table contains the total execution times required to
                complete the listed number of executions.

                \begin{table}[H]
                    \scriptsize{}
                    \begin{tabular}{l l l}
                        \toprule{}
                        Title & Number Of Executions & Execution Time \\
                        \multirow{3}{*}{Firefox} \\
                        & 1 & 300ms \\
                        & 10 & 3000ms \\
                        & 100 & 30000ms \\
                        \multirow{3}{*}{Firefox (CrashSimulator)} \\
                        & 1 & 450ms \\
                        & 10 & 4500ms \\
                        & 100 & 45000ms \\
                        \bottomrule{}
                    \end{tabular}
                \end{table}

                From these results it is apparent that CrashSimulator  increased the time required to complete a given
                number of executions of the modified version of Firefox. The results above show an approximately 50
                percent increase in run time between the non-CrashSimulator and CrashSimulator runs. This increase is
                greater than the increase observed during the sample program evaluations. This is likely due to the high
                count of opportunities for fault injection present in a complex piece of software like Firefox as
                opposed to the sample programs. From this, it can be inferred that CrashSimulator would cause a similar
                increase in the amount of time required to complete a test of a standard version of Firefox.

            \paragraph{Apache}

                Evaluating CrashSimulator's performance against Apache saw similar challenges to its evaluation against
                Firefox. Apache is a complex piece of software and, while its normal use cases don't require user
                interaction, it presents challenges due to its typically long-lived nature. Apache's primary use case is
                evaluation methodology should primarily exercise this functionality to serve content to a client program
                upon the client's request. As such, it was determined that the evaluation methodology should primarily
                exercise this functionality evaluation methodology should primarily exercise this functionality. To this
                end, the following evaluation setup was constructed.

                For this evaluation one execution of Apache consists of the following steps:

                \begin{enumerate}
                    \item{} Launch an instance of Apache
                    \item{} Request the static web page from Apache
                    \item{} Wait for transmission of the static web page to complete
                    \item{} Terminate the execution of Apache
                \end{enumerate}

                A Python script was written in order to execute these steps in a consistent manner. First, an instance
                of Apache was launched. This instance of Apache was configured to serve a single directory on the host
                machine using a single static host entry. This directory contained a single static web page with 100
                bytes of HTML content. Apache was also configured to perform \emph{no} caching of local resources. This
                setup was chosen in order to reduce the amount of extraneous functionality executed by Apache. A single
                static host entry eliminates Apache's virtual host parsing and a single static HTML document eliminates
                the need for apache to execute any external parsing engine such as mod\_php. Once Apache has completed
                its startup, the script requests the static web page using the \emph{wget} utility. The page is not
                rendered or displayed in any way upon receipt as these processes are not relevant to this evaluation.
                Once the web page has been retrieved the script instructs the Apache instance to terminate and waits for
                it to do so. Termination of the Apache instance concludes the measured run time for the execution.
                Because the control script needed to be able to directly manage both client and server activates, in
                this setup, both the ``client'' script and the Apache instance were run on the same machine.

                Once again, CrashSimulator's performance was evaluated in terms of the run time of CrashSimulator's
                executions versus the run time of the same number of executions performed without CrashSimulator's
                involvement. The actual run times measured are for the complete execution for the Apache control script
                described above. As such, some run time accumulated is the result of non-Apache client code and control
                script overhead. The following table contains the results collected.

                \begin{table}[H]
                    \scriptsize{}
                    \begin{tabular}{l l l}
                        \toprule{}
                        Title & Number Of Executions & Execution Time \\
                        \multirow{3}{*}{Apache Control Script} \\
                        & 1 & 300ms \\
                        & 10 & 3000ms \\
                        & 100 & 30000ms \\
                        \multirow{3}{*}{Apache Control Script (CrashSimulator)} \\
                        & 1 & 450ms \\
                        & 10 & 4500ms \\
                        & 100 & 45000ms \\
                        \bottomrule{}
                    \end{tabular}
                \end{table}

                As expected, the run times of the CrashSimulator-involved executions shows an increase over the
                non-CrashSimulator executions. The results above show an approximately 50 percent increase in run time
                between the CrashSimulator controlled executions and the non-CrashSimulator executions. This increase is
                in line with what was expected based on the previously discussed Firefox evaluation for similar reasons
                related to complexity. One difference with this evaluation is there is a higher likely-hood of the
                accumulated run time being affected by overhead introduced by the control script. That said, this
                overhead would be similar across all executions.

                This evaluation has shown that CrashSimulator causes an increase in run time for executions of both the
                sample programs and common major applications. That said, the total time is still well within what the
                authors define as a reasonable time frame for test suite completion. Given the high degree of success
                CrashSimulator sees in identifying faults in the applications under test, the authors do not see this
                increase in execution time as a significant impediment to the adoption of CrashSimulator as part of a
                rigorous testing process.

    \subsection{Bugs Identified}

        A second way CrashSimulator was evaluated was on the basis of how well it was able to independently identify
        known bugs that were listed in the bug tracking suites of two major open source projects. Firefox and Apache 2
        were chosen because of their use in the previously discussed evaluations. The goal was to identify bugs from
        each tool's bug tracker (both fixed and unfixed) that are related to the faults that CrashSimulator can inject
        and determine whether or not CrashSimulator was able to identify them. As a side effect, this process also
        yielded a number of new bugs that were previously unreported in these projects' bug trackers.

        \subsection{Firefox}

            For this evaluation, the same Firefox test setup as was described in the Firefox performance evaluation was
            used.

            \textbf{\emph{Note: This is fake data. Also, I would like to have better ways of quantifying some of this}}
            From Firefox's bug tracker 10 bugs in the ``networking'' category with successful resolutions were selected
            as good candidates for this portion of CrashSimulator's evaluation. Successfully resolved bugs were chosen
            because they had sufficient documentation to identify their exact cause. This was essential to determining
            the likelihood of their being identified.

            Five of the bugs were judged as ``likely to be identified by CrashSimulator'' by the authors. This opinion
            was based on how closely the abnormal behavior, and its resolution, described in the bug report related to
            faults that CrashSimulator can inject. \textbf{\emph{FAKE}} For example, one of these bugs related to a
            failure in Firefox's handling of DNS responses due to duplicated UDP datagrams. This is a fault that
            CrashSimulator is readily able to inject so it was expected that CrashSimulator would cause the fault to
            occur and report it.

            The other five bugs where judged by the authors as ``unlikely to be identified by CrashSimulator.'' These
            bugs were, once again, selected from the ``networking'' category and had successful resolutions. These bugs
            were chosen because the abnormal behavior they produced and their resolutions indicated that they were
            caused by faults that CrashSimulator does not currently have the capability of injecting. For example, one
            bug related to an application logic error related to a specific edge case malformation in a TCP packet
            received from a particular version of the IIS web server. CrashSimulator is unlikely identify this bug
            because it does not support the specific modification of TCP packets required to trigger it.

            The identification status of each bug is listed in the table below.

            \begin{table}[H]
                \scriptsize{}
                \begin{tabular}{l  l  l  l}
                    \toprule{}
                        Bug Tracker ID & Likely to be Identified & Identified & Notes \\
                        0001 & True & True & Note \\
                        0002 & True & True & Note \\
                        0003 & True & True & Note \\
                        0004 & True & True & Note \\
                        0005 & True & True & Note \\
                        0006 & False & False & Note \\
                        0007 & False & False & Note \\
                        0008 & False & False & Note \\
                        0009 & False & False & Note \\
                        0010 & False & False & Note \\
                    \bottomrule{}
                \end{tabular}
            \end{table}

            As can be see from the results above, CrashSimulator performed as expected based on the selected bug's
            likeliness to be identified. In addition, CrashSimulator also identified a previously unreported bug in
            Firefox's network code. This bug results in a denial of service condition caused by a failure to check the
            return value of a call to \emph{socket} and the subsequent use of the invalid socket handle it can produce.
            This bug was reported to Mozilla (bug ID\# 0011) and was corrected in the next release of Firefox.

        \subsection{Apache 2}

            \textbf{\emph{This is fake, placeholder data and information}}
            For this evaluation, the same setup as was described in the Apache performance evaluation was used.

            Like the Firefox evaluation, a set of bugs with successful resolutions were chosen from Apache's bug tracker
            from the ``networking and communication'' category. Once again, these bugs were divided into ``likely to be
            identified by CrashSimulator'' and ``not likely to be identified by CrashSimulator'' based on an analysis of
            their root cause and eventual resolution. The table below lists the bugs that were chosen, whether or not
            they were likely to be identified and whether or not CrashSimulator identified them in the end.

            \begin{table}[H]
                \scriptsize{}
                \begin{tabular}{l  l  l  l}
                    \toprule{}
                        Bug Tracker ID & Likely to be Identified & Identified & Notes \\
                        0001 & True & True & Note \\
                        0002 & True & True & Note \\
                        0003 & True & True & Note \\
                        0004 & True & True & Note \\
                        0005 & True & True & Note \\
                        0006 & False & False & Note \\
                        0007 & False & False & Note \\
                        0008 & False & False & Note \\
                        0009 & False & False & Note \\
                        0010 & False & False & Note \\
                    \bottomrule{}
                \end{tabular}
            \end{table}

            In addition to these bugs, one yet undiscovered bug was identified during the course of this evaluation.
            The bug related to a denial of service condition caused by one of Apache's low level network operations
            failing to handling low MTU values. This bug was reported to Apache's bug tracker (Bug ID\# 2007) and was
            successfully resolved in the latest release.

    \subsection{Test Coverage Improvement}

        \textbf{\emph{There needs to be more exploration put into this test methodology. Much more description needs to
        be written.}}

        A final way CrashSimulator was evaluated was on the extent to which it was able to improve upon the existing
        unit test suites of major applications make them available publicly. At a high level, this evaluation was
        performed as follows. The unit test suites of Firefox and Apache were reviewed and a count was made of tests
        that fell into CrashSimulator's domain was made. Next, a CrashSimulator was run against each application and a
        count of tests performed was recorded. These values were used to determine a percent increase number of useful
        tests ``added'' through the use of CrashSimulator

    \subsection{Limitations}

        Evaluation of CrashSimulator has yielded the following limitations to be addressed by future work.

        \paragraph{Coupling to Architecture}

            As some faults injected by CrashSimulator require low level access to the test system's hardware or
            operating system data structures there exists some degree of coupling between CrashSimulator and these
            components. One area of expansion for CrashSimulator is support for more processor architectures and more
            operating systems.  CrashSimulator's test launcher as been designed in such a way that these improvements
            should be trivial to plug in once they have been implemented.

        \paragraph{Parallel Execution of Tests}

            In its current implementation, CrashSimulator injects faults in a serial manner. The primary cause of this
            design decision was the potential existence of dependencies between the application under test and finite
            operating system resources such as network ports, database connections, or hardware devices. Should an
            application under test have such a dependency it is non-trivial to transparently allocate this resource to
            the hundreds of instances of the application CrashSimulator could potentially launch simultaneously if it
            ran tests in parallel.

        % TODO: Verify this stuff
        \paragraph{Analysis of System Call Traces for Multi-Threaded Applications}

            Part of CrashSimulator's system call analysis is determining a global ordering for system calls based on the
            input traces. This is accomplished by making the assumption that all system calls are atomic and executed
            one at a time. This assumption does not hold in the case of multi-threaded applications. As a result, in
            these cases CrashSimulator is only able to inject faults that depend on single system calls.

        \paragraph{Testing of Applications With No Definite End Point}

            Some applications, typically daemons of some sort, are written in such a way that they will not end unless
            they are manually stopped if they encounter some fatal error condition. In these cases CrashSimulator must
            take a heuristic approach to its application monitoring post-injection. Cases where the application exhibits
            some invalid behavior immediately after injection are trivially identified. Cases where the injected fault
            causes a invalid behavior to become apparent at some undetermined future time are much more difficult to
            nail down. In these cases, CrashSimulator uses a user defined time period to kill the process and move on to
            the next test.

        \paragraph{Testing of Applications Written in Interpreted Languages}

            Because CrashSimulator operates on the system calls made by an application it does not make any attempt to
            determine what the cause of a fault may have been at a higher level. For example, in situations where
            CrashSimulator is testing an application written in an interpreted language the possibility exists that
            faults will be found in the interpreter rather than the application itself. For example, CrashSimulator may
            modify system calls made by the interpreter for purposes that are independent from the application under
            test.  If this results results in improper output CrashSimulator will simply report it as a fault in the
            application despite the fact that the user's code was not responsible for the error.
