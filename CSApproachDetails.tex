\section{CrashSimulator Approach Details}

    %%% Verify information on supported trace formats %%%
    \subsection{System Call Traces}

    The first step in CrashSimulator's operation is to gather a trace of the system
    calls made by the application during a normal run. Where other tools base their operation of direct analysis of the
    application under test CrashSimulator operates based on information gleaned from system call traces. This gives
    CrashSimulator several advantages over similar tools. First, CrashSimulator is language and platform independent.
    Because CrashSimulator's trace analysis engine is based on prior work from the NetCheck supported trace gathering
    tools include \emph{strace} on Linux and \emph{dtrace} on OS X.

    %%% How much information should be included about NetCheck's inner workings here? Should they just be referred to
    %%% the NetCheck paper?
    %%% Talk about multiple traces?
    \subsection{Trace Analysis}



    \subsection{Fault Injection}

     Once a set of potential fault injection points has been identified CrashSimulator uses
    \emph{ptrace} inject faults on a live execution of the program. At this point, two faults types identifiable by
    NetCheck are supported: modification of return values from network system calls and reordering of UDP packets.
    These faults were chosen because of high impact they can have on applications that don't handle them properly and
    the frequency with which developers have an incorrect understanding of how these system calls can behave in an
    imperfect environment.

    %% I can go into much more detail with the ptrace implementation stuff if I need to
    \subsection{Return Value Modification}

     One way CrashSimulator can inject faults into the running application is to
    modify the return values of interesting system calls identified in the previous trace analysis step. First, the
    CrashSimulator parent application is launched and configured.

        launching of parent application
        setup of child application under test
        run child application
        For for each system call determine if it is one we are interested in as determined by analysis
            e.g.\ is this the 4th call to recv?
        Modify EAX/RAX register after system call completion to contain new return value
            New return value random? Smaller than present return value? Some constant -1?
            What do we do about 32-bit vs 64-bit

    %% What is the process for culling our set of unit tests for instances where there are a huge number of permutations
    %% of packet orderings
    \subsection{Catalog-Based Anomalies}

    A second category of faults CrashSimulator can produce are known as catalog-based
    faults. These faults are injected in two steps. First, the normal run trace is parsed in its entirety for system
    calls in a specific set associated with the fault being injected and the data items passed into these system calls
    are recorded in a data item catalog. Next, the application under test is run repeatedly with each run receiving a
    different ordering of data items from the catalog for the corresponding system calls. One example of a fault
    CrashSimulator can inject in such situations is unhandled out of order UDP datagrams. Consider the following
    pseudo-code listing:

    %% Use the real C Code here
    \begin{verbatim}
        int main() {
            socket = setupUdpSocket()
            data1 = recvfrom(socket)
            processData1(data1)
            data2 = recvfrom(socket)
            processData2(data2)
        }
    \end{verbatim}

    This listing sets up a UDP socket and receives two datagrams from the socket processing each with the appropriate
    function. A C program that implements this pseudo-code will produce a normal flow system call trace as follows:

    \begin{verbatim}
        ...
        socket(PF_INET, SOCK_DGRAM, IPPROTO_UDP) = 3
        bind(3, {sa_family=AF_INET, sin_port=htons(6666), sin_addr=inet_addr("0.0.0.0")}, 16) = 0
        recvfrom(3, "test\n", 256, 0, {sa_family=AF_INET, sin_port=htons(51490), sin_addr=inet_addr("127.0.0.1")}, [16]) = 5
        ...
        write(1, "Process 1: test\n", 16)       = 16
        recvfrom(3, "testagain\n", 256, 0, {sa_family=AF_INET, sin_port=htons(51490), sin_addr=inet_addr("127.0.0.1")}, [16]) = 10
        write(1, "Process 2: testagain\n", 21)  = 21
        ...
    \end{verbatim}

    The above program assumes that datagram 1 will always arrive first and datagram 2 will always arrive second. UDP
    makes no ordering guarantees so the reverse is possible. This would result in datagram 2 being processed as datagram
    1 and vice versa. Crash simulator would inject this fault as follows. First, it would parse the system call trace
    and identify all calls to recvfrom, storing the data that was received in a data catalog and the identifying
    information pertaining to the socket it was received from. Second, it would re-run the application under test and
    send a different ordering of data from the data catalog to the socket in question. From the above example,
    ``testagain'' would be sent to the first receive from and ``test'' would be sent to the second recvfrom resulting in
    each data item being parsed by the incorrect function. CrashSimulator would then report abnormalities in the
    application's behavior. \textbf{How are we going to handle situations where this is a silent failure}

    \subsection{Limitations}

    %% This text will need to be updated based on discussion about handling testing of interpreted languages
    In situations where CrashSimulator is testing an application written in an interpreted language the possibility
    exists that faults will be found in the interpreter rather than the application itself. For example, CrashSimulator
    may modify system calls made by the interpreter for purposes that are independent from the application under test
    This could result in the interpreter itself behaving improperly...........
