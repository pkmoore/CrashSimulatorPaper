\section{Conclusion}

    Minimizing the number of faults present in the applications they write is a key concern for software developers. A
    number of automated testing techniques have been created in order to meet this goal. The most sophisticated of these
    existing techniques focus on either analyzing an applications source code or machine code in order to gain an
    understanding of the inputs required to exercise an applications code paths and then exercising them with the goal
    of exposing faults. These techniques focus on the application itself to the neglect of the application's environment
    which can be a source of numerous faults that cannot be identified by testing the application in isolation. To
    combat this problem this work presents CrashSimulator, a tool that analyzes system call traces of application in
    order to facilitate the injection of environment-based faults into the application for testing purposes.

    The goal of CrashSimulator is to allow developers to identify unhandled edge cases at the intersection of their
    application and the environment it is running in an automated fashion. CrashSimulator's ability to test an
    application in a completely black-box manner, as well as its language-agnostic approach to application analysis,
    gives considerable advantage over similar tools in the space. \textbf{NUMBER} major applications with public facing
    bug trackers were tested using CrashSimulator and CrashSimulator was able to identify \textbf{PERCENT} of bugs
    related to anomalies it tests for that had previously been located manually and reported. Additionally,
    \textbf{NUMBER} of open source projects with existing unit test suites were examined and CrashSimulator was able to
    improve test coverage in each of them by providing additional meaningful unit tests.


