\section{Conclusion} \label{sec:conclusion}
Many post-deployment failures result from
unexpected interactions between an application and its environment.
Although finding and eliminating faults in applications is a key concern for software developers, it is impractical for them to test an application in all of the
environments in which it will be deployed.
 To address this
problem, we introduced CrashSimulator, a testing
approach that utilizes mutation, replay, and analysis of system call traces
in order to determine whether an application responds correctly to
anomalous environmental conditions.
A significant benefit of CrashSimulator is that failures observed
in executing one application in an anomalous environment can
easily be leveraged to test whether {\em other} applications
suffer from the same underlying problem.

CrashSimulator's key features, operating on system calls and analysis of replayed executions allow it to identify bugs
resulting from an application's interactions with its environment.
%Performing its analysis on a replayed execution of an application allows
%CrashSimulator to remove the filesystem and network dependancies of the application under test.  
To test whether an application responds appropriately to a particular
anomaly (such as an unexpected file type, an unexpected device, an unexpected
network conditions, etc.)
CrashSimulator records system call traces from the application under test,
mutates return values and/or program state to simulate execution in 
the anomalous environment, and invokes a checker to see whether
subsequent system calls indicate a correct or incorrect response.
This allows
CrashSimulator to test an application running in one environment
as if it were running in another.

The tool evaluates application behavior in the (simulated) anomalous environment
by using
finite state models to characterize correct (or incorrect) behavior.
This provides a generic ``signature'' for that behavior, which is
portable across applications. 
Consequently, a set of mutations and checkers can be collected from any existing
application for use in testing other new or existing applications. 
In this way, an ever-expanding ``test suite'' can
be created, allowing lessons learned from bugs in one application to benefit many
others.

Our evaluation of CrashSimulator has shown it to be both effective and
efficient.  CrashSimulator was able to identify bugs related to unusual file
types in 15 applications and bugs related to slow network performance in 10
network applications and libraries.  Additionally CrashSimulator found
filesystem related bugs in 7 applications and libraries with facilities for
moving files within a system's filesystem.  The low overhead
introduced by CrashSimulator's technique meant that it was able to
find these bugs quickly in spite of its unoptimized implementation.

Future work will include expanding the repository of anomalies and their
checkers, as well as exploring opportunities to further automate the discovery
of anomalies and checkers.  In the long term, we envision a public repository of
anomalies along with CrashSimulator test patterns that can be applied to new or
existing applications.
