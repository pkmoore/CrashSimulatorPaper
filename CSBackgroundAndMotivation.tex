\section{What Is An Environment?}

Before looking at how environments differ, it is important to clearly
define what an environment is.  An environment \cappos{fill in}

There are many examples of ways in which environments differ, even in
situations where they implement a `standard interface', such as Java or
POSIX. 

\begin{itemize}

\item {\bf Operating Systems.}  A commonly observed difference is that many
applications behave in different ways due to the varying ways in which the
operating systems implement system calls.  For example, on Linux it is 
possible to remove an open file while this is 
not allowed on Windows systems.... \cappos{extend / cite...}


\item {\bf File Systems.}  The exact file system used will also have a
substantial impact on the behavior of a system, independent of the
operating system.  The popular XYZ file system on Linux is case 
sensitive, so that `a' and `A' are different files, while Linux's ABC file 
system is case insensitive so those file names would refer to the same
file.  Different file systems have varying limits or behaviors for other 
items, including file name length (popularized due to the 8.3 limitations
of the FAT file system), maximum file length, number of directory entries, 
depth of directories supported, which leads to errors when programs are not
written to take these issues into account \cappos{cites / extend}. 

\item {\bf Network.}  Aspects of the network environment also have a 
substantial impact on the behavior of application.  This can be either
local things or remote network behaviors.  For example, POSIX operating
systems support the notion of limiting the kernel buffer set aside for a
socket.  However, popular POSIX operating systems (Windows, Linux, and Mac) 
implement this quite differently.  On ... \cappos{This example is in one of
the papers, likely CheckAPI.  It deals with when datagrams are discarded
and how the buffer size is reported back to the application.}

\item {\bf Processor.}  The processor can also make a substantial
difference in the behavior of an application.  This is very frequently
evidenced through the variety of different floating point behaviors on
different processors \cappos{cites}.  However, this is not the only
difference.  Bugs in processors are fairly common and cause variance.
There are also intended differences that occur from varying interpretations
of how to execute complex instructions.  \cappos{cite if true, remove if
not.}


\end{itemize}

In this work, we focus on operating system, file system, and network
environmental issues.  We leave processor-based environmental differences
for future work because the mitigations are prohibitively expensive and
their occurrance is much less.  We focus on operating system, file system, 
and network issues, which are the bulk of enviornmental issues that
applications struggle with today.
