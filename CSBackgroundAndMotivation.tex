% TODO: Re-read and rework
\section{Background and Motivation}

    %%This sentence is very similar to the one from the Sniper paper
    This section describes the techniques and terminologies used in other automatic testing frameworks.

    \subsection{Black-box vs White-box Testing}

        There are two primary approaches to automated software testing: black-box testing and white-box testing.
        White-box testing tools require access to an applications source code in order to test it. Typically this means
        that the testing tool is able to better ``understand'' the application. This can result in more efficient and
        targeted testing. In the case of DART, the applications source code is used to generate a ``test interface'' for
        each of the functions that make up the application. SAGE takes a slightly different approach in that it works
        with compiled application code rather than source code. Its testing still falls into the white-box category
        because of the way it attempts interpret and reason about the application's machine code.

        Black-box tools work only by manipulating inputs into the application and observing outputs. Because they do not
        attempt to understand the internal workings of an application, these tools are typically less complex from an
        implementation and runtime standpoint. The price of this lack of complexity is reduced the inability to target
        specific code paths or identify where faults come from.

    \subsection{System Call Trace Analysis}

        CrashSimulator is based on prior work from two projects: NetCheck and CheckAPI. It, like these tools, analyzes
        system call traces captured from an execution of the application under test in order to determine locations that
        are likely to to be sources of faults. These traces are captured using strace and dtrace, the standard tools
        made available for this purpose in Linux and OS X respectively. Each system call is evaluated within the context
        of prior system calls, their parameters, and return values. This system call context is compared an ideal model
        and deviations from this model are logged.

    \subsection{Crash Simulation}

        Once CrashSimulator has generated a series of potential faults from its analysis of an application's system call
        trace the process of verifying the presence of faults can begin. Verification handled by running an instance of
        the application and injecting one of the potential faults identified during the analysis step.  The results of
        this faulty execution are monitored and any deviations in output from the ``correct'' output are recorded.
        Faults that result in an application crashing are also recorded.

