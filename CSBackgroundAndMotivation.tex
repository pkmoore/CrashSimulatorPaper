\section{Background and Motivation}

CrashSimulator operates in a unique niche in relation to other testing tools and techniques. As a result, it is better
suited to identify and report on certain types of errors that relate to the environment in which an application is
running.

    \subsection{Existing Techniques}

    Existing tools can be roughly divided into two categories, black-box and white-box, based on the techniques they
    use to perform their testing. Black-box tools simply manipulate the inputs of the application under test and
    observe the resultant outputs. White-box tools, on the other hand, perform complex analysis of the application's
    source code in order to reason about what inputs are likely to produce interesting outputs. Each of these
    methodologies have their own advantages and disadvantages.

    White-box testing tools typically rely on a similar set of techniques, including constraint solving of branch
    statements in an application's code and symbolic execution of an application's code in order to generate inputs
    that optimally exercise the application's code paths. These techniques, while powerful, are not without their
    downsides. First, both techniques are computationally-expensive. Furthermore, symbolic execution can not always
    accurately represent actual execution and so there may be deviations in results. Similarly, efficiently solving
    a series of constraints in order to exercise a particular code path can be can be difficult to guarantee that a
    particular set of generated inputs will exercise the intended code path in many circumstances due to external
    dependencies that the tool cannot analyze. For example, a white-box testing tool cannot reliably generate inputs
    that are guaranteed to exercise a code path that relies on an operating system resource being available.
    Finally, white-box tools typically require that an application's source code be available which is not always
    the case. Even advanced white-box tools that analyze an application's machine code can be stymied in situations
    where an application's executable has been packed or encrypted.

    The alternative, black-box tools, have their own set of issues. They do not have an understanding of what an
    application is actually doing during execution which means they are only able to submit inputs and observe
    outputs.  The upside of this technique is simplicity. Black-box tools do not need the capability to understand
    and analyze an application's code which reduces their complexity immensely. Also, their testing process,
    mutating inputs and observing outputs, is computationally inexpensive. The downside of simplicity is that they
    cannot craft inputs with any sort of intelligence. This means that a great deal of time can be spent mutating
    inputs without much success in terms of bug identification. Also, they cannot identify specifically the source
    of faults in an application. They can only signal that a fault has occurred at some point during a test run.
    Furthermore, like white-box tools, these tools fail to take into account the environment in which the
    application is running.

    \paragraph{What can a test tell us?}

    In short, a test can only tell is if an application performs a specific bad behavior in a specific situation.  This
    is in opposition to telling us that an application is bug free in a given situation.  With this in mind, the usual
    approach is to construct a large enough test suite that some desired degree of coverage is achieved.  CrashSimulator
    operates according to this approach as well.  Each injected anomaly can be considered a test case and a set of
    anomalies can be considered a test suite.  The advantage CrashSimulator has over other tools is that these ``tests''
    can be utilized across a variety of applications whereas a traditional test suite is closely coupled to the
    application it tests.  As with traditional tests, CrashSimulator cannot tell us that an applcation is bug free in a
    given situation -- instead it can assert that an application has incorrectly handled a given situation based.

    \subsection{What is an ``Environment''}

    An application's environment is the collection of all resources external to the application with which the
    application communicates.  These external resources can be thought of as an implicit input to the program as their
    presence and contents alter the flow of any application executed in there presence.  This work is concerned with bugs
    that arise from subtle differences in behavior between executions in one environment versus another. Specifically, the
    files and network communications visible through the monitoring the system calls the application makes during the course
    of its execution.  Consider the following hypothetical example: an application utilizing the POSIX API is executed on a
    Linux machine and an OS X machine.  Because the API's called in each situation superficially act the same, a developer might
    make the assumption that they are identical in all regards -- an assumption that will result in bugs when in cases where
    it does not hold.

    Another interesting case is the situation where the actual API's in question work correctly but the resources they
    operate on do not.  Consider the case where an application is executed on two different hosts with the same environment
    present locally but connected two different networks.  In this case, differences in the network itself can cause issues
    with the application's execution.  For example, if the application assumes that it will receive messages of a particular
    length from a network host it will not execute correctly when messages of that length are dropped, or fragmented, by a
    intermediary host with strict MTU requirements. \emph{Do I need to define/describe MTU here? -Preston}

    \subsection{A Real World Example}

    During the course of its evaluation, NetCheck was able to identify a bug in Python's socket handling code that
    compromised portability of Python applications. In short, calling \emph{accept} on a socket that was created in
    non-blocking mode would return a new socket in blocking mode on Linux where the same calls would return a
    non-blocking socket on OS X, Windows, and various flavors of BSD\@. This caused code that assumed the return
    socket would always be blocking to work correctly on Linux but fail when an unhandled EWOULDBLOCK error was
    encountered on the non-blocking socket under OS X, Windows, and BSD\@. This fault was a source of error for
    several major Python applications. Many put in place work-arounds without knowing the actual source of the
    problems.  This type of environment-related error is exactly the type of fault that CrashSimulator was designed
    to identify.

    Because NetCheck was able to identify this anomaly in Python, CrashSimulator is able to produce mutated system
    call traces that can induce it in other applications. NetCheck identifies situations where a socket received
    from a call to \emph{accept} has inherited the non-blocking flag from the socket initially passed into
    \emph{accept}. This anomaly will be recored and during future test runs CrashSimulator will be able to produce
    mutated system calls containing when it encounters a call to accept on a non-blocking socket in a system call
    trace taken from the application under test.
