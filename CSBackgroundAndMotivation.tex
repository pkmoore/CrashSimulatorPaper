\section{What Is An Environment?}

Before looking at how environments differ, it is important to clearly
define what an environment is.
An application's environment is the collection of all resources
external to the application with which the application communicates.
These external resources can be thought of as providing implicit
inputs to the program that affect its flow of execution.
%One way in which applications can interact with their environment is by
%making requests in the form of system calls and receiving responses in the
%form of the results and side effects generated by these system calls.  From an
%application's perspective, these interactions comprise the small differences
%that make a particular environment unique. Our approach supports systems that
%make use of the system call pattern and takes advantage of this
%differences-property to subject an application to the unusual attributes of a
%particular environment.  Specifically, our approach can simulate environmental
%anomalies that are visible in an application's system call interactions as
%opposed to environmental conditions that are visible to the application in
%other ways, such as direct influence on values present in the application's
%memory.

There are many examples of ways in which environments differ, even in
situations where they implement a ``standard interface'', such as Java or
POSIX.

\begin{itemize}

\item {\bf Operating Systems.}  A commonly observed difference is that many
applications behave in different ways due to the varying ways in which the
operating systems implement system calls.  For example, on Linux it is 
possible to remove an open file while this is 
not allowed on Windows systems~\cite{UnlinkStandard}.  An application written
without this difference in mind could fail should if it relies on one
implementation or the other.

\item {\bf File Systems.}  The exact file system used will also have a
substantial impact on the behavior of a system, independent of the
operating system.  The popular Ext4 file system on Linux is case 
sensitive, so that ``a'' and ``A'' are different files, while OS X's HFS+ file 
system is case insensitive so those file names would refer to the same
file.  Different file systems have varying limits or behaviors for other 
items, including file name length (popularized due to the 8.3 limitations
of the FAT file system), maximum file length, number of directory entries, 
depth of directories supported, which leads to errors when programs are not
written to take these issues into account~\cite{EXT4Layout, AppleHFS}.

\item {\bf Network.}  Aspects of the network environment also have a 
substantial impact on the behavior of application.  This can be either
local things or remote network behaviors.  For example, POSIX operating
systems support the notion of limiting the kernel buffer set aside for a
socket.  
However, popular operating systems (Windows, Linux, and Mac) 
implement this quite differently.  For example, in Linux, if a UDP datagram larger than the
specified buffer size is received it will be dropped.  In Windows, however, UDP
datagrams larger than the specified buffer size are received but system calls
that retrieve data from the buffer they are stored in will only return a number
of bytes less than or equal to the buffer size at one time~\cite{Zhuang_NSDI_2014}.

\item {\bf Processor.}  The processor can also make a substantial
difference in the behavior of an application.  This is very frequently
evidenced through the variety of different floating point behaviors on
different processors~\cite{ArbitraryPrecision}.  However, this is not the only
difference.  Bugs in processors are fairly common and cause variance.
There are also intended differences that occur from varying interpretations
of how to execute complex instructions~\cite{Microarch}.

\end{itemize}

In this work, we focus on operating system, file system, and network
environmental issues which are the bulk of environmental issues that
applications struggle with today. Specifically, we are concerned with issues
from the above sources that are visible in the results and side effects of the
system calls an application makes.  We leave processor-based environmental
differences and differences not visible in system call activity for future efforts.


