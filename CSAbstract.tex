\begin{abstract}

  One of the challenges of software development is anticipating and dealing with the variety of environments in which
  an application may run. For example, an application may work correctly on the developer's low latency, local network
  but fail when it communicates across a network that requires data to be fragmented across multiple packets. In this
  case, the application was likely written to expect complete messages in a single packet when it should have been
  written to handle fragmentation, a common phenomenon. These types of failures are difficult anticipate and test for
  without deploying the application to each target environment. This paper introduces CrashSimulator, a tool designed to
  determine whether or not an application can handle these, and other, unanticipated conditions as part of the
  application's normal testing process.  This is accomplished in two steps. In the first step, system call traces are
  recorded for applications running in the ``problematic'' environment. These system call traces are analyzed and
  anomalous system call behaviors are identified. These anomalies are encoded as deterministic computational models. In
  the second step, the an execution of the application under test is evaluated using these models in order to determine
  whether or not the application correctly handles the anomalous condition.  In this way, CrashSimulator allows the
  application to be tested as if it were running in one or more of the environments from which anomalous traces were
  collected. As a result, the developer is made aware of deficiencies in the application that may result in it operating
  incorrectly in the environment from which the anomalous traces were taken. CrashSimulator was used to analyze
  \emph{<<somenumber>>} commonly used open and closed source applications. Its Findingsfindings were compared to tickets
  listed in the bug tracking suites of these applications.  It was able to successfully identify \emph{<<somenumber>>} of
  bugs listed in these suites and \emph{<<someothernumber>>} of bugs that had not yet been discovered.

\end{abstract}
