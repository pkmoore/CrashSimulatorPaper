\begin{abstract}

    One of the challenges of software development is anticipating and dealing with the variety of environments in which
    an application may run. For example, an application may work correctly on the developer's low latency, local network
    but fail when it communicates across a node that performs network address translation or  a link that introduces
    high latency into network communications. This paper introduces CrashSimulator, a tool designed to determine whether
    or not an application can handle these, and other,  unanticipated conditions.  CrashSimulator accomplishes this in
    two phases. First, is the anomaly identification phase. In this phase, CrashSimulator's anomaly identification tools
    analyze system call traces from existing applications that are running in a given environment with the purpose of
    identifying anomalous behaviors or conditions. These anomalies are collected an anomaly store for later during the
    testing phase. In the testing phase, the anomalies stored in the anomaly store are used to mutate a system call
    trace taken from the application under test running under normal circumstances in order to introduce into it one or
    more of the anomalous behaviors identified previously. Once a set of mutated traces has been generated the
    application under test is run repeatedly with each run replaying one of the mutated system call traces. These test
    runs are monitored and any faults in the application are reported. As a result, the developer is made aware of
    deficiencies in the application that may result in it operating incorrectly in the environment from which the
    anomalous traces were taken. This allows the application to be tested as if it were running in one or more of the
    environments from which anomalous traces were collected. CrashSimulator was used to analyze \emph{<<somenumber>>}
    commonly used open and closed source applications.  Its findings were compared to tickets listed in the bug tracking
    suites of these applications.  It was able to successfully identify \emph{<<somenumber>>} of bugs listed in these
    suites and \emph{<<someothernumber>>} of bugs that had not yet been discovered.

\end{abstract}
