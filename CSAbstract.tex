\begin{abstract}
  A major problem when introducing new applications is that
  the environment
  into which an application is introduced can be a source of flaws that remain
  hidden until after a costly deployment.
  CrashSimulator is a technique and tool that identifies incorrect application
  behavior caused by unhandled anomalous environmental conditions such as
  differing API behavior, network performance, or filesystem state.
  The tool replays system call traces recorded from the application under
  test as a means to monitor and control execution. During replay,
  CrashSimulator modifies system call return values and program states to
  simulate an anomalous environment. Next, a checker determines whether the
  application is responding to the anomaly appropriately. This process allows
  CrashSimulator to use incorrect behavior identified in one application running
  in one environment to determine whether other applications
  would make the same mistake in that environment. By testing an
  application against a series of these test cases, CrashSimulator 
  lets the developer see how the application would respond if executed in
  a similar environment.
  We evaluated CrashSimulator by using it to test popular applications such as
  vim, nano, aspell, and others selected from the Coreutils project and the
  Debian popularity contest. The result of this evaluation was 84 bugs 
  identified in 30 applications with consequences ranging from hangs and
  crashes to data loss and remote denial of service conditions.
  %All but one of the bugs identified by CrashSimulator were previously unknown.
  %One was independently identified and reported to Python's bug tracker.

\end{abstract}
