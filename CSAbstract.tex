\begin{abstract}
  A major problem when introducing new applications is that not all potential
  flaws can be identified during development. Unfortunately, the environment
  into which an application is introduced can negatively impact its
  execution. But, these flaws remain hidden until the application is
  deployed—and the resulting consequence can be costly. 
  CrashSimulator is a technique and tool that identifies incorrect application
  behavior caused by unhandled anomalous environmental conditions such as
  differing API behavior, network performance, or filesystem state.
  The tool replays system call traces recorded from the application under
  test as a means to monitor and control execution. During replay,
  CrashSimulator modifies system call return values and program states to
  simulate an anomalous environment. Next, a checker determines whether the
  application is responding to the anomaly appropriately. This process allows
  CrashSimulator to use incorrect behavior identified in one application running
  in one environment to determine whether another application
  in a different environment is making the same mistake. By testing an
  application against a series of these test cases, CrashSimulator 
  lets the developer see how the application would respond if executed in
  a similar environment. When tested, CrashSimulator was
  able to identify 84 bugs resulting from anomalous filesystem and network
  conditions in 30 popular applications.  By identifying these bugs before
  deployment, the tool helped to ward off such potential damaging consequences as hangs,
  data loss, or remote denial of service conditions
  All but one of the bugs identified by CrashSimulator were previously unknown.
  One was independently identified and reported to Python's bug tracker.

\end{abstract}
