\begin{abstract}

    One of the challenges of software development is anticipating and dealing with the variety of environments in which
    an application may run. For example, an application may work correctly on the developer's low latency, local network
    but fail when it communicates across a node that performs network address translation or a link that introduces high
    latency into network communications. These types of failures are difficult anticipate and test for without deploying
    the application to each target environment. This paper introduces CrashSimulator, a tool designed to determine
    whether or not an application can handle these, and other, unanticipated conditions as part of its normal testing
    process. CrashSimulator accomplishes this in two phases. First, its anomaly identification tools analyze system call
    traces taken from the application's target environment with the purpose of identifying irregular (or unusual)
    behaviors. These anomalies are used to mutate a system call trace taken from the application under test when it was
    running under normal circumstances. This process generates mutated system call traces that each contain one of the
    previously identified anomalies. These traces are replayed in test runs of the application and any faults that occur
    are reported. In this way, CrashSimulator allows the application to be tested as if it were running in one or more
    of the environments from which anomalous traces were collected. As a result, the developer is made aware of
    deficiencies in the application that may result in it operating incorrectly in the environment from which the
    anomalous traces were taken. CrashSimulator was used to analyze \emph{<<somenumber>>} commonly used open and closed
    source applications. Its findings were compared to tickets listed in the bug tracking suites of these applications.
    It was able to successfully identify \emph{<<somenumber>>} of bugs listed in these suites and
    \emph{<<someothernumber>>} of bugs that had not yet been discovered.

\end{abstract}
