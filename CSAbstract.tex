\begin{abstract}

  One of the challenges of software development is anticipating and dealing with the variety of environments in which an
  application may run.  This work introduces CrashSimulator, a technique and tool that identifies incorrect application
  behavior which has arisen as a result of unhandled anomalous environmental conditions.  CrashSimulator replays system
  call traces recorded from the application under test as a means to monitor and control execution.  During these replay
  executions, CrashSimulator classifies application behavior as either correct or incorrect based on ``checkers'' that
  encode system call sequences indicative of bad behavior. This process allows CrashSimulator to use incorrect behavior
  identified in one application running in one environment as a ``test case'' that determines whether another
  application in a different environment is making the same mistake.  By testing an application against a series of
  these ``test cases'', CrashSimulator is able to give an application's developers an idea of how the application will
  respond were it to be deployed and executed in the environment from which the ``test cases'' were generated.''
  CrashSimulator was able to identify bugs resultant from anomalous filesystem and network condtions in 30 popular Linux
  applications.

\end{abstract}
