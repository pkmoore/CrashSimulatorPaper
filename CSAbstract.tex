\begin{abstract}

    One of the challenges of software development is anticipating and dealing with the variety of environments in which
    an application may run. For example, an application may work correctly on the developer's low latency, local network
    but fail when run across a node that performs network address translation or across a link that introduces high
    latency into network communications. This paper introduces CrashSimulator, a tool designed to determine whether or
    not an application can handle these unanticipated conditions.  CrashSimulator accomplishes this in three steps.
    First, it analyzes a system call trace of an application collected using a utility like \emph{strace} in order to
    identify areas where anomalies could potentially be injected. Next, it produces a new system call trace with the
    anomalies present. It then evaluates whether or not the application properly handles the anomaly replaying the
    mutated system call trace. Finally, it observes whether or not the application continued to operate normally after
    the encountering the anomaly and reports its findings to the developer. As a result, the developer is made aware of
    deficiencies in the application without having to deploy the application to other environments for the purpose of
    testing. CrashSimulator was used to analyze \emph{<<somenumber>>} commonly used open and closed source applications.
    Its findings were compared to tickets listed in the bug tracking suites of these applications.  It was able to
    successfully identify \emph{<<somenumber>>} of bugs listed in these suites and \emph{<<someothernumber>>} of bugs
    that had not yet been discovered.

\end{abstract}
