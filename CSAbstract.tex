\begin{abstract}

  This work introduces CrashSimulator, a technique and tool that identifies
  incorrect application behavior caused by unhandled anomalous environmental
  conditions such as differing API behavior, network performance,
  or filesystem state.
  CrashSimulator replays system call traces recorded from the
  application under test as a means to monitor and control execution. During
  replay, CrashSimulator modifies system call return values and program states to
  simulate an anomalous environment. Next, a checker determines whether the
  application is responding to the anomaly appropriately. This process allows
  CrashSimulator to use incorrect behavior identified in one application running
  in one environment as a test case that determines whether another application
  in a different environment is making the same mistake. By testing an
  application against a series of these test cases, CrashSimulator shows an
  application's developer how the application would respond were
  it executed in the environment from which the test cases were
  generated. CrashSimulator was able to identify 84 bugs resulting from anomalous
  filesystem and network conditions in 30 popular applications. The damage
  caused by these bugs ranges from application hangs to data loss and remote
  denial of service conditions. While one of the bugs we found was independently
  identified and reported to Python's bug tracker, the rest had not yet been
  discovered.

\end{abstract}
