\begin{abstract}
%\boldmath

    % What does "correctly" mean in general? % Correct -> does the program behave the same as in the happy path when it
    % encounters an error. % Too specific on the what we are breaking, only return values. Could be misordered UDP.  Be
    % more general about the faults we are trying to do.  Intro sentence -> may programmers have programs that work fine
    % in development environment and then fail in production % We are able to use this tool to induce and observe
    % behavior without having to go through deployment % Introduction has to have a story like this % Motivation about
    % problem, summary about contribution, more detail about contribution %

    One of the challenges of software development is anticipating and dealing with the differences in the variety of
    environments in which an application may run. For example, an application may work correctly on the developer's low
    latency, local network but fail when run across a node that performs network address translation or across a link
    that introduces high latency into network communications. This paper introduces CrashSimulator, a tool designed to
    induce crashes in applications that do not properly handle these unanticipated conditions.  CrashSimulator
    accomplishes this in three steps. First, it analyzes a system call trace of an application collected using a utility
    like \emph{strace} in order to identify a set of system calls that are commonly mishandled by developers. Next, it
    injects faults into the running application at the previously identified points by using \emph{ptrace}. Finally, it
    observes whether or not the application continued to operate normally after the encountering the error and reports
    its findings to the developer. As a result, the developer is made aware of deficiencies in the application without
    having to go through a potentially complex and costly deployment or beta test scenario. Because CrashSimulator
    operates using black-box analysis utilities it does not require access to an application's source code.
    CrashSimulator was used to analyze \emph{<<somenumber>>} commonly used open and closed source applications.  Its
    findings were compared to tickets listed in the bug tracking suites of these applications.  It was able to
    successfully identify \emph{<<somenumber>>} of bugs listed in these suites and \emph{<<someothernumber>>} of bugs
    that had not yet been discovered.

\end{abstract}
